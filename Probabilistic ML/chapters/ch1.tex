
\chapter{Introduction}

\section{Models and Probability}

Machine learning is a field of computer science about \textbf{learning models}.

\begin{definitionblock}[Machine learning]
    \textit{\textbf{Machine learning} explores the study and construction of algorithms that can learn from and make predictions on data. \hfill \~ Wikipedia}
\end{definitionblock}


If we dig into this statement, we may wonder what precisely do ML algorithms learn.

The answer to this question is (apparently) simple: they learn models of
the observed data. Models that can then be used to make predictions or
to extract information from such data.

\begin{definitionblock}[Model]
    \begin{itemize}
        \item A \textbf{\textit{Model}} is a hypothesis that certain features of a system of interest are well replicated in another, simpler syetem.
        
        \item \textbf{\textit{Mathematical Model}} is a model where the simpler system consists of a set of mathematical relations between objects (equations, inequalities, etc).
        
        \item A \textbf{\textit{Stochastic Model}} is a mathematical model where the objects are probability distributions.
    \end{itemize}
\end{definitionblock}

All modelling usually starts by defining a family of models indexed by some parameters, which are tweaked to reflect how well the feature of interest are replicated.

Machine learning deals with algorithms for automatic selecrion of a model from observations of the system.

\subsubsection{Generative and Discriminative Learning}

\begin{itemize}
    \item \textbf{Generative Learning} im at describing the full probability distribution of inputs $x$ or input/output pairs $(x, y)$.
    $$
    p(x,y) = p(x)p(y|x)
    $$

    \item \textbf{Discriminative Learning} aims at describing the conditional probability of output given the input, or a statistics/function of such probability
    $$
    p(y|x) \quad or \quad y = f(x)
    $$
\end{itemize}

[to fix:]
\begin{itemize}
    \item \textbf{Supervised Learning}: The algorithm learns from labeled data by mapping inputs to outputs.
    \item \textbf{Unsupervised Learning}: The algorithm identifies patterns or structures in unlabeled data.
    \item \textbf{Data Generatiion}: The algorithm generates new data points.
\end{itemize}

\subsubsection{Inference and Estimation}

Two central concepts for probabilistic machine learning are:

\begin{itemize}
    \item \textbf{Inference}: Compute marginals and contitionals probability distributions applying the laws of probability. 
    \item \textbf{Estimation}: Given data and a family of models, find the best parameters/models that explains the data.
\end{itemize}

In the Bayesian world: estimation \approx \ inference.

\subsubsection{Probability}

\textbf{Probability} is a mathematical theory that deals with \textbf{uncertainty}

When a certain problems has to face practical difficulties due to it's complexity, we can use probability to model the \textbf{\textit{aleatorical uncertainty}}, which is the uncertainty due to the randomness of the system. 

More often, we have a limited knowledge of the system, and we can use probability to model the \textbf{\textit{epistemic uncertainty}}, which is the uncertainty due to the lack of knowledge.

\begin{tipsblock}[Everything is a probability distribution]
    In machine learning \textbf{everything is a probability distribution}, even if not explicitly stated.
\end{tipsblock}

\section{Probability basics}

\subsection{Random Variables}

\textbf{\textit{Random Variables}} are functions mapping outcomes of an experiment to real numbers. They serve as abstract representations of the outcomes in randomized experiments. Note that what we observe are the \textit{realizations} (values resulting from an observed outcome) of these random variables.

\begin{exampleblock}[Random Variable]
Consider the following example:
$$
\{Head, Tail\}, \quad \quad
\{0, 1\}, \quad \quad
\left\{\tfrac{1}{2}, \tfrac{1}{2}\right\}.
$$
Only the second is the random variable itself; the third is its probability distribution, while the first is the sample space of potential outcomes.
\end{exampleblock}

We consider a \textbf{\textit{Sample Space}} $\Omega$, which is the set of all possible outcomes of a random experiment. A random variable $X$ is a function:
$$
X: \Omega \rightarrow E, \qquad \text{where } E \subseteq \mathbb{R} \quad (\text{or } E \subseteq \mathbb{N})
$$
with the probability measure 
$$
P(X \in S) = P(\{\omega \in \Omega \mid X(\omega) \in S\}), \quad S \subseteq E.
$$

A model for our random outcome is the probability distribution of $X$. In particular, if the sample space is finite or countable the \textbf{probability mass function (pmf)} is given by:
$$
p(x) := P(X=x).
$$
If the sample space is infinite, we use the \textbf{probability density function (pdf)} where
$$
P(a \leq X \leq b) = \int_a^b p(x)dx \quad \text{and} \quad \int_{\mathbb{R}} p(x)dx=1.
$$

\subsection{Notable Probability Distributions}

Below are some of the most common probability distributions.

\subsubsection*{Discrete Distributions}
\begin{center}
\small
\renewcommand{\arraystretch}{1.5} % increase row spacing
\begin{tabular*}{\textwidth}{@{\extracolsep{\fill}} l l l l }
\toprule
Distribution & pmf & Mean & Variance \\
\midrule
Binomial $\text{Bin}(n,p)$ & $\displaystyle \binom{n}{x} p^x (1-p)^{n-x}$ & $np$ & $np(1-p)$ \\
Bernoulli $\text{Bern}(p)$ & $\displaystyle p \quad (x=1), \quad 1-p \quad (x=0)$ & $p$ & $p(1-p)$ \\
Discrete Uniform $\mathcal{U}(a,b)$ & $\displaystyle \frac{1}{b-a+1}$ & $\displaystyle \frac{a+b}{2}$ & $\displaystyle \frac{(b-a+1)^2-1}{12}$ \\
Geometric $\text{Geom}(p)$ & $\displaystyle (1-p)^{x-1}p$ & $\displaystyle \frac{1}{p}$ & $\displaystyle \frac{1-p}{p^2}$ \\
Poisson $\text{Pois}(\lambda)$ & $\displaystyle \frac{\lambda^x e^{-\lambda}}{x!}$ & $\lambda$ & $\lambda$ \\
\bottomrule
\end{tabular*}
\end{center}

\subsubsection*{Continuous Distributions}
\begin{center}
\small
\renewcommand{\arraystretch}{2.5} % increase row spacing
\begin{tabular*}{\textwidth}{@{\extracolsep{\fill}} l l l l }
\toprule
Distribution & pdf & Mean & Variance \\
\midrule
Continuous Uniform $\mathcal{U}(a,b)$ & 
$\displaystyle
\begin{cases}
\frac{1}{b-a} & x \in [a,b]\\[0.5em]
0 & \text{otherwise}
\end{cases}$ 
& $\displaystyle \frac{a+b}{2}$ & $\displaystyle \frac{(b-a)^2}{12}$\\[1.5ex]
Exponential $\text{Exp}(\lambda)$ & $\displaystyle \lambda e^{-\lambda x}$ & $\displaystyle \frac{1}{\lambda}$ & $\displaystyle \frac{1}{\lambda^2}$\\[1.5ex]
Gaussian $\mathcal{N}(\mu,\sigma^2)$ & 
$\displaystyle \frac{1}{\sigma\sqrt{2\pi}} \exp\!\Biggl(-\frac{1}{2}\Bigl(\frac{x-\mu}{\sigma}\Bigr)^2\Biggr)$ 
& $\mu$ & $\sigma^2$\\[1.5ex]
Beta $\text{Beta}(\alpha,\beta)$ & 
$\displaystyle \frac{x^{\alpha-1}(1-x)^{\beta-1}}{B(\alpha,\beta)}$ 
& $\displaystyle \frac{\alpha}{\alpha+\beta}$ & $\displaystyle \frac{\alpha\beta}{(\alpha+\beta)^2(\alpha+\beta+1)}$\\[1.5ex]
Gamma $\text{Gamma}(\alpha,\beta)$ & 
$\displaystyle \frac{\beta^\alpha}{\Gamma(\alpha)} x^{\alpha-1}e^{-\beta x}$ 
& $\displaystyle \frac{\alpha}{\beta}$ & $\displaystyle \frac{\alpha}{\beta^2}$\\[1.5ex]
Dirichlet $Dir(\alpha)$ & 
$\displaystyle \frac{1}{B(\alpha)} \prod_{i=1}^{K}x_i^{\alpha_i-1}$ 
& $\tilde{\alpha}_i$ & $\displaystyle \frac{\tilde{\alpha}_i (1-\tilde{\alpha}_i)}{\alpha_0+1}$ \\[1.5ex]
Student's t $St(\nu)$ & 
$\displaystyle \frac{\Gamma\Bigl(\frac{\nu+1}{2}\Bigr)}{\sqrt{\nu\pi}\;\Gamma\Bigl(\frac{\nu}{2}\Bigr)}\Bigl(1+\frac{x^2}{\nu}\Bigr)^{-\frac{\nu+1}{2}}$ 
& $0$ & 
$\displaystyle \begin{cases}
\frac{\nu}{\nu-2} & \nu>2 \\[0.5em]
\infty & 1<\nu\leq2 
\end{cases}$\\
\bottomrule
\end{tabular*}
\end{center}
  
\subsubsection*{Notes:}
\begin{itemize}
    \item For discrete distributions, $n\in\{0,1,2,\dots\}$, $p\in[0,1]$, and $x$ runs over the support.
    \item For continuous distributions, parameters such as $\lambda$, $\mu$, $\sigma$, $\alpha$, and $\beta$ belong to $\mathbb{R}$ (with appropriate restrictions) and $x\in\mathbb{R}$.
    \item In the Dirichlet distribution, $\tilde{\alpha}_i=\frac{\alpha_i}{\sum_{h=1}^K\alpha_h}$ and $\alpha_0=\sum_{i=1}^K \alpha_i$.
    \item For Student's t-distribution, $\nu>1$.
\end{itemize}
