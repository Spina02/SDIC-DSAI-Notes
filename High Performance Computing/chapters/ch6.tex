\chapter{Lecture 29/04/2025}

% TODO: adapt this part

As we have seen when we discussed the modern architectures, a unique
central memory with a fixed bandwidth would be a major bottleneck in a
system with a fast growing number of cores/sockets and sockets.
The problem is avoided by physically disjointing the memory in separated units
(the memory banks) each of which is connected to a socket; All the sockets are
inter-connected so that each core can access all the memory and a cache-
coherency system “glues” the data.
This way, the resulting aggregated bandwidth scales as the number of sockets
(although, we know, the cache-coherency becomes the new limiting factor).
However, the major drawback is that the access time is no more uniform.
This has severe consequences on how you have to write and run your codes.

OpemMP and OS offer the capacity to decide where each thread have to tun, i.e. on which core and/or how the threads have to distribute on the available cores.

We know that each core may have the capability of running more than
one thread, which is called (∗) Simultaneous MultiThreading (SMT).
In the next slides, let’s call strands or hwthreads (hardware threads) the
different threads that a physical core could run, as opposed to “swthreads”
(software threads) that indicates the OpenMP threads.

The placement of OpenMP threads on cores is called “threads affinity”.

%TODO: -------------------------------------------------

\missing{part to "Threads affinity - PLACES"}

\subsubsection{Places}

To pass OpemMP the information about the placement of threads, we have to use the environment variable \texttt{OMP\_PLACES}:

\begin{codeblock}[language=bash]
export OMP_PLACES = { sockets | cores | threads }

# examples:
    # 1 socket
    export OMP_PLACES = “{0}:4:12
    # 2 sockets
    export OMP_PLACES = “{0:11,48:11},{24:12,72:12}
\end{codeblock}

\missing{part to "Threads affinity - BIND"}

\subsubsection{Binding}

\textbf{Binding} defines how the OpenMP software threads (swthreads) are mapped onto hardware \textit{places}.

\subsubsection{Binding Types}

\begin{itemize}
    \item \textbf{NONE}: Placement is left up to the operating system (OS).
    \item \textbf{CLOSE}: Threads are placed as close as possible to each other (assigned in a round-robin way).
    \item \textbf{SPREAD}: Threads are distributed as evenly as possible, then filled in round-robin.
    \item \textbf{MASTER}: Threads run on the same place as the master thread.
\end{itemize}

\subsubsection{Using \texttt{OMP\_PROC\_BIND}}

You can specify binding in OpenMP using the environment variable:

\begin{codeblock}[language=bash]
export OMP_PROC_BIND = { false | true | master | close | spread }
\end{codeblock}

\begin{itemize}
    \item \texttt{false}: No placement policy; OS decides and may migrate threads.
    \item \texttt{true}: No placement policy; OS decides but cannot migrate threads.
    \item \texttt{master}, \texttt{close}, \texttt{spread}: Specify exact placement; OS cannot migrate threads.
\end{itemize}