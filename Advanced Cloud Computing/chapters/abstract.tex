\chapter*{Preface}

As a student of Scientific and Data Intensive Computing, I've created these notes while attending the \textbf{Advanced Cloud Computing} - module of \textit{HPC and Cloud Computing} course - to deepen my understanding of modern cloud infrastructure. This document represents my journey through the advanced concepts of cloud computing, focusing particularly on containerization and orchestration at scale.

The course material I've documented is structured around two main sections:
\begin{itemize}
    \item \textbf{Foundation Technologies:} My exploration into the core technologies that make containerization possible, including Linux kernel namespaces, resource limitations, networking, and layered filesystems. Through hands-on experiments, I've worked to understand how these components create isolated environments for applications.
    \item \textbf{Orchestration with Kubernetes:} My practical experience with Kubernetes, documenting the essential resources needed for deploying applications at scale. I've covered topics from basic concepts like replica management to more complex subjects such as deployment strategies, autoscaling, and pod networking with various CNI solutions like Calico, Flannel, and Cilium.
\end{itemize}

While these notes were primarily created for my personal study, they may serve as a valuable resource for fellow students and professionals interested in cloud computing.

The content focuses on both theoretical foundations and practical implementations, documenting the challenges and solutions I've encountered while learning these advanced concepts. My goal in sharing these notes is to provide a practical perspective on cloud computing from a student's point of view, hopefully making these complex topics more approachable for others who are on a similar learning path.