\chapter{Introduction}

\begin{definitionblock}[High Performance Computing]
    High Performance Computing (HPC) is the use of servers, clusters and supercomputers - plus associated software, tools, components, storage and services - \textbf{for scientific, engineering or AI tasks} that are particularly intense in computation, memory usage or data management.
\end{definitionblock}

As models become more complex and new data bring in more information, we require ever increasing computational power to solve these problems in a reasonable amount of time. This is where High-Performance Computing comes in.

Insert image...

Organizations are explaning their definitions of HPC to include worlloads such as AI and high-performance data analytics (HPDA) in addition to traditional HPC simulation and modeling workloads. 

Its elements include: 
\begin{itemize}
    \item \textbf{Hardware:} servers, clusters, supercomputers, storage, networking, etc.
    \item \textbf{Software:} compilers, libraries, tools, applications, etc.
    \item \textbf{Services:} consulting, training, support, etc.
    \item \textbf{People:} users, administrators, developers, etc.
    \item \textbf{Data:} input, output, storage, management, etc.
\end{itemize}

\textbf{Human capital} is by far the most important aspect of HPC. The best hardware and software in the world is useless without skilled people to use it.
\begin{itemize}
    \item HPC providers need to provide training and support to users.
    \item HPC users need to be trained and supported to use the hardware and software effectively.
\end{itemize}

\subsection*{What about performance and metrics?}

\textbf{Performance} is the most important metric in HPC. It is the measure of how well a system is performing a given task. It is usually measured in terms of time to solution, which is the time it takes to complete a given task. But, it is not always what matters$\dots$ to reflect a greater focus on the productivity, rather than just the performance, of HPC systems, the term \textbf{Productivity} is often used.
\[
    \text{Productivity} = \frac{\text{Application Performance}}{\text{Application Effort}}
\]

\begin{itemize}
    \item \texttt{How fast can I do things on my CPU?}
    
    We count the number of operations per second that a CPU can perform. This is called \textbf{FLOPS} (Floating Point Operations Per Second). The theoretical peak performance is determined by the number of cores, the clock speed, and the number of operations per cycle.
    \[
        \text{FLOPS} = \text{Cores} \times \text{Clock Speed} \times \text{Operations per Cycle}
    \]
    It is not easy to be defined for real applications, so benchmarks are used to measure the performance of a system. The TOP500 list is a list of the 500 most powerful supercomputers in the world, ranked by their performance on the LINPACK benchmark. The LINPACK benchmark measures the performance of a system solving a system of linear equations. See \url{https://www.top500.org/}.
    \item \texttt{How fast can I move data around?}
    

    \item \texttt{How much data can I store?}
    

\end{itemize}


