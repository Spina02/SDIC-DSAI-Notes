\newpage

\missing{Lecture 21/10/2025}

\chapter{Lecture 23/10/2025}

\missing{Start of the lecture}

% Mass of $E_s$:

% \begin{enumerate}
%     \item \textbf{Kinematic of stars}: 
%     \item \textbf{Hot gas}: $T_x \rarr M$
% \end{enumerate}

\section{Spiral Galaxies}

\begin{center}
\begin{tabular}{l|cccc}
    \toprule
    & $S_a$ & $S_b$ & $S_c$ & $S_d$ \\
    \midrule
    $M_B$ & $-17 - 23$ & $-17 - 23$ & $-16 -22$ & $-15 -20$\\
    $M/M_{\odot}$ & $10^{9-12}$ & $10^{9-12}$ & $10^{9-12}$ & $10^{8-10}$\\
    $V_{max}$ & $163 - 367$ & $163 - 367$ & $99 - 304$ & $99 - 304$ \\
    $L_{bulge}/L_{tot}$ & $0.3$ & $0.13$ & $0.05$ & ?\\
    $B-V$ & $0.75$ & $0.64$ & $0.52$ & $0.47$\\
    $M_{gas}/M_{B} $ & $0.04$ & $0.08$ & $0.16$ & $0.25$\\
    \bottomrule
\end{tabular}
\end{center}


Another important parameter is the \textbf{Opening angle} $\theta_0$, which is the angle between the major axis of the galaxy and the line of sight.

We can distinguish "grand design" and "flocculent" galaxies.

\begin{tabular}{cccc}
    \toprule
    Surface & brightness & profile & M/L\\
    \midrule
    Bulge & $2^{1/4}$ & low & $\sim 5 \dfrac{M_d}{L_\odot}$\\
    Disk & expo & low & $\sim 3\dfrac{M_\odot}{L_\odot}$\\
    \bottomrule
\end{tabular}

There exists "Bulges" and "Pseudobulges".

We know very few pseudobulges.

Bulges have law of E

Pseudobulges have exopnential profile -> rotation

Freeman Law:

$\mu_0 \sim const$

$$
\mu_0 =  21.52 \pm 0.39 \qquad (S_a \rightarrow S_c)
$$

$$
\mu_0 =  22.61 \pm 0.39 \qquad (S_d)
$$

\todo{How evolution can change the spiral}:

\textbf{Stelar Halo}

MW, M31 $\Rightarrow$ $\rho \propto r^{-3}$

If a galaxy is nearby and it is edge-on, we can see a thick disk.

the size of the disk  is $\mu_B \sim 22.5 mag/arcsec^2$

Another important law is the \textbf{Law of Star Formation Rate} (Schmidt-Kennicutt Law):

$$
\sum_{SFR} \propto \sum_{gas}^N
$$

$$
\dfrac{M_\odot}{y kpc^2}
$$

\subsubsection{Surface photometry bias}

If we observe a portion of a galaxy, and we must "look through" a portion of the galaxy we will see a redder light than observing it from an angle where we don't have to "look through" any portion of the galaxy.

% add plot 2 of my handwritten notes

The same happens for the bulge.

Let's suppose a spiral with no dust: if we observe a "face-on" galaxy, we will see a brighter galaxy than observing the same galaxy "edge-on".

If we consider a very dusty spiral instead, we will notice that the galaxy appears with a similar brightness in both face-on and edge-on views, due to the dust.

% add plot 3 of my handwritten notes

% Disney observed that $L \propto D^2$ -> not true?
% add plot 4 of my handwritten notes - Disney

\subsubsection{Rotation curves}

The rotation curve is the curve that shows the velocity of the stars in the galaxy as a function of the distance from the center of the galaxy.

\dots

A part of this velocity is related to the baryonic mass.

% add plot 5 of my handwritten notes

We can decompose the velocity:

DM Halo
disk

% add plot 6 of my handwritten notes

% talk aboutmaximum disk hypothesis

If $\dfrac ML \sim stars$ We do not need to consider Dark Matter in the disk.
If $\dfrac ML \gg stars$ We need to consider Dark Matter in the disk.

In general, the max velocity is higher if the luminosity is higher.
If we pick galaxies with the same luminosity, the max velocity is higher for the galaxies with the higher mass. % (or lower?)

% missing something here

arms:

grand design -> density moves (internal)
$S_c$ flocculent -> tidal interaction (external)


\subsection{Virial Theorem}

[karttunen demonstration]

If a system is virialized, the following equation holds:

$$
2T + U = 0
$$

where T is the kinetic energy and U is the potential energy.

Suppose we have a system of $n$ point masses $m_i$ with radius vectors $r_i$ and velocities $\dot r_i$. We define a quantity $A$ (the “virial” of the system) as follows:

$$
A = \sum_{i=1}^n m_i \dot r_i \cdot r_i
$$

Time derivative:

$$
\dot A = \sum_{i=1}^n (\underbrace{m_i \dot r_i \cdot \dot r_i}_{2T} + \underbrace{m_i \ddot r_i \cdot r_i}_{F_i})
$$

$$
\dot A = 2T + \sum_{i=1}^n n F_i \cdot r_i
$$

Time average:

$$
\langle \dot A \rangle = \dfrac 1\tau \int_0^\tau \dot A dt = \langle 2T \rangle + \langle \sum_{i=1}^n F_i \cdot r_i \rangle
$$

If the system remains bounded—that is, none of the particles escapes—all positions $\vec r_i$ and velocities $\dot{\vec r}_i$ stay finite. In this case, the virial $A$ does not grow without bound, and the integral in the previous equation also remains finite. When we consider an increasingly long timespan ($\tau \to \infty$), the time average $\langle \dot A \rangle$ approaches zero, and we obtain:

$$
\langle 2T \rangle + \langle \sum_{i=1}^n F_i \cdot r_i \rangle = 0
$$

where $F_i$ is the garvitational force:

$$
\vec F_i = -G m_i \sum_{j = 1, j \neq i}^n m_j \dfrac {\vec r_i - \vec r_j}{r_{ij}^3}
$$

where $r_{ij} = |\vec r_i - \vec r_j|$.

The latter term od the viral theorem becomes:

$$
\begin{array}{rl}
\sum_{i=1}^n F_i \cdot r_i 
& = -G \sum_{i=1}^n \sum_{j = 1, j \neq i}^n m_i m_j \dfrac {r_i - r_j}{r_{ij}^3} \cdot r_i\\
& = -G \sum_{i=1}^n \sum_{j = i + 1}^n m_i m_j \dfrac {r_i - r_j}{r_{ij}^3} (r_i - r_j)
\end{array}
$$

Which is obtained combnining:

$$
\begin{array}{rl}
(1) & = -G \sum_{i=1}^n \sum_{j = 1, j \neq i}^n m_i m_j \dfrac {\vec r_i - \vec r_j}{r_{ij}^3} \cdot \vec r_i\\
(2) & = -G \sum_{j = 1}^n \sum_{i = 1, i \neq j}^n m_i m_j \dfrac {\vec r_j - \vec r_i}{r_{ji}^3} \cdot \vec r_j\\
(3) & = -G \sum \sum m_i m_j \dfrac {\vec r_i - \vec r_j}{r_{ij}^3} (- \vec r_j)
\end{array}
$$

\dots (maybee missing something)

we get:

$$
\sum_{i=1}^n \vec F_i \vec r_i = -G \sum_{i=1}^n \sum_{j = i + 1}^n \dfrac{m_i m_j}{r_{ij}} = U
$$

From the virial theorem we want to derive \bfit{observational quantities}.

$$
\sum_i m_i (v_i - \langle v \rangle)^2 - G \sum_{i > j} \dfrac{m_i m_j}{r_{ij}} = 0
$$

$$
\underbrace{\boxed{\dfrac{\sum_i m_i (v_i - \langle v \rangle)^2}{\sum_i m_i}}}_{\text{velocity dispersion}} - G \underbrace{\boxed{\dfrac{\sum_{i > j} \dfrac{m_i m_j}{r_{ij}}}{(\sum_i m_i)(\sum_i m_i)}}}_{\equiv \frac 1{R_v} \text{viral radius}} \sum_i m_i = 0
$$

\dots

2D position, 1D observations

$$
\sigma^2_v - G \dfrac{M}{R_v} = 0
$$

$$
M = \dfrac{\sigma^2_v R_v}{G}
$$

If the system is spherical, we can use the projected radius to get the virial radius.

$$
\sigma^2_v = 3 \sigma^2_{v, los}
$$

Therefore, we have:

$$
R_v = \dfrac {\pi}2 R_{v, projected}
$$

$$
M = \dfrac{3\pi}2 \dfrac{\sigma^2_{v, los} R_{v, projected}}{G}
$$

\textbf{Segregation effect:} Some objects behaves differently: not always mass is proportional to luminosity. Therefore we need to use the formula above.

Sometimes we will see "tensorial virial theorem", and "generalized virial theorem".

\begin{observationblock}[Virial Theorem validity]
The viral theorem is valid if the mass follows the same distribution of light, buth this is not always the case.
\end{observationblock}

The definition of the \textbf{viral radius} (or better, the \textit{radius of the virial theorem}) is:

$$
R_v = \dfrac {n^2}{\sum_{i>j} \dfrac{1}{r_{ij}}}
$$

The \textbf{harmonic radius} is:

$$
R_H = \dfrac{n(n-1)/2}{\sum_{i>j} \dfrac{1}{r_{ij}}}
$$

\subsubsection{Scaling relations - Spiral galaxies}

The \textbf{Trully-Fisher relation} allows us to relate the luminosity of a galaxy to its velocity dispersion.

$$
L \propto V_{max}^\alpha \qquad \alpha \sim 4
$$

Calibrated TF relation fron nearby galaxies.

For distant galaxies, we can obtain the redshift from the spectrum, and the rotation velocity (and, in particular,  $V_{max}$). Then we can calculate $M$, and then the distance from $m - M$.

\begin{itemize}
    \item For nearby galaxies we can use cepheids to calculate de distance
    \item For very far galaxies we can use the Hubble law
    \item For the ones in the middle we can use the TF relation
\end{itemize}


VT + Spiral structure -> T-F relation

$$
\mathcal{M} \propto \dfrac{V^2_{max}R}{G} \left( \times \dfrac{L}{\mathcal M} \right)
$$

so we get:

$$
L \propto \left( \dfrac{\mathcal{M}}{L} \right)^{-1} \dfrac{V^2_{max}R}{G}
$$

$$
L \propto \left( \dfrac{\mathcal{M}}{L} \right)^{-2} \dfrac{R^2}{L G^2} v_{max}^4
$$

$$
L^2 \propto \left( \dfrac{\mathcal{M}}{L} \right)^{-2} \dfrac{R^2}{G^2} v_{max}^4
$$

$$
L \propto \left( \dfrac{\mathcal{M}}{L} \right)^{-2} \left(\dfrac 1{\langle I \rangle G^2} \right) v_{max}^4
$$

$$
\dfrac{\mathcal{M}}{L} \sim const
$$

\subsubsection{The Faber-Jackson relation}

% not properly working

\subsubsection{The fundamental plane}

$$
(R_e, \sigma_0, \langle I \rangle_e)
$$

$$
R_e \propto \sigma_0^{1.4} \langle I \rangle_e^{-0.85}
$$

Virial Theorem + Kormendy Relation -> Fundamental Plane

$$
R_e \propto \langle I \rangle_e^{-0.82} + \dfrac{\mathcal{M}}{L} \propto \mathcal{M}^{0.2}
$$

\subsubsection{The $D_n - \sigma$ Relation}

Be $D_n$ the diameter of an ellipse within the average surface brightness $I_n$ corresponds to a value of $20.75 mag/arcsec^2$.

We have thet:

$$
D_n \propto \sigma_0^{1.33}
$$

Spectrum -> $\sigma_0$ -> $D_n$ -> distance

\subsubsection{Boh}

\begin{enumerate}
    \item Luminous galaxies $\rightarrow$ Hubble types
    \item Elliptical $\neq$ Spiral galaxies
    \begin{itemize}
        \item morphology
        \item kinematics
        \item gas content
        \item SF (= color)
    \end{itemize}
    \item Kinematics
    \begin{itemize}
        \item Spiral: ordered $m^+$, $V_{max}$
        \item Ellipticals: random $m$, $\sigma_v$
    \end{itemize}
\end{enumerate}

\newpage

\section{Lecture 28/10/2025}

\subsubsection{Galaxies' luminosity function}

The luminosity function $(\Phi(L))$ is the number of galaxies per unit volume per unit luminosity.
$$
v = \int_{-\infty}^{\infty} \Phi(M)dM = \int_0^\infty \Phi(L)dL
$$

We encounter different problems:

\begin{itemize}
    \item what's the distance of a cluster of galaxies?
    \item large scale structure
    \item Maluquist bias - limited surveys in $m$
\end{itemize} 

\todo{reproduce plot in the notebook}

During the years, multiple attempts to solve the problem have been made:

\begin{itemize}
    \item Press-Schechter (74) halos -> Mass function
    \item Schechter (79) -> Luminosity function
\end{itemize}

$$
\Phi(L) = \left( \dfrac{\Phi^*}{L^*} \right) \left( \dfrac{L}{L^*} \right)^{\alpha} e^{-\left( \frac{L}{L^*} \right)}
$$

where $\alpha \sim -1$

thypical B-band:

$$
\Phi^* = 1.6 \cdot 10^{-2} \{ h ^{+3}\}
$$

$$
M^*_B = -19.7 + 5 \log h \quad \Leftarrow \quad L^*_B = 1.2 \cdot 10^{10} h^{-2} L_{\odot, B}
$$

Therefore, $\alpha = -1.07$

For K-band:

$$
\Phi^*_K = 1.6 \cdot 10^{-2} Mpc^{-3}
$$

$$
M^*_K = -23.1
$$

therefore, $\alpha = -0.9$

\begin{observationblock}[$h^{+3}$]
    $h^{+3}$ is used to rescale the luminosity function. At first it was related to $\Omega_m \sim 1$, nowadays it is not so easy to determine, and it depends on $H_0$.
\end{observationblock}

$$
L_{tot} = \int_0^\infty dL \Phi(L) L = \Phi^* L^* \Gamma(2 + \alpha)
$$

which is finite for $\alpha \geq -2$

$$
N_{tot} = \int_0^\infty \Phi(L) dL
$$

which is finite for $\alpha > -1$

The 60\% of $L_{tot}$ is contained in the range $0.22 L^* < L < 1.6 L^*$

The 90\% of $L_{tot}$ is contained in the range $0.1 L^* < L < 2.3 L^*$

Typical luminosity of galaxies is $\Phi^* \sim 2 \cdot 10^{-2} Mpc^{-3}$, with an average separation of $4 Mpc$.

In clusters, the density is way higher than in the field (the rest of the universe). In fact, the average distance between galaxies is $\sim 1-2 Mpc$.

\subsubsection{Specific Luminosity Function}

LF for morphological types:

\begin{minipage}{0.49\textwidth}
    \textbf{Field}

    \todo{put here plot (a) sheet G14 Fig 3.51}
\end{minipage}
\begin{minipage}{0.49\textwidth}
    \textbf{Clusters}

    \todo{put here plot (b) sheet G14 Fig 3.51}
\end{minipage}

In general it is valid $L_{s\emptyset} > L_{sa}$ - Lenticular galaxies are brighter than spiral galaxies.

Effects due to evolution of galaxies:

A merge of spiral galaxies can result in an elliptical galaxy $\qquad S + S \rightarrow E$

% $In \rightarrow dE$ % what is this?

\begin{minipage}{0.49\textwidth}
    \textbf{Groups}
    \begin{itemize}
        \item $N \lesssim 50$ galaxies
        \item $M \lesssim 3\cdot 10^{13} M_{\odot}$
        \item $T_x \sim 1 \rightarrow 3 KeV$
        \item $\sigma_v \sim 200 \rightarrow 300 km/s$    
    \end{itemize}
\end{minipage}%
\begin{minipage}{0.40\textwidth}
    \textbf{Clusters}
    \begin{itemize}
        \item $N \gtrsim 50$ galaxies
        \item $M \gtrsim 3\cdot 10^{14} M_{\odot}$
        \item $T_x \sim 4 \rightarrow 10 KeV$
        \item $\sigma_v \sim 400 \rightarrow 1000 km/s$
    \end{itemize}
\end{minipage}

$$
R_{Abell} = 1.5 h_{100}^{-1} Mpc \quad \rightarrow \quad 1.5 \cdot \dfrac{100}{70} h_{70}^{-1} M_r
$$

Both are composed mainly by:

\begin{itemize}
    \item Stars $\sim 3-5\%$
    \item Hot gas ($3\cdot 10^7 K$) $\sim 15-20\%$
    \item Dark Matter $\sim 80\%$
\end{itemize}


\subsubsection{Galaxy distribution}

The following figure shows the distribution of galaxies in the universe.

\begin{figure}[H]
    \centering
    \includegraphics[width=0.6\textwidth]{assets/gal-distribution.jpg}
\end{figure}

Some structures are "real", such as the \textit{Great Wall}, others are "apparent", such as the \textit{Finger of God}. % explain why

\begin{tipsblock}[Galaxy distribution]
    In high redshift zones, the galaxies seem to be less dense than in low redshift zones. This is because the farthest galaxies have a lower apparent magnitude than the nearest ones.
\end{tipsblock}

% plots from slidesGalaxiesNew and slidesGalaxiesBlanton, slidesGalsKinematics, slidesSEDznew