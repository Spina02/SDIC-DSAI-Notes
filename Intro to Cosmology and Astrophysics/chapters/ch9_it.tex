\chapter{Lezione 11/04/2025}

\dots

$$
k_\sigma k^\sigma = 0
$$

Possiamo riscrivere:

$$
k_\sigma k^\sigma = \eta_{\sigma \alpha}k^\sigma k^\alpha = k^0 k^0 - (k^1 k^1 + k^2 k^2 + k^3 k^3) = \dfrac {\omega^2}{c^2} - |\bar k|^2 \equiv 0
$$

Da cui si ha 
$$
\omega = kc \quad \to \quad \hslash \omega = \hslash k \cdot c \quad \Rightarrow \quad E = P \cdot c 
$$
come per i fotoni, con massa a riposo nulla: il quanto mediatore dell'interazione gravitazionale, il gravitone, ha massa nulla.

OSserviamo anche che 

$$
k_{\gamma} \cdot x^\gamma = \eta_{\gamma \sigma} \cdot x^{\gamma} = k^0 x^0 - |\bar k \cdot \bar x| = \omega t - \bar k \cdot \bar x
$$

poichè $\quad h_{\delta}^\alpha = \eta^{\alpha \sigma} h_{\sigma \delta} = \eta^{\alpha \sigma} A_{\sigma \delta} e^{i k_\gamma \cdot x^{\gamma}}\quad$ si ha $\quad\dfrac{\partial h_{\delta^\alpha}}{\partial x^\alpha}\quad$

$$
\frac{\partial}{\partial x^\alpha} \left(
    A_{\sigma \delta} \eta^{\alpha \sigma} e^{i k_\gamma \cdot x^\gamma}
\right)
= A_{\sigma \delta} \eta^{\alpha \sigma} e^{i k_\gamma \cdot x^\gamma} \cdot i k_\alpha
$$

cioè

$$
i k_\alpha A_{\delta}^{\alpha} e^{i k_\gamma \cdot x^\gamma} = i k_\alpha h_{\delta}^{\alpha} = 0 \quad \Rightarrow \quad h_{\delta}^{\alpha} k_\alpha = 0
$$

che è detta \bfit{condizione di tarsversalità}.

Scegliamo la direzione di propagazione lungo l'asse $x$, ovvero $\bar k = (k, 0, 0)$ e quindi $h_\sigma^\alpha \cdot k_\alpha = \eta^{\alpha \delta} h_{\delta \sigma} k_\alpha = h_{\delta \sigma}k^\delta = 0$.

Ricordiamo che le condizioni di scelta del sistema di riferimento ($h = 0$ e $h_{0i} = 0$):

$$
\begin{array}{ccccc}
    \sigma = 0 & \quad \to \quad  & h_{00} k^0 + h_{10} k^1 + h_{20} k^2 + h_{30} k^3 = 0 & \quad \to \quad & h_{00} = 0 \\
    \sigma = 1 & \quad  \to \quad  & h_{01} k^0 + h_{11} k^1 + h_{21} k^2 + h_{31} k^3 = 0 & \quad \to \quad & h_{11} = 0 \\
    \sigma = 2 & \quad  \to \quad  & h_{02} k^0 + h_{12} k^1 + h_{22} k^2 + h_{32} k^3 = 0 & \quad \to \quad & h_{12} = h_{21} = 0 \\
    \sigma = 3 & \quad  \to \quad  & h_{03} k^0 + h_{13} k^1 + h_{23} k^2 + h_{33} k^3 = 0 & \quad \to \quad & h_{13} = h_{31} = 0
\end{array}
$$

In forma matriciale:

$$
h_{\beta \delta} = \begin{pmatrix}
    0 & 0 & 0 & 0 \\
    0 & 0 & 0 & 0 \\
    0 & 0 & h_{22} & h_{23} \\
    0 & 0 & h_{32} & h_{33}
\end{pmatrix}   
$$
dove $h = 0 \Rightarrow h_{22} + h_{33} = 0 \Rightarrow h_{22} = - h_{33} \equiv h_+$  e per simmetria $h_{23} = h_{32} \equiv h_\times$.

Quindi, in forma matriciale:

$$
h_{\beta \delta} = \begin{pmatrix}
    0 & 0 & 0 & 0 \\
    0 & 0 & 0 & 0 \\
    0 & 0 & h_+ & h_\times \\
    0 & 0 & h_\times & -h_+
\end{pmatrix}
$$

se $\bar k$ è lungo l'asse $x$, le componenti non nulle dell'onda sono perpendicolari all'asse $x$ e quindi sono onde trasversali a due componenti (polarizzazioni):

$$
\begin{cases}
    h_+ = A_+ e^{i (\omega t - \bar k \bar x)} \\
    h_\times = A_\times e^{i (\omega t - \bar k \bar x)}
\end{cases}
$$

\dots

$$
\dfrac {d u^\mu}{d \tau} + \Gamma^\mu_{\alpha \beta} u^\alpha u^\beta = 0
$$

$$
\dfrac {d u^\mu}{d \tau} + \Gamma^\mu_{00} = 0
\quad \Rightarrow \quad
\dfrac {d u^\mu}{d \tau} += - \Gamma^\mu_{00} = \dfrac 12 \eta^{\mu \sigma}
\bigg(
    \underbrace{\dfrac{\partial h_{\sigma 0}}{\partial x^0}}_{\to 0} + 
    \underbrace{\dfrac{\partial h_{0 \sigma}}{\partial x^0}}_{\to 0} - 
    \underbrace{\dfrac{\partial h_{00}}{\partial x^\sigma} }_{\to 0}
\bigg)
$$

\dots

Per onde di tipo $h_+$ abbiamo:

$$
h_{\mu \nu} = \begin{pmatrix}
    0 & 0 & 0 & 0 \\
    0 & 0 & 0 & 0 \\
    0 & 0 & h_+ & 0 \\
    0 & 0 & 0 & 0-h_+
\end{pmatrix}
e^{-(\nu x - \omega t)}
=
\begin{pmatrix}
    0 & 0 & 0 & 0 \\
    0 & 0 & 0 & 0 \\
    0 & 0 & h_+ & 0 \\
    0 & 0 & 0 & 0-h_+
\end{pmatrix}
\cos(\omega t - x \nu + \phi)
$$

$$
g_{\mu \nu}
= 
\begin{pmatrix}
    1 & 0 & 0 & 0 \\
    0 & -1 & 0 & 0 \\
    0 & 0 & -1 & 0 \\
    0 & 0 & 0 & -1
\end{pmatrix}
+
\begin{pmatrix}
    0 & 0 & 0 & 0 \\
    0 & 0 & 0 & 0 \\
    0 & 0 & h_+ & 0 \\
    0 & 0 & 0 & -h_+
\end{pmatrix}
e^{-(\mu x - \omega t)}
$$

\dots

$$
ds^2 = 
c^2 dt^2 - \left\{
    dx + \left[
        1 - h_+ \sin (\omega t)
    \right] dy^2
    + \left[
        1 - h_+ \sin (\omega t)
    \right] dz^2
\right\}
$$

$$
\begin{cases}
    Y = (1 + \dfrac 12 h_+ \sin (\omega t)) y \\
    Z = (1 - \dfrac 12 h_+ \sin (\omega t)) z
\end{cases}
\quad \Rightarrow \quad
\begin{cases}
    dY^2 = dy^2 + dy^2h_+ \sin (\omega t)
    dZ^2 = dz^2 + dz^2h_+ \sin (\omega t)
\end{cases}
$$

\dots

Per onde $h_\times$ avremo similmente:

$$
\begin{array}{rcl}
    ds^2 & = & c^2 dt^2 - [ dx^2 + dy^2 + dy dz h_\times \sin \omega t + dz^2 + dy dz h_\times \sin(\omega t) ]\\
    & = & c^2 dt^2 - [ dy^2 + dz^2 + 2 h_\times \sin \omega t dy dz]
\end{array}
$$
        
$$
\begin{cases}
    Y = y + \dfrac 12 h_\times \sin \omega t z \\
    Z = z + \dfrac 12 h_\times \sin \omega t y
\end{cases}
\quad \Rightarrow \quad
\begin{cases}
    dY^2 = dy^2 + h_\times \sin \omega t \ dy \ dz \\
    dZ^2 = dz^2 + h_\times \sin \omega t \ dy \ dz
\end{cases}
$$

Dunque possiamo accorgerci del passaggio di un'onda gravitazionale osservando lo spostamento tra le particelle.

Lo \textit{strain} è definito come la variazione della lunghezza di un braccio dell'interferometro rispetto alla lunghezza originale:
$$
h = \dfrac{dL}L = \dfrac 12 \sqrt{h_\times^2 + h_+^2} \approxeq 10^{-21}
$$

