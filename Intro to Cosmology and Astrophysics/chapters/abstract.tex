\chapter*{Prefazione}

Come studente di Scientific and Data Intensive Computing, ho creato questi appunti durante la frequentazione del corso di \textbf{Introduzione a Galassie e Astrofisica}.

Questo è un corso opzionale per il secondo semestre del primo anno della mia laurea magistrale, ma è disponibile anche per gli studenti del dipartimento di Fisica; in ogni caso, il prossimo anno verrà probabilmente rimosso da entrambi i curricula. 

Il corso è tenuto dai Prof. Alexandro Saro e Costanzi Matteo, ricercatori dell'Osservatorio Astronomico INAF di Trieste; le lezioni del corso sono tenute in italiano e di conseguenza anche questi appunti sono stati scritti in italiano.

\vspace{0.5em}

La prima parte del corso si concentrerà sulla gravità e sulla teoria della relatività generale di Einstein:

\begin{itemize}
    \item Geometria non euclidea
    \item Tensori
    \item Principi ed equazioni di Einstein
    \item Onde gravitazionali
\end{itemize}

\vspace{0.5em}

La seconda parte del corso si concentrerà sulla cosmologia:

\begin{itemize}
    \item Metrica di Robertson-Walker
    \item Legge di Hubble
    \item Equazioni di Friedmann
    \item Modelli cosmologici
    \item Cosmologia di precisione (cenni)
\end{itemize}