\chapter{Lecture 28/03/2025}

\dots

L'equazione delle geodetiche è sempre:

$$
\dfrac{d^2x^\alpha}{ds^2} + \Gamma^\alpha_{\beta \gamma} \dfrac{dx^\beta}{ds} \dfrac{dx^\gamma}{ds} = 0
$$ 

se la metrica è data da $\eta_{\alpha \beta}$, allora i $\Gamma^\alpha_{\beta \gamma}$ osno nulli e resta $d^2x^\alpha/ds^2 = 0$, cioè $x^\alpha = a^\alpha \cdot s + b^\alpha$, ovvero:

$$
\begin{cases}
    ct = a^0 \cdot s + b^0\\
    \bar x = \bar a \cdot s + \bar b
\end{cases}
$$

e la traiettoria è una retta percorsa di moto rettilineo uniforme.

\subsubsection{Tensore energia-impulso}

Vogliamo ora descrivere le proprietà gravitazionali di un fluido

Consideriamo un fluido in cui l'unica forza presente è quella gravitazionale. Chiamiamo tale fluido "polvere" (dust).

Le due quantità che ci interessano sono densità e velocità.

Il tensore più sempliche che possiamo pensare che descriva tali quantità è il \bfit{tensore energia-impulso}:

$$
T^{\alpha \beta} = \rho_0 c^2 u^\alpha u^\beta
$$

Nel dettaglio abbiamo:

$$
T^{00} = \rho_0 c^2 \gamma^2 = \gamma^2 \rho_0 c^2 = \rho c^2 \qquad \text{posto } \rho = \gamma^2 \rho_0
$$

Per interpretare questo risultato ricordiamo che la massa è $m = \gamma m_0$ (con $m_0$ massa a riposo) e che un elemento di volume in moto appare contratto di un fattore $1 / \gamma$, per cui la densità cresce di un ulteriore fattore $\gamma$.

Perciò, se la densità propria è $\rho_0$, un osservatre rispetto al quale il fluido ha velocità $\bar v$ misurerà una densità $\gamma^2 \rho_0$.

$T^{00}$ misura quindi la densità di massa-energia (qui l'unico contributo all'energia viene dal moto della materia).

Le componenti di $T^{\alpha \beta}$ sono:

$$
T^{\alpha \beta} =
\rho c^2 \cdot
\begin{array}{cccc}
    1 & v_x/c & v_y/c & v_z/c\\
    v_x/c & v_x^2/c^2 & v_x v_y/c^2 & v_x v_z/c^2\\
    v_y/c & v_x v_y/c^2 & v_y^2/c^2 & v_y v_z/c^2\\
    v_z/c & v_x v_z/c^2 & v_y v_z/c^2 & v_z^2/c^2
\end{array}
$$

Vediamo come le equazioni del moto possono essere ricavate in modo sintetico come $\partial_\beta T^{\alpha \beta} = 0$

\begin{itemize}
    \item Per $\alpha = 0$ si ha $\partial_\beta T^{0 \beta} = 0 \quad \Leftrightarrow \quad \frac {\partial T^{0\beta}}{\partial x^\beta} = 0$, che scritta per esteso:
    
    $$
    \dfrac 1c \dfrac {\partial(\rho c^2)}{\partial x} + \dfrac{\partial(\rho c v_x)}{\partial x} + \dfrac{\partial(\rho c v_y)}{\partial y} + \dfrac{\partial(\rho c v_z)}{\partial z} = 0
    $$
    che si può semplificare in:
    $$
    \dfrac {\partial \rho}{\partial t} + \nabla \cdot (\rho \bar v) = 0
    $$
    (dove $\bar v$ è la velocità totale) cioè l'equazione di continuità per un fluido, che esprime la conservazione della massa-energia.

    \item Per $\alpha = 1, 2, 3$ si ha:

    $$
    \begin{cases}
        \dfrac 1c \dfrac{\partial(\rho c v_x)}{\partial t} + \dfrac{\partial(\rho c v_x v_x)}{\partial x} + \dfrac{\partial(\rho c v_x v_y)}{\partial y} + \dfrac{\partial(\rho c v_x v_z)}{\partial z} = 0 \quad (\alpha = 1)
        \\[0.5em]
        \dfrac 1c \dfrac{\partial(\rho c v_y)}{\partial t} + \dfrac{\partial(\rho c v_y v_x)}{\partial x} + \dfrac{\partial(\rho c v_y v_y)}{\partial y} + \dfrac{\partial(\rho c v_y v_z)}{\partial z} = 0 \quad (\alpha = 2)
        \\[0.5em]
        \dfrac 1c \dfrac{\partial(\rho c v_z)}{\partial t} + \dfrac{\partial(\rho c v_z v_x)}{\partial x} + \dfrac{\partial(\rho c v_z v_y)}{\partial y} + \dfrac{\partial(\rho c v_z v_z)}{\partial z} = 0 \quad (\alpha = 3)
    \end{cases}
    $$  
\end{itemize}

moltipichiamo le tre equazioni per i versori $\bar i, \bar j, \bar k$ e sommiamo membro a membro:

$$
\dfrac \partial {\partial t}(\rho \bar v) + \dfrac \partial {\partial x}(\rho v_x \bar v) + \dfrac \partial {\partial y}(\rho v_y \bar v) + \dfrac \partial {\partial z}(\rho v_z \bar v) = 0
$$

che, sviluppando e usando poi l'equazione di continuità, si riduce a:

$$
\rho \dfrac {\partial \bar v}{\partial t} + \bar v \left[
    \dfrac {\partial \rho}{\partial t} + \nabla \cdot (\rho \bar v)
\right] + 
\rho v_x \dfrac {\partial \bar v}{\partial x} +
\rho v_y \dfrac {\partial \bar v}{\partial y} +
\rho v_z \dfrac {\partial \bar v}{\partial z} = 0
$$

cioè:

$$
\underbrace{\rho \left[
    \dfrac {\partial \bar v}{\partial t} + (\bar v \cdot \nabla) \bar v
\right] = 0}_{(I)}
\qquad \Leftrightarrow \qquad 
\underbrace{\rho \dfrac {d \bar v}{d t} = 0}_{(II)}
$$

Questa relazione, tipica della fluidodinamica, rappresenta l'equazione del moto per un fluido senza pressione, viscosità e forze esterne.

...