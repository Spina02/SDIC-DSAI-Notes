\newpage

\chapter{Tensori}

Avremo a che fare con grandezze \emph{tensoriali}, grandezze e cui proprietà sono legate al modo di trasformarsi cambiando sistema di riferimento.

\section{Tensori Controvarianti e Covarianti}

\subsubsection{Tensori Controvarianti}


Passando da un sistema generico di dimensione $n$ ad un altro 
$
u^i = (i = 1, \dots, n) \ \rightarrow \ u'^j = (j = 1, ..., n)
$
abbiamo visto che vale la regola di trasformazione:
\vspace{0.5em}
$$
du'^j = \sum_{i=1}^{n} \dfrac {\partial u'^j}{\partial u^i} du^i
$$
Ogni grandezza $V^j$ che si trasforma con la regola:
\vspace{0.5em}
$$
\boxed{
    V'^{j} = \sum_{i=1}^{n} \dfrac {\partial u'^j}{\partial u^i} V^{i} 
    \quad \xrightarrow{\text{notazione di Einstein}}\quad 
    V'^{j} = \dfrac {\partial u'^j}{\partial u^i} V^{i} 
    }
$$
è un \textbf{tensore controvariante}, quindi anche $du^i$ (o $u^i$) è un tensore controvariante. Un vettore è un tensore di \bfit{ordine} o (\bfit{rango}) 1. Una grandezza scalare, invece, il cui valore non muta in un dato punto anche se cambia il sistema di coordinate, è un tensore di ordine 0.

\subsubsection{Tensori Covarianti}

Consideriamo ora un campo scalare $\Phi(u^i) = \Phi(u'^j)$. 
Ad esempio un campo che definisce la temperatura in un punto dello spazio, per la quale, indipendentemente dal sistema di riferimento $u^i$ o $u'^j$, il valore della temperatura è sempre lo stesso.

La relazione che sussiste tra il gradiente di $\Phi$ in un sistema di riferimento e quello in un altro è:
\vspace{0.5em}
$$
% \begin{array}{rcl}
\partial \Phi_{'j} = \dfrac{\partial \Phi} {\partial u'^j} 
= \sum_i \dfrac{\partial \Phi} {\partial u^i} \dfrac{\partial u^i} {\partial u'^j}
= \sum_i \dfrac{\partial u^i} {\partial u'^j} \dfrac{\partial \Phi} {\partial u'} 
= \sum_i \dfrac{\partial u^i} {\partial u'^j} \partial \Phi_{'i} = W
% \end{array}
$$

Ogni grandezza $W_{'j}$ che si trasforma con la regola:
\vspace{0.5em}
$$
\boxed{
    W_{'j} = \sum_i \dfrac{\partial u^i} {\partial u'^j} W_j
    \quad \xrightarrow{\text{notazione di Einstein}}\quad 
    W_{'j} = \dfrac{\partial u^i} {\partial u'^j} W_j 
    }
$$

è un \textbf{tensore covariante}.

Abbiamo visto la grandezza $ds^2 = g_{ij} du^i du^j$, che indica la lunghezza, al quadrato, di un segmento infinitesimo, e quindi non dipende dal sistema di riferimento usato (è uno scalare). In due sistemi diversi sarà quindi:
$$
ds^2 = g_{ij} du^i du^j = g_{ij} \dfrac {\partial u^i}{\partial u'^l} du'^l \dfrac {\partial u^j}{\partial u'^k} du'^k = g'_{lk} du'^l du'^k 
$$
poichè
$$
g'_{lk} = \dfrac {\partial u^i}{\partial u'^l} \dfrac {\partial u^j}{\partial u'^k} g_{ij} \quad \rightarrow \quad \text{"covariante"}
$$
abbiamo quindi:
$$
g'^{lk} = \dfrac {\partial u'^l}{\partial u^i} \dfrac {\partial u'^k}{\partial u^j} g^{ij} \quad \rightarrow \quad \text{"controvariante"}
$$
Cioè $g_{ij}$ è un tensore covariante di rango 2, mentre $g^{ij}$ è un tensore controvariante di rango 2.

Se un tensore ha sia indici covarianti che controvarianti, come abbiamo visto in $g_{ij}g^{jk} = \delta_i^k$ allora si dice \textbf{misto}. In tal caso è necessario applicare una trasformazione controvariante per gli indici controvarianti e una covariante per gli indici covarianti:
\vspace{0.5em}
$$
V'^k_l = \dfrac {\partial u'^k}{\partial u^j} \dfrac {\partial u^i}{\partial u'^l} V^j_i
$$

\begin{observationblock}[Tensori e indici]
    Un tensore ha \textbf{rango} $n$ se ha $n$ indici, sia covarianti che controvarianti.
    
    \underline{\textbf{Nota}}: \textit{Non tutti gli ogetti che hanno indici sono tensori}
\end{observationblock}

\newpage

Consderiamo un tensore covariante $D_i$ e un tensore metrico $g_{ij}$:

$$
D_i = g_{ij} C^j
$$


$$
D_i D^i = g_{ij} C^j D^i \quad \Rightarrow \quad 
\begin{cases}
    D_i = g_{ij} C^j\\
    C^j = D^j
\end{cases}
\quad \rightarrow \quad 
D_i = g_{ij} D^j
$$

\vspace{5em}

$$
\vec v = v^i \bar x_1
$$

$$
v_m = \vec v^i \cdot \bar x_k = v \bar x_i \bar x_k = v^i g_{ik}
$$

Consideriamo un vettore $A$:

...
