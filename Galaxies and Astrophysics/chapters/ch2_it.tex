\chapter{Tensori}

Introduciamo il concetto di \textit{tensore}, fondamentale per descrivere grandezze geometriche e fisiche in modo indipendente dal sistema di coordinate scelto. Analizzeremo come queste grandezze si trasformano passando da un sistema di coordinate a un altro, definendo tensori controvarianti, covarianti e misti. Vedremo infine come il tensore metrico permetta di convertire tra componenti controvarianti e covarianti.

\section{Leggi di Trasformazione Tensoriale}

Il concetto chiave per definire un tensore è la sua \textbf{legge di trasformazione} al cambiare delle coordinate. Consideriamo due sistemi di coordinate $n$-dimensionali, $u^i = (u^1, ..., u^n)$ e $u'^j = (u'^1, ..., u'^n)$.

\subsubsection{Tensori Controvarianti (di Rango 1)}

Ricordiamo dal \cref{ch:geometria_differenziale} che i differenziali delle coordinate si trasformano secondo la regola:
$$
du'^j = \sum_{i=1}^{n} \dfrac {\partial u'^j}{\partial u^i} du^i
$$
Una qualsiasi grandezza le cui componenti $V^i$ nel sistema $u$ si trasformano in $V'^j$ nel sistema $u'$ seguendo la \emph{stessa} regola dei differenziali:
\vspace{0.5em}
$$
\boxed{
    V'^{j} = \sum_{i=1}^{n} \dfrac {\partial u'^j}{\partial u^i} V^{i}
    \quad \xrightarrow{\text{notazione di Einstein}}\quad
    V'^{j} = \dfrac {\partial u'^j}{\partial u^i} V^{i}
    }0
$$
è definita \textbf{tensore controvariante di rango 1}, o semplicemente \bfit{vettore controvariante}.
Ne consegue che le componenti del differenziale delle coordinate, $du^i$, costituiscono esse stesse un vettore controvariante.

\begin{observationblock}[Scalari]
Una grandezza \bfit{scalare} $\Phi$ è una quantità il cui valore in un punto non dipende dal sistema di coordinate ($\Phi(u^i) = \Phi(u'^j)$). È considerato un \textbf{tensore di rango 0}. Un esempio fisico è la temperatura in un punto dello spazio.
\end{observationblock}

\subsubsection{Tensori Covarianti (di Rango 1)}

Consideriamo ora il \bfit{gradiente} di un campo scalare $\Phi$. Le sue componenti nel sistema $u^i$ sono date dalle derivate parziali $W_i = \partial \Phi / \partial u^i$. Vediamo come si trasformano queste componenti passando al sistema $u'^j$. Le componenti nel nuovo sistema sono $W'_j = \partial \Phi / \partial u'^j$. Usando la regola della catena per la derivazione:
$$
W'_{j} = \dfrac{\partial \Phi} {\partial u'^j}
= \sum_i \dfrac{\partial \Phi} {\partial u^i} \dfrac{\partial u^i} {\partial u'^j}
= \sum_i W_i \dfrac{\partial u^i} {\partial u'^j}
$$
Una qualsiasi grandezza le cui componenti $W_i$ si trasformano secondo questa regola:
\vspace{0.4em}
$$
\boxed{
    W'_{j} = \sum_i \dfrac{\partial u^i} {\partial u'^j} W_i
    \quad \xrightarrow{\text{notazione di Einstein}}\quad
    W'_{j} = \dfrac{\partial u^i} {\partial u'^j} W_i
    }
$$
è definita \textbf{tensore covariante di rango 1}, o \bfit{covettore} (o \bfit{forma 1}).

Notiamo che la trasformazione usa le derivate $\partial u^i / \partial u'^j$, inverse rispetto a quelle usate per i tensori controvarianti ($\partial u'^j / \partial u^i$). Le componenti del gradiente $\partial \Phi / \partial u^i$ formano un covettore.

\subsubsection{Tensore Metrico ($g_{ij}$) e suo Inverso ($g^{ij}$)}

Abbiamo visto nel \cref{ch:geometria_differenziale} che la distanza infinitesima al quadrato, $ds^2$, è uno scalare (invariante per cambi di coordinate). La sua espressione è data dalla prima forma fondamentale $ds^2 = g_{ij} du^i du^j$. Poiché $ds^2$ è invariante, deve valere in qualsiasi sistema di coordinate $ds^2 = g_{ij} du^i du^j = g'_{lk} du'^l du'^k$.

Sostituendo la legge di trasformazione per i differenziali $du^i = \frac{\partial u^i}{\partial u'^l} du'^l$ (e analogamente per $du^j$), otteniamo:
\vspace{0.2em}
$$
ds^2 = g_{ij} \left( \dfrac {\partial u^i}{\partial u'^l} du'^l \right) \left( \dfrac {\partial u^j}{\partial u'^k} du'^k \right) = \left( g_{ij} \dfrac {\partial u^i}{\partial u'^l} \dfrac {\partial u^j}{\partial u'^k} \right) du'^l du'^k
$$
Confrontando questa espressione con $ds^2 = g'_{lk} du'^l du'^k$, e poiché l'uguaglianza deve valere per qualsiasi scelta dei differenziali $du'^l, du'^k$, deduciamo la legge di trasformazione per le componenti del tensore metrico:
\vspace{0.2em}
$$
\boxed{ g'_{lk} = g_{ij} \dfrac {\partial u^i}{\partial u'^l} \dfrac {\partial u^j}{\partial u'^k} }
$$
Tale trasformazione coinvolge due fattori $\partial u / \partial u'$ e conferma che $g_{ij}$ è un \textbf{tensore covariante di rango 2}.

\vspace{0.5em}

Consideriamo ora la matrice inversa del tensore metrico, le cui componenti sono indicate con $g^{ij}$, definita dalla relazione fondamentale $g_{ij}\,g^{jk} = \delta_i^k$. Vogliamo determinare come si trasformano le componenti $g^{ij}$ al cambiare delle coordinate. Supponiamo che $g^{ij}$ si trasformi come un tensore controvariante di rango 2, ovvero:
$$
g'^{pq} = g^{ik} \dfrac{\partial u'^p}{\partial u^i} \dfrac{\partial u'^q}{\partial u^k} \quad \text{(Ipotesi)}
$$
Per verificare questa ipotesi, controlliamo se essa è consistente con l'invarianza della delta di Kronecker, cioè se $g'^{pq} g'_{qn} = \delta^p_n$. Utilizziamo le leggi di trasformazione per $g'^{pq}$ (ipotizzata) e per $g'_{qn}$ (nota):

\begin{enumerate}
\item Partiamo da  
    \vspace{0.4em}
    $$
    g'^{pq}\,g'_{qn}
    =\bigl(g^{ik}\,\dfrac{\partial u'^p}{\partial u^i}\,\dfrac{\partial u'^q}{\partial u^k}\bigr)\,
        \bigl(g_{mj}\,\dfrac{\partial u^m}{\partial u'^q}\,\dfrac{\partial u^j}{\partial u'^n}\bigr)
    $$
\item Raggruppiamo le derivate e i tensori metrici:
    \vspace{0.4em}
    $$
    =\,g^{ik}g_{mj}\;
        \dfrac{\partial u'^p}{\partial u^i}\;
        \bigl(\dfrac{\partial u'^q}{\partial u^k}\,\dfrac{\partial u^m}{\partial u'^q}\bigr)\;
        \dfrac{\partial u^j}{\partial u'^n}
    $$
\item Applichiamo la regola della catena all'interno delle parentesi:  
    \vspace{0.4em}
    $$
    \dfrac{\partial u'^q}{\partial u^k}\,\dfrac{\partial u^m}{\partial u'^q} 
    = \delta^m_k
    \qquad \rightarrow \qquad
    =\,g^{ik}g_{mj}\;
        \dfrac{\partial u'^p}{\partial u^i}\;\delta^m_k\;
        \dfrac{\partial u^j}{\partial u'^n}
    =\,g^{ik}g_{kj}\;
        \dfrac{\partial u'^p}{\partial u^i}\;
        \dfrac{\partial u^j}{\partial u'^n}
    $$
\item Usiamo \(g^{ik}g_{kj}=\delta^i_j\):
    \vspace{0.4em}
    $$
    =\,\delta^i_j\;
        \dfrac{\partial u'^p}{\partial u^i}\;
        \dfrac{\partial u^j}{\partial u'^n}
    =\,\dfrac{\partial u'^p}{\partial u^i}\,
        \dfrac{\partial u^i}{\partial u'^n}
    $$
\item Di nuovo la catena dà
    \vspace{0.4em}
   $$
   \dfrac{\partial u'^p}{\partial u^i}\,
   \dfrac{\partial u^i}{\partial u'^n}
   =\delta^p_n
   $$
\end{enumerate}

Poiché la verifica $g'^{pq} g'_{qn} = \delta^p_n$ ha successo, l'ipotesi iniziale era corretta. La legge di trasformazione per le componenti della metrica inversa è quindi:
$$
\boxed{ g'^{lk} = g^{ij} \dfrac {\partial u'^l}{\partial u^i} \dfrac {\partial u'^k}{\partial u^j} }
$$
Tale trasformazione coinvolge due fattori $\frac{\partial u'}{\partial u}$ e conferma che $g^{ij}$ è un \textbf{tensore controvariante di rango 2}.

\subsubsection{Tensori Misti}
Un tensore può avere sia indici controvarianti (in alto) che covarianti (in basso). Un esempio fondamentale è la \textbf{delta di Kronecker} $\delta^k_i$, che è un \bfit{tensore misto di rango 2} (1 controvariante, 1 covariante).
Per un generico tensore misto, come $V^j_i$, la legge di trasformazione combina le regole precedenti: un fattore $\partial u'/\partial u$ per ogni indice controvariante e un fattore $\partial u/\partial u'$ per ogni indice covariante:
\vspace{0.5em}
$$
V'^k_l = \dfrac {\partial u'^k}{\partial u^j} \dfrac {\partial u^i}{\partial u'^l} V^j_i
$$

\begin{observationblock}[Rango e Indici]
    Un tensore ha \textbf{rango} $N+M$ se ha $N$ indici controvarianti e $M$ covarianti (spesso detto di tipo $(N,M)$).

    \underline{\textbf{Nota}}: \textit{Non tutti gli oggetti matematici definiti con indici si trasformano come tensori. Un esempio importante che vedremo più avanti sono i simboli di Christoffel $\Gamma^k_{ij}$.}
\end{observationblock}

\begin{tipsblock}[Invarianza della Delta di Kronecker]
    Si può dimostrare che $\delta^k_i$ è l'unico tensore (a parte gli scalari costanti e il tensore nullo) le cui componenti sono le stesse in tutti i sistemi di coordinate.
    $$
    \delta'^k_l = \dfrac {\partial u'^k}{\partial u^j} \dfrac {\partial u^i}{\partial u'^l} \delta^j_i = \dfrac {\partial u'^k}{\partial u^j} \dfrac {\partial u^j}{\partial u'^l} = \dfrac{\partial u'^k}{\partial u'^l} = \delta^k_l
    $$
\end{tipsblock}

\subsection{Algebra Tensoriale}

Le operazioni algebriche di base (somma, prodotto per scalare) si applicano ai tensori componente per componente, a patto che abbiano lo stesso rango e lo stesso tipo (stesso numero di indici controvarianti e covarianti). Esistono inoltre operazioni specifiche per i tensori:

\begin{itemize}
    \item \textbf{Prodotto Esterno (o Tensoriale):} Il prodotto esterno di due tensori genera un nuovo tensore il cui rango è la somma dei ranghi dei tensori originali. Ad esempio, il prodotto esterno di un tensore di tipo (0,2), $A_{ij}$, e di un vettore controvariante (tipo (1,0)), $C^k$, è un tensore di tipo (1,2):
    $$
    D^k_{ij} = A_{ij} C^k
    $$
    Le componenti del nuovo tensore sono semplicemente il prodotto delle componenti dei tensori originali. Non si applica la somma di Einstein in questo caso, poiché non ci sono indici ripetuti.

    \item \textbf{Contrazione:} La contrazione riduce il rango di un tensore di 2. Si applica a un tensore che abbia almeno un indice controvariante e uno covariante. Consiste nell'eguagliare uno degli indici superiori con uno degli indici inferiori e sommare su quell'indice secondo la convenzione di Einstein. Ad esempio:
    \vspace{0.4em}
    $$
    \text{Sia } T^j_{kl} \text{ un tensore (1,2). Contraendo gli indici } j \text{ e } l \text{ si ottiene:} \quad B_k = T^j_{kj}
    $$
    $B_k$ è un tensore di rango 1 (covariante).

    \item \textbf{Prodotto Scalare (Contrazione + Prodotto Esterno):} Il prodotto scalare tra due vettori (che geometricamente è $\bar{v} \cdot \bar{w}$) può essere visto come il prodotto esterno seguito da una doppia contrazione usando il tensore metrico. Per due vettori controvarianti $v^i$ e $w^j$, il prodotto esterno è $v^i w^j$. Contraendo con $g_{ij}$:
    $$
    \text{Scalare } = g_{ij} v^i w^j \quad (= \bar{v} \cdot \bar{w})
    $$
    Analogamente, per due vettori covarianti $v_i$ e $w_j$:
    $$
    \text{Scalare } = g^{ij} v_i w_j
    $$
    E per un vettore controvariante $v^i$ e uno covariante $w_j$:
    $$
    \text{Scalare } = v^i w_i \quad (= v^i g_{ij} w^j)
    $$

    \item \textbf{Innalzare e Abbassare Indici:} Il tensore metrico $g_{ij}$ e il suo inverso $g^{ij}$ permettono di convertire componenti controvarianti in covarianti e viceversa, mantenendo l'identità del tensore sottostante. Per un vettore $\mathbf{D}$ con componenti controvarianti $D^j$ e covarianti $D_i$:
    $$
    \boxed{ D_i = g_{ij}\,D^j } \quad (\text{abbassare l'indice } j)
    $$
    $$
    \boxed{ D^i = g^{ij}\,D_j } \quad (\text{innalzare l'indice } j)
    $$
    Questo si estende a tensori di rango superiore. Ad esempio, per un tensore $T^{ij}$: $T^i_j = g_{jk} T^{ik}$, $T_{ij} = g_{ik} T^k_j = g_{ik} g_{jl} T^{kl}$, ecc.

    \begin{tipsblock}
    L'operazione di abbassare l'indice di un vettore controvariante $v^i$ per ottenere $v_k = g_{ik} v^i$ corrisponde geometricamente a calcolare la proiezione del vettore $\bar{v}$ sulla direzione definita dal vettore di base $\bar{x}_k$: $v_k \equiv \bar{v} \cdot \bar{x}_k$. Questo sarà utile nell'interpretazione grafica.
    \end{tipsblock}

\end{itemize}

\subsection{Rappresentazione Geometrica delle Componenti Vettoriali}

Mentre un vettore $\bar{A}$ è un'entità geometrica indipendente dalle coordinate, le sue \emph{componenti} dipendono dal sistema di coordinate e dalla base scelta. Capire la differenza tra componenti controvarianti ($A^i$) e covarianti ($A_i$) può essere più intuitivo con un esempio grafico.

Consideriamo un piano con un sistema di coordinate $u^1, u^2$ non necessariamente ortogonali. In un punto $P$, i \textbf{vettori di base covarianti} $\bar{x}_1 = \partial \bar{r}/\partial u^1$ e $\bar{x}_2 = \partial \bar{r}/\partial u^2$ sono tangenti alle linee coordinate passanti per $P$. Questi vettori formano la \textit{base naturale} o \textit{base covariante}.

Un vettore $\bar{A}$ può essere scomposto in questa base:
$$ \bar{A} = A^1 \bar{x}_1 + A^2 \bar{x}_2 = A^i \bar{x}_i $$
Le $A^i$ sono le \textbf{componenti controvarianti}. Geometricamente, esse corrispondono alle componenti ottenute scomponendo il vettore $\bar{A}$ tramite la \emph{regola del parallelogramma}, usando direzioni parallele ai vettori di base $\bar{x}_i$. Sono le "usuali" componenti vettoriali.

\begin{figure}[H]
    \centering
    \includegraphics[width = 0.6\textwidth]{assets/tensors.png}
    \caption{Rappresentazione geometrica delle componenti controvarianti ($A^i$) e covarianti ($A_i$) di un vettore $\bar{A}$ in una base non ortogonale ($\bar{x}_1, \bar{x}_2$). Le componenti controvarianti sono ottenute con proiezioni parallele agli assi, mentre quelle covarianti con proiezioni ortogonali ai vettori della base duale (non mostrata) o equivalentemente $A_i = \bar{A} \cdot \bar{x}_i$.}
\end{figure}

Le \textbf{componenti covarianti} $A_i$ sono definite dal prodotto scalare del vettore $\bar{A}$ con i vettori della base covariante:
$$ A_i = \bar{A} \cdot \bar{x}_i $$
Geometricamente, $A_i$ rappresenta la \emph{proiezione ortogonale} del vettore $\bar{A}$ sulla direzione del vettore di base $\bar{x}_i$, moltiplicata per la lunghezza di $\bar{x}_i$ (infatti $A_i = |\bar{A}| |\bar{x}_i| \cos \theta_i$). 

\textbf{Attenzione:} questo \underline{non} corrisponde alla proiezione ortogonale sugli assi coordinati, a meno che la base non sia ortonormale.

\begin{itemize}
    \item Se la base $\{\bar{x}_i\}$ è \textbf{ortonormale} ($g_{ij} = \delta_{ij}$), allora $\bar{A} = A^i \bar{x}_i$ e $A_i = \bar{A} \cdot \bar{x}_i = (A^j \bar{x}_j) \cdot \bar{x}_i = A^j (\bar{x}_j \cdot \bar{x}_i) = A^j \delta_{ji} = A^i$. In questo caso, e solo in questo caso, le componenti controvarianti e covarianti coincidono.
    
    \item Se la base è solo \textbf{ortogonale} ($g_{ij} = 0$ per $i \neq j$), le componenti controvarianti e covarianti sono proporzionali: $A_i = g_{ii} A^i$ (no somma).
    
    \item Nel caso generale \textbf{non ortogonale}, le componenti sono distinte e legate dalla metrica: $A_i = g_{ij} A^j$.
\end{itemize}

\begin{observationblock}[Principio di Covarianza Generale]
La distinzione tra componenti controvarianti e covarianti è cruciale in Relatività Generale. Il \textbf{Principio di Covarianza Generale} afferma che le leggi fisiche devono essere espresse da equazioni tensoriali. Un'equazione tra tensori (es. $A^\alpha_{\beta \gamma} = B^\alpha_{\beta\gamma}$) è valida in un sistema di coordinate se e solo se è valida in tutti i sistemi di coordinate, poiché entrambi i membri si trasformano allo stesso modo. Questo garantisce che le leggi fisiche siano indipendenti dalla scelta (arbitraria) delle coordinate.
\end{observationblock}

\begin{exampleblock}[Vettori nel piano in coordinate polari]
    \begin{minipage}[H]{0.64\textwidth}
        $$
        ds^2 = dr^2 + r^2 d\theta^2 \quad \rightarrow \quad
        u^1 \equiv r \qquad u^2 = \theta
        $$
        \vspace{0.3em}
        $$
        g_{ij} = \begin{pmatrix} 1 & 0 \\ 0 & r^2 \end{pmatrix} 
        \quad
        g^{ij} = \begin{pmatrix} 1 & 0 \\ 0 & \frac 1{r^2} \end{pmatrix}    
        $$
        \vspace{0.3em}
        $$
        g = r^2 \ \ \rightarrow \ \ \sqrt{g} = r
        $$

        \vspace{1.5em}

        Dati $A_i = (5,9)$ e $B^i = (3, 7)$, abbiamo:
    \end{minipage}%
    \begin{minipage}[H]{0.35\textwidth}
        \centering
        \includegraphics[width = 0.9\textwidth]{assets/vec.png}
    \end{minipage}
    \vspace{0.5em}
    $$A_i B^i = A_1B^1 + A_2B^2 = 5 \cdot 3 + 9 \cdot 7 = 78$$
    in particolare:
    $$
    \begin{array}{rcl}
        A^i = g^{ij} A_j & \quad \rightarrow \quad & A^1 = g^{11}A_1 + g^{12}A_2 = 1 \cdot 5 + 0 \cdot 9 = 5\\
        & \rightarrow & A^2 = g^{21}A_1 + g^{22}A_2 = 0 \cdot 5 + 1/r^2 \cdot 9 = 9 / r^2 \\
        B_i = g_{ij} B^j & \rightarrow & B_1 = g_{11}B^1 + g_{12}B^2 = 1 \cdot 3 + 0 \cdot 7 = 3\\
        & \rightarrow & B_2 = g_{21}B^1 + g_{22}B^2 = 0 \cdot 3 + r^2 \cdot 7 = 7 r^2
    \end{array}
    $$

    Quindi:
    $$
    A^iB_i = A^1B_1 + A^2B_2 = 5 \cdot 3 + 9 / r^2 \cdot 7 r^2 = 78 = A_iB^i
    $$
    Cioè $A^iB_i = g_{ij}A^iB^j = g^{ij}A_jB_i  =A_iB^i$ è invariante.

    \vspace{1em}

    \underline{\textbf{Nota}}: Abbiamo che $dS = \sqrt{g} \dd u^1 \dd u^2 \to r \dd r \dd \theta$ è l'elemento di superficie.
\end{exampleblock}

\newpage

\section{Curvatura di una Superficie}
\label{sec:curvatura_superficie}

Dopo aver definito la metrica intrinseca di una superficie tramite la prima forma fondamentale, possiamo esplorare un concetto geometrico fondamentale: la \bfit{curvatura}. Mentre la curvatura di una curva piana è descritta da un singolo numero (il raggio di curvatura o il suo inverso), la curvatura di una superficie in un punto è più complessa e dipende dalla direzione considerata.

\subsection{Curvature Normali e Principali}
\label{subsec:curvature_principali}

Consideriamo un punto $P$ su una superficie regolare $M$ e il versore normale $\hat{n}$ in $P$. Per ogni vettore tangente $\bar{v}$ in $P$, il piano definito da $\hat{n}$ e $\bar{v}$ interseca la superficie $M$ lungo una curva $\mathcal{C}_{\bar{v}}$.

\begin{minipage}[H]{0.62\textwidth}
    Questa curva $\mathcal{C}_{\bar{v}}$ ha una sua curvatura $k$ nel punto $P$, definita come visto per le curve piane (\cref{sec:curva_piana}). La \textbf{curvatura normale} $k_n(\bar{v})$ della superficie $M$ in $P$ lungo la direzione $\bar{v}$ è definita come:
    $$ k_n(\bar{v}) = k \, \cos \alpha $$
    dove $\alpha$ è l'angolo tra il versore normale alla curva $\mathcal{C}_{\bar{v}}$ e il versore normale alla superficie $\hat{n}$ in $P$. Alternativamente, si può definire $k_n = \pm 1/R$, dove $R$ è il raggio di curvatura della curva $\mathcal{C}_{\bar{v}}$ e il segno dipende da quale lato giace il centro di curvatura rispetto a $\hat{n}$.

    \vspace{1em}

    Si dimostra (Teorema di Eulero sulla curvatura) che al variare della direzione del versore tangente $\bar{v}$ nel piano tangente in $P$, la curvatura normale $k_n(\bar{v})$ assume un valore massimo $k_1$ e un valore minimo $k_2$ in due direzioni tra loro ortogonali.
\end{minipage}%
\hspace{0.02\textwidth}
\begin{minipage}{0.35\textwidth}
    \centering
    \begin{figure}[H]
        \centering
        \includegraphics[width=0.95\textwidth]{assets/surface_curv.png}
        \caption{Curvatura normale $k_n$ in un punto $P$ lungo la direzione $\bar{v}$.}
        \label{fig:curvatura_sup}
    \end{figure}
\end{minipage}

$k_1$ e $k_2$ sono chiamate \textbf{curvature principali} e le rispettive direzioni sono dette \textbf{direzioni principali}.

\subsection{Curvatura Gaussiana e Curvatura Media}
\label{subsec:curvatura_gaussiana}

A partire dalle curvature principali, si definiscono due importanti misure scalari della curvatura in $P$:

\begin{itemize}
    \item La \textbf{Curvatura Gaussiana} $K$: È il prodotto delle curvature principali.
    \vspace{0.4em}
    $$ K = k_1 \cdot k_2 $$
    \vspace{-1.5em}

    La curvatura Gaussiana è una misura \emph{intrinseca} della curvatura (Teorema Egregium di Gauss): può essere calcolata conoscendo solo la prima forma fondamentale ($g_{ij}$) e le sue derivate, senza riferimento allo spazio ambiente $\mathbb{R}^3$. Questo la rende fondamentale in Relatività Generale.
    \vspace{0.5em}
    \item La \textbf{Curvatura Media} $H$: È la media aritmetica delle curvature principali.
    \vspace{0.4em}
    $$ H = \frac{k_1 + k_2}{2} $$
    \vspace{-1.5em}

    La curvatura media è una misura \emph{estrinseca}, dipende da come la superficie è immersa nello spazio ambiente.
\end{itemize}

\begin{figure}[H]
    \centering
    \includegraphics[width=0.8\textwidth]{assets/ex_curvature.png}
    \caption{Esempi di curvature principali e Gaussiana: (a) Piano: $k_1=k_2=0 \Rightarrow K=0$. (b) Sfera di raggio R: $k_1=k_2=1/R \Rightarrow K=1/R^2 > 0$. (c) Cilindro: $k_1=1/R, k_2=0 \Rightarrow K=0$.}
    \label{fig:ex_curvature}
\end{figure}

\begin{observationblock}[Curvatura e Forma Locale della Superficie]
    \label{ex:forma_locale}
    La curvatura Gaussiana $K$ fornisce informazioni qualitative sulla forma locale della superficie attorno a un punto $P$.
    
    \vspace{0.4em}

    \begin{itemize}
        \item Se $K > 0$ (punto ellittico), la superficie sta "tutta da una parte" rispetto al piano tangente $\pi$ in $P$ (almeno localmente), come nel caso della sfera.
        \item Se $K < 0$ (punto iperbolico o a sella), la superficie attraversa il piano tangente, assomigliando localmente a una sella. Un esempio è il paraboloide iperbolico $z = x^2 - y^2$.
        \item Se $K = 0$ (punto parabolico), la situazione è intermedia (come nel cilindro o nel piano).
    \end{itemize}

    \vspace{0.4em}

    Possiamo capire meglio questa relazione analizzando lo sviluppo in serie di Taylor della superficie attorno al punto $P$. 
    
    \vspace{0.4em}

    Supponiamo, per semplicità, di scegliere un sistema di coordinate $(x,y)$ tale che $P$ sia l'origine $(0,0)$ e il piano tangente in $P$ sia il piano $z=0$. 
    
    \begin{figure}[H]
        \centering
        \begin{minipage}{0.45\textwidth}
            \centering
            \includegraphics[width=0.95\textwidth]{assets/hyperbolic_point.png}
            \caption{Punto iperbolico ($K<0$)}
        \end{minipage}%
        \hspace{0.05\textwidth}%
        \begin{minipage}{0.45\textwidth}
            \centering
            \includegraphics[width=0.85\textwidth]{assets/taylor_surface.png}
            \caption{Sviluppo locale attorno a $P$}
        \end{minipage}
    \end{figure}
    \vspace{-1em}
    
    L'equazione locale della superficie sarà $z = f(x,y)$, con $f(0,0)=0$ e $\partial f/\partial x |_P = \partial f/\partial y |_P = 0$.
    
    Lo sviluppo di Taylor al secondo ordine è:
    \vspace{0.4em}
    $$
    z = \underbrace{f(0,0)}_{0} + \underbrace{\left. \frac{\partial f}{\partial x} \right|_P}_{0} x + \underbrace{\left. \frac{\partial f}{\partial y} \right|_P}_{0} y 
    + \frac{1}{2} \left[ \left. \frac{\partial^2 f}{\partial x^2} \right|_P x^2 + 2 \left. \frac{\partial^2 f}{\partial x \partial y} \right|_P xy + \left. \frac{\partial^2 f}{\partial y^2} \right|_P y^2 \right] + \mathcal{O}(3)
    $$
    Raccogliendo i termini del secondo ordine:
    \vspace{0.4em}
    $$
    z \approx \frac{1}{2} \left( a x^2 + 2b xy + c y^2 \right)
    $$
    dove $a = \partial^2 f/\partial x^2 |_P$, $b = \partial^2 f/\partial x \partial y |_P$, $c = \partial^2 f/\partial y^2 |_P$.
    
    L'equazione $ax^2 + 2bxy + cy^2 = \text{costante}$ (corrispondente a tagliare la superficie con un piano $z=\text{costante}$ vicino a $P$) descrive una conica nel piano $xy$. La natura di questa conica dipende dal segno del discriminante della forma quadratica, $\Delta = ac - b^2$. 
    
    \vspace{0.4em}
    
    Si può dimostrare che questo segno è direttamente legato al segno della curvatura Gaussiana $K$ in $P$:

    \vspace{0.4em}

    \begin{itemize}
        \item Se $ac - b^2 > 0$ (corrisponde a $K>0$): l'equazione descrive un'ellisse. La superficie è localmente concava o convessa come una scodella (punto ellittico).
        \item Se $ac - b^2 = 0$ (corrisponde a $K=0$): l'equazione descrive una parabola o linee parallele (punto parabolico).
        \item Se $ac - b^2 < 0$ (corrisponde a $K<0$): l'equazione descrive un'iperbole (punto iperbolico o a sella).
    \end{itemize}

    \vspace{0.4em}

    Inoltre, orientando opportunamente gli assi $x, y$ lungo le direzioni principali in $P$, il termine misto $b$ si annulla e lo sviluppo diventa $z \approx \frac{1}{2}(a' x'^2 + c' y'^2)$. Le costanti $a'$ e $c'$ sono direttamente proporzionali alle curvature principali $k_1$ e $k_2$.
\end{observationblock}

\newpage

\section{Geodetiche}
\label{sec:geodetiche}

\vspace{-1.3em}

\begin{minipage}{0.63\textwidth}
Intuitivamente, una \bfit{geodetica} su una superficie (o più in generale, su una varietà Riemanniana o pseudo-Riemanniana) è la curva che generalizza il concetto di "linea retta" in uno spazio curvo. È la curva di lunghezza "stazionaria" (spesso minima, ma non sempre) tra due punti sufficientemente vicini.

\vspace{0.5em}

Ad esempio su una sfera sono geodetiche tra i punti $P_1$ e $P_2$ sia $\mathcal C_1$ che $\mathcal C_2$ (entrambi archi di cerchio massimo), ma il cammino più breve corrisponde a $\mathcal C_1$ .
\end{minipage}
\hfill
\begin{minipage}{0.35\textwidth}
    \begin{figure}[H]
        \centering
        \includegraphics[width=0.8\textwidth]{assets/geodetica.png}
        % \caption{Geodetiche su una superficie sferica.}
        \label{fig:geodetiche}
    \end{figure}
\end{minipage}

\vspace{-1.5em}

\subsubsection{Definizione Variazionale}
\label{subsec:definizione_variazionale}

Consideriamo una curva $\bar{r}(t) = \bar{x}(u^i(t))$ su una superficie, che collega due punti $P_1 = \bar{r}(t_1)$ e $P_2 = \bar{r}(t_2)$. 
\vspace{0.6em}
La lunghezza di questa curva è data dall'integrale dell'elemento di lunghezza $ds$:
\vspace{0.5em}
$$ 
ds^2 = g_{jk} du^j du^k
\qquad \Rightarrow \qquad
S = \int_{P_1}^{P_2} ds = \int_{t_1}^{t_2} \sqrt{g_{jk}(u^l) \dfrac{du^j}{dt} \dfrac{du^k}{dt}}\, dt
$$

Definiamo il Lagrangiano $L$ come l'integrando (assumendo che $g_{jk}$ non dipenda esplicitamente da $t$):
\vspace{0.5em}
$$
L(u^i, \dot{u}^i) = \sqrt{g_{jk}(u^l) \dot{u}^j \dot{u}^k} \equiv \sqrt{F} \qquad \text{dove } \dot{u}^i = \frac{du^i}{dt}
$$

\begin{definitionblock}[Geodetiche]
Una curva è una geodetica se la sua lunghezza $S$ è stazionaria rispetto a piccole variazioni della curva che mantengano fissi gli estremi $P_1$ e $P_2$. Dal calcolo delle variazioni, ciò avviene se le coordinate $u^i(t)$ soddisfano le \textbf{equazioni di Eulero-Lagrange}:
\vspace{0.5em}
$$
\boxed{ \frac{\partial L}{\partial u^i} - \frac{d}{dt} \left( \frac{\partial L}{\partial \dot{u}^i} \right) = 0 \quad \text{per ogni } i }
$$
\end{definitionblock}

\subsection{Derivazione dell'Equazione delle Geodetiche}
\label{subsec:derivazione_geodetiche_notes}

Calcoliamo i termini delle equazioni di Eulero-Lagrange usando $L = \sqrt{F}$.

\begin{enumerate}
    \item \textbf{Calcolo di $\partial L / \partial u^i$}:
        \vspace{0.5em}
        $$ \frac{\partial L}{\partial u^i} = \frac{1}{2\sqrt{F}} \frac{\partial F}{\partial u^i} = \frac{1}{2L} \frac{\partial g_{jk}}{\partial u^i} \dot{u}^j \dot{u}^k $$

    \item \textbf{Calcolo di $\partial L / \partial \dot{u}^i$}:
        \vspace{0.5em}
        $$ \frac{\partial L}{\partial \dot{u}^i} = \frac{1}{2\sqrt{F}} \frac{\partial F}{\partial \dot{u}^i} = \frac{1}{2L} (g_{ik} \dot{u}^k + g_{ji} \dot{u}^j) $$
        Usando la simmetria $g_{ji}=g_{ij}$ e scambiando gli indici muti $k \leftrightarrow j$ nel secondo termine:
        \vspace{0.5em}
        $$ \frac{\partial L}{\partial \dot{u}^i} = \frac{1}{2L} (g_{ik} \dot{u}^k + g_{ik} \dot{u}^k) = \frac{1}{L} g_{ik} \dot{u}^k $$

    \item \textbf{Equazione di Eulero-Lagrange}: Sostituendo i termini calcolati:
        \vspace{0.5em}
        $$ \frac{1}{2L} \frac{\partial g_{jk}}{\partial u^i} \dot{u}^j \dot{u}^k - \frac{d}{dt} \left( \frac{g_{ik} \dot{u}^k}{L} \right) = 0 $$

    \item \textbf{Semplificazione con Parametrizzazione Affine}:
    L'equazione si semplifica notevolmente se scegliamo un parametro $t$ tale che sia proporzionale all'ascissa curvilinea $s$, cioè $s = \alpha t + \beta$. Per semplicità, assumiamo $t=s$ (scegliendo $\alpha=1, \beta=0$), che corrisponde a parametrizzare la curva con la sua lunghezza d'arco. In questo caso:
    \vspace{0.4em}
    $$ L = \sqrt{g_{jk} \frac{du^j}{ds} \frac{du^k}{ds}} = \frac{ds}{ds} = 1 \quad (\text{costante}) $$
    Poiché $L=1$, la derivata $\frac{dL}{dt} = \frac{dL}{ds} = 0$. L'equazione di Eulero-Lagrange diventa:
    \vspace{0.4em}
    $$ \frac{1}{2} \frac{\partial g_{jk}}{\partial u^i} \dot{u}^j \dot{u}^k - \frac{d}{ds} \left( g_{ik} \dot{u}^k \right) = 0 $$
    dove ora $\dot{u}^k = du^k/ds$. Calcolando la derivata totale rispetto a $s$:
    \vspace{0.4em}
    $$ \frac{1}{2} \frac{\partial g_{jk}}{\partial u^i} \dot{u}^j \dot{u}^k - \left( \frac{\partial g_{ik}}{\partial u^l} \frac{du^l}{ds} \dot{u}^k + g_{ik} \frac{d\dot{u}^k}{ds} \right) = 0 $$
    $$ \frac{1}{2} \frac{\partial g_{jk}}{\partial u^i} \dot{u}^j \dot{u}^k - \frac{\partial g_{ik}}{\partial u^l} \dot{u}^l \dot{u}^k - g_{ik} \ddot{u}^k = 0 $$
    dove $\ddot{u}^k = d^2u^k/ds^2$. Moltiplicando per -1 e riordinando:
    $$ g_{ik} \ddot{u}^k + \frac{\partial g_{ik}}{\partial u^l} \dot{u}^l \dot{u}^k - \frac{1}{2} \frac{\partial g_{lk}}{\partial u^i} \dot{u}^l \dot{u}^k = 0 $$
    (Nell'ultimo termine abbiamo scambiato gli indici muti $j \leftrightarrow l$).

    \item \textbf{Introduzione dei Simboli di Christoffel}:
    Vogliamo isolare $\ddot{u}^m$. Utilizziamo la seguente manipolazione algebrica basata sulla definizione dei simboli di Christoffel di prima specie $\Gamma_{lik} = \frac{1}{2} (\partial_l g_{ik} + \partial_k g_{il} - \partial_i g_{lk})$:
    $$ \frac{\partial g_{ik}}{\partial u^l} = \Gamma_{lik} + \Gamma_{kil} $$
    $$ \frac{1}{2} \frac{\partial g_{lk}}{\partial u^i} = \frac{1}{2} (\Gamma_{ilk} + \Gamma_{kil} - \Gamma_{lik}) $$ % Errata, la formula corretta è g_lk,i = Gamma_ilk + Gamma_kil
    In realtà, il modo più diretto (come suggerito dall'immagine 2) è riscrivere il termine $\frac{\partial g_{ik}}{\partial u^l} \dot{u}^l \dot{u}^k$ usando la simmetria degli indici $l, k$:
    $$ \frac{\partial g_{ik}}{\partial u^l} \dot{u}^l \dot{u}^k = \frac{1}{2} \left( \frac{\partial g_{ik}}{\partial u^l} + \frac{\partial g_{il}}{\partial u^k} \right) \dot{u}^l \dot{u}^k $$
    Sostituendo nell'equazione precedente:
    $$ g_{ik} \ddot{u}^k + \frac{1}{2} \left( \frac{\partial g_{ik}}{\partial u^l} + \frac{\partial g_{il}}{\partial u^k} - \frac{\partial g_{lk}}{\partial u^i} \right) \dot{u}^l \dot{u}^k = 0 $$
    Il termine tra parentesi quadre è proprio $\Gamma_{ilk}$. Quindi:
    $$ g_{ik} \ddot{u}^k + \Gamma_{ilk} \dot{u}^l \dot{u}^k = 0 $$
    (Abbiamo rinominato l'indice muto $j \to l$ rispetto alla formula $\Gamma_{ijk}$ per coerenza).

    \item \textbf{Equazione Finale}:
    Moltiplichiamo ora per $g^{mi}$ e sommiamo sull'indice $i$:
    $$ g^{mi} g_{ik} \ddot{u}^k + g^{mi} \Gamma_{ilk} \dot{u}^l \dot{u}^k = 0 $$
    $$ \delta^m_k \ddot{u}^k + \Gamma^m_{lk} \dot{u}^l \dot{u}^k = 0 $$
    dove abbiamo usato la definizione dei simboli di Christoffel di seconda specie $\Gamma^m_{lk} = g^{mi} \Gamma_{ilk}$. Ridenominando l'indice libero $m \to l$ e gli indici muti $(l,k) \to (j,k)$:
    $$ \boxed{ \frac{d^2u^l}{ds^2} + \Gamma^l_{jk} \frac{du^j}{ds} \frac{du^k}{ds} = 0 } $$
    Questa è l'\textbf{equazione delle geodetiche} quando la curva è parametrizzata rispetto alla sua lunghezza d'arco $s$.
\end{enumerate}

\begin{definitionblock}[Simboli di Christoffel]
    I \textbf{simboli di Christoffel di prima specie} sono:
    $$ \Gamma_{ijk} = \frac{1}{2} \left( \frac{\partial g_{ik}}{\partial u^j} + \frac{\partial g_{ji}}{\partial u^k} - \frac{\partial g_{jk}}{\partial u^i} \right) $$
    I \textbf{simboli di Christoffel di seconda specie} (o \bfit{connessione affine} di Levi-Civita) sono:
    $$ \Gamma^l_{jk} = g^{li} \Gamma_{ijk} = \frac{1}{2} g^{li} \left( \frac{\partial g_{ik}}{\partial u^j} + \frac{\partial g_{ji}}{\partial u^k} - \frac{\partial g_{jk}}{\partial u^i} \right) $$
    Entrambi sono simmetrici negli ultimi due indici inferiori: $\Gamma_{ijk} = \Gamma_{ikj}$ e $\Gamma^l_{jk} = \Gamma^l_{kj}$. Per convenienza, si usa spesso la notazione con la virgola per le derivate parziali: $\frac{\partial g_{ik}}{\partial u^j} \equiv g_{ik,j}$.
\end{definitionblock}

\begin{observationblock}[Natura non Tensoriale dei Simboli di Christoffel]
    È fondamentale notare che i simboli di Christoffel $\Gamma^l_{jk}$, nonostante abbiano tre indici, \textbf{non sono le componenti di un tensore}. La loro legge di trasformazione sotto un cambio di coordinate $u^i \to u'^m$ è più complessa:
    $$ \Gamma'^p_{mn} = \frac{\partial u'^p}{\partial u^l} \frac{\partial u^j}{\partial u'^m} \frac{\partial u^k}{\partial u'^n} \Gamma^l_{jk} + \frac{\partial u'^p}{\partial u^l} \frac{\partial^2 u^l}{\partial u'^m \partial u'^n} $$
    La presenza del secondo termine (che coinvolge le derivate seconde delle funzioni di trasformazione) impedisce ai $\Gamma^l_{jk}$ di trasformarsi come un tensore. Tuttavia, l'intera equazione delle geodetiche è un'equazione tensoriale (uguaglianza tra due tensori nulli, o meglio, tra l'accelerazione covariante e zero), garantendo la sua validità in ogni sistema di coordinate.
\end{observationblock}