\newpage

\chapter{Lezione 3: 11/03/2025}

\section{Tensori Controvarianti e Covarianti}

\subsubsection{Tensori Controvarianti}

$$
u^i = (i = 1, \dots, n) \quad \Rightarrow \quad u'^j = (j = 1, ..., n)
$$

Regola di trasformazione:

$$
du'^j = \sum_{i=1}^{n} \dfrac {\partial u'^j}{\partial u^i} du^i
$$

Otteniamo il \textbf{tensore controvariante}:

$$
\boxed{
V^{j} = \sum_{i=1}^{n} \dfrac {\partial u'^j}{\partial u^i} V^{i} 
}
$$

\subsubsection{Tensori Covarianti}

Consideriamo ora un campo scalare $\Phi$:

$$
% \begin{array}{rcl}
\partial \Phi_{'j} = \dfrac{\partial \Phi} {\partial u'^j} 
= \sum_i \dfrac{\partial \Phi} {\partial u'} \dfrac{\partial u'} {\partial u'^j}
= \sum_i \dfrac{\partial u'} {\partial u'^j} \dfrac{\partial \Phi} {\partial u'} 
= \sum_i \dfrac{\partial u'} {\partial u'^j} \partial \Phi_{'i} = W
% \end{array}
$$

Otteniamo il \textbf{tensore covariante}:

$$
\boxed{
W_{ij} = \sum_i \dfrac{\partial u'} {\partial u'^j} W_j
}
$$

$$
ds^2 = g_{ij} du^i du^j = g_{ij} \dfrac {\partial u^i}{\partial u'^l} du'^l \dfrac {\partial u^j}{\partial u'^k} du'^k = g'_{lk} du'^l du'^k 
$$

$$
g'_{lk} = \dfrac {\partial u^i}{\partial u'^l} \dfrac {\partial u^j}{\partial u'^k} g_{ij} \quad \rightarrow \quad \text{"covariante"}
$$

$$
g'^{lk} = \dfrac {\partial u'^l}{\partial u^i} \dfrac {\partial u'^k}{\partial u^j} g^{ij} \quad \rightarrow \quad \text{"controvariante"}
$$

Se un tensore ha sia indici covarianti che controvarianti, allora si dice \textbf{misto}. In tal caso è necessario applicare una trasformazione controvariante per gli indici controvarianti e una covariante per gli indici covarianti:

$$
V'^k_l = \dfrac {\partial u'^k}{\partial u^j} \dfrac {\partial u^i}{\partial u'^l} V^j_i
$$

\begin{observationblock}[Tensori e indici]
    Un tensore ha \textbf{rango} $n$ se ha $n$ indici, sia covarianti che controvarianti.

    Nota: \textit{Non tutti gli ogetti che hanno indici sono tensori}
\end{observationblock}

\newpage

Consderiamo un tensore covariante $D_i$ e un tensore metrico $g_{ij}$:

$$
D_i = g_{ij} C^j
$$


$$
D_i D^i = g_{ij} C^j D^i \quad \Rightarrow \quad 
\begin{cases}
    D_i = g_{ij} C^j\\
    C^j = D^j
\end{cases}
\quad \rightarrow \quad 
D_i = g_{ij} D^j
$$

\vspace{5em}

$$
\vec v = v^i \bar x_1
$$

$$
v_m = \vec v^i \cdot \bar x_k = v \bar x_i \bar x_k = v^i g_{ik}
$$

Consideriamo un vettore $A$:

...
