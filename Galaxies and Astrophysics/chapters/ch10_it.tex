\chapter{Lecture 15/05/2025}

Una massa singola, ferma nello spazio (Mass Monopole):
$$
\int \rho(x) \, d^3x = M
$$
non genera una  perturbazione del tensore metrico, dunque non genera onde gravitazionali.

Una massa in movimento (Mass Dipole):

\dots

La \textbf{configurazione quadrupolare} è la principale sorgente di onde gravitazionali nei sistemi astrofisici a bassa frequenza, poiché i termini di ordine inferiore (monopolo e dipolo) non producono radiazione gravitazionale. In questo schema, le masse in orbita o in vibrazione generano una variazione del \emph{momento di quadrupolo} del sistema, che si traduce in un'emissione di onde gravitazionali. Matematicamente, la \emph{formula del quadrupolo} (in approssimazione newtoniana e gauge di Lorentz) fornisce un'espressione per l'ampiezza delle onde, proporzionale alla seconda derivata del momento di quadrupolo e che decresce come $1/r$ con la distanza dalla sorgente. 

Questi oggetti emanano una radiazione di sincrotrone (osservabile dai segnali radio). Ogni qual volta ci arriva un raggio di tale tipo, sappiamo che il sistema ha compiuto un giro completo. La radiazione di sincrotrone è generata da particelle cariche che si muovono in un campo magnetico, e la sua emissione è associata a fenomeni astrofisici come le esplosioni di supernova o i getti relativistici emessi da buchi neri.

La prima osservazione diretta di onde gravitazionali è avvenuta il 14 settembre 2015, quando gli interferometri LIGO (Laser Interferometer Gravitational-Wave Observatory) hanno registrato le onde gravitazionali generate dalla fusione di due buchi neri. L'interferometro, in breve, funziona come segue:

\begin{itemize}
    \item 2 bracci lunghi 4 km
    \item misura le onde gravitazionali attraverso la variazione della lunghezza dei bracci
    \item il raggio laser viene diviso in due e inviato lungo i due bracci, dove viene riflesso da specchi e poi ricombinato
    \item la variazione della lunghezza dei bracci provoca un cambiamento nell'interferenza del raggio laser, che viene misurato
\end{itemize}

Quando i due oggetti si avvicinano, la loro forza gravitazionale aumenta, accelerando le masse e generando onde gravitazionali, questa fase prende il nome di \bfit{inspiral}. 
Quando i due corpi iniziano a fondersi, la loro velocità aumenta e la forza gravitazionale diventa così intensa che le onde gravitazionali emesse aumentano di intensità, questa fase prende il nome di \bfit{merger}. In questa fase, la radiazione gravitazionale emessa è così intensa che può essere rilevata anche a distanze cosmiche. Infine, quando i due oggetti si fondono in un unico corpo, si verifica una fase di \bfit{ringdown}, in cui il nuovo oggetto emette onde gravitazionali mentre si stabilizza.

Finora abbiamo potuto osservare qualche decina di fusione di buchi neri, e una sola fusione di stelle di neutroni. La fusione di stelle di neutroni, emettendo anche una controparte elettromagnetica, è stata osservata anche in altre lunghezze d'onda, come i raggi gamma e le onde radio, grazie alla sinergia tra diversi telescopi e osservatori.

Questi interferometri sono sensibili a corpi con masse di circa una decina di masse solari.

Buchi neri supermassicci (come quelli al centro di molte galassie), quando due galassie si fondono, generano onde gravitazionali a bassa frequenza, che non possono essere rilevate da interferometri come LIGO. Per questo, l'ESA e la NASA hanno in serbo un nuovo progetto, LISA (Laser Interferometer Space Antenna), che prevede il lancio di tre satelliti in formazione (triangolare, con distanza reciproca di $2.3\cdot 10^6 km$) per rilevare onde gravitazionali a bassa frequenza.

\section{Test classici}

\subsubsection{Precessione - Perielio di Mercurio (2° ordine)}

\subsubsection{Gravitational Redshift}

\subsubsection{Deflection of light - Gravitational lensing}

\subsubsection{Shapiro delay}

---

\section{MEtrica nel campo debole (stazionario)}

Scriviamo l'eq di Einstein in forma di onda:

$$
R_{\beta \delta} = -\dfrac 12 \square^2 h_{\beta \delta} = -\dfrac 12 \left( \dfrac 1c \dfrac {\partial^2 h_{\beta \delta}}{\partial t^2} - \nabla^2 h_{\beta \delta} \right)
$$

Se il campo è stazionario, la derivata temporale si annulla, e quindi possiamo scrivere:

$$
\dfrac 12 \nabla^2 h_{\beta \delta} = \dfrac {8 \pi G} {c^4} \left[T_{\beta \delta} - \dfrac 12 \eta_{\beta \delta} T^\gamma_\gamma\right]
$$

Dove $T_{\beta \delta}$ è il tensore energia-momento, che in questo caso è dato da:

$$
\begin{cases}
    T_{SILQ}^{\beta \delta} = \rho_0 c^2 u^\alpha u^\beta \\
    u^\alpha = (1, 0, 0, 0) \\
    T^\gamma_\gamma = \eta_{\alpha \gamma} T^{\gamma\alpha}  \\
\end{cases},
\qquad
T^{\beta \delta} = \begin{pmatrix}
    \rho_o c^2 & 0 & 0 & 0\\
    0 & 0 & 0 & 0\\
    0 & 0 & 0 & 0\\
    0 & 0 & 0 & 0
\end{pmatrix}
$$

Dunque otteniamo:
$$
R_{11} = \dfrac 12 \nabla^2 h_{11} = \dfrac {8 \pi G} {c^4} \left[\underbrace{T_{11}}_{0} - \dfrac 12 \rho_0 c^2 \right] = \dfrac{4 \pi G}{c^2} \rho_0
$$

\dots

$$
h_{11} = h_{22} = h_{33} = \dfrac{2 \phi}{c^2}, \quad h_{\alpha \beta} = 0 \ if\ \alpha \neq \beta
$$

\dots

$$
ds^2 = \left(1 + \dfrac {2 \phi}{c^2}\right) c^2 dt^2 - \left(1 - \dfrac {2 \phi}{c^2}\right)(dx^2 + dy^2 + dz^2)
$$

\dots

\newpage

$$
\dfrac{d^2x^\mu}{ds^2} + \Gamma^\mu_{\alpha\beta} \dfrac{dx^\alpha}{ds} \dfrac{dx^\beta}{ds} = 0
\quad \Rightarrow \quad
\dfrac{d^2x^\mu}{dt^2} = -  \Gamma^\mu_{\alpha\beta} \dfrac{dx^\alpha}{ds} \dfrac{dx^\beta}{ds}
$$

$$
\begin{array}{rcl}
\dfrac{dU^\mu}{ds} & = & - \Gamma^\mu_{\alpha\beta} U^\alpha U^\beta \\
& = &
\dfrac 12 g^{\mu\nu} \left[ g_{\alpha \beta, \nu} - g_{\beta \nu, \alpha} - g_{\alpha \nu, \beta} \right] U^\alpha U^\beta \\
& = &
\dfrac 12 g^{\mu\nu} g_{\alpha \beta, \nu} U^\alpha U^\beta - \dfrac 12 g^{\mu\nu} g_{\alpha \nu, \beta} U^\alpha U^\beta - \dfrac 12 g^{\mu\nu} g_{\beta \nu, \alpha} U^\alpha U^\beta
\end{array}
$$

$$
\dfrac {\dd U_{\sigma}}{\dd s} = \dfrac \dd {\dd s} \left( g_{\sigma \mu} U^\mu \right) = \dfrac{\dd g_{\sigma \mu}}{\dd s} U^\mu + g_{\sigma \mu} \dfrac {\dd U^\mu}{\dd s} = \dfrac{\dd g_{\sigma \mu}}{\dd x^\nu} \dfrac{\dd x^\nu}{\dd s} U^\mu + g_{\sigma \mu} \left[
    \dfrac 12 g^{\mu\nu} g_{\alpha \beta, \nu} U^\alpha U^\beta - g^{\mu\nu} g_{\alpha \nu, \beta} U^\alpha U^\beta 
\right]
$$

\dots

$$
\Rightarrow \dfrac{\dd U_{\sigma}}{\dd s} = \dfrac 12 g_{\alpha \beta, \sigma} U^\alpha U^\beta
$$

abbiamo dunque che il quadrimomento $P_0$ si conserva:

$$P_0(P) = P_0(\theta)$$

\dots

$$
E = P_\alpha U^\alpha, 
\qquad
ds^2 = g_{\mu \nu} dx^\mu dx^\nu,
\qquad
\dfrac{\dd x^\mu}{\dd s} \dfrac {\dd x^\nu}{\dd s} = \dfrac 1 {g_{\mu \nu}} = U^\mu U^\nu
$$

da cui

$$
(U^0)^2 = \dfrac 1{g_{00}}  \quad \Rightarrow \quad U^0 = \dfrac 1{\sqrt{g_{00}}} 
$$



e questo era il Pound Rebka test