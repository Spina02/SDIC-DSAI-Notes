\chapter{Relatività Generale}

La relatività generale è una teoria fisica che descrive la gravità come una manifestazione della curvatura dello spazio-tempo causata dalla presenza di massa ed energia. Sviluppata da Albert Einstein tra il 1907 e il 1915, questa teoria rappresenta una generalizzazione della relatività speciale e della legge di gravitazione universale di Newton.

A differenza della relatività speciale, che si occupa di sistemi di riferimento inerziali in assenza di gravità, la relatività generale estende questi concetti a sistemi di riferimento accelerati e include gli effetti gravitazionali. La teoria si basa sul principio di equivalenza, secondo il quale gli effetti della gravità sono localmente indistinguibili da quelli di un'accelerazione.

La relatività generale ha avuto un impatto profondo sulla nostra comprensione dell'universo, permettendo di spiegare fenomeni come la precessione dell'orbita di Mercurio, la deflessione della luce da parte di corpi massivi e il redshift gravitazionale. La teoria ha anche portato alla predizione di oggetti esotici come i buchi neri e le onde gravitazionali, quest'ultime osservate per la prima volta nel 2015.

\begin{tipsblock}[Convenzioni sugli indici]
    Nel seguito useremo, per convenzione, gli indici greci $\alpha, \beta, \gamma, \dots$
    se questi variano da 0 a 3, mentre useremo gli indici italici $i, j, k, \dots$ se questi variano da 1 a 3.
    \end{tipsblock}

\section{Lo spazio di Minkovski}

Nella relatività speciale, la distanza infinitesima tra due eventi è data da:
\vspace{0.4em}
$$
ds^2 = -c^2 dt^2 + (dx^2 + dy^2 + dz^2)
$$

Se definiamo $x^0 = ct; x^1 = x; x^2 = y; x^3 = z$, si può riscrivere l'espressione come:
\vspace{0.4em}
$$
ds^2 = \eta_{\alpha \beta} dx^\alpha dx^\beta
\qquad \text{con} \qquad
\eta_{\alpha \beta} = \begin{pmatrix}
    1 & 0 & 0 & 0 \\
    0 & -1 & 0 & 0 \\
    0 & 0 & -1 & 0 \\
    0 & 0 & 0 & -1
\end{pmatrix}
$$

Dove $\eta_{\alpha \beta}$ è il \bfit{tensore metrico dello spazio di Minkoswki}, che è "pseudo-euclideo", ma è comunque piatto: infatti gli $\eta_{\alpha \beta}$
sono costanti e quindi i $\Gamma^i_{jk}$ e $R^i_{jkl}$ sono nulli.

\begin{warningblock}[Attenzione ai segni]
    In letteratura si trova anche la convenzione opposta. Anche $\eta_{\alpha \beta}$ è definito spesso con i segni opposti, cioè con
    la segnatura $(-1, 1, 1, 1)$ invece di $(1, -1, -1, -1)$.
\end{warningblock}

Diciamo anche che l'intervallo $ds^2$ è:
\begin{itemize}
    \item di \textbf{tipo tempo} se $ds^2 > 0$ (corrispondente ad una traiettoria fisica con $v < c$)
    \item di \textbf{tipo spazio} se $ds^2 < 0$
    \item di \textbf{tipo luce} o \textbf{nulla} se $ds^2 = 0$ (corrispondente al moto di particelle, come i fotoni, con velocità $c$)
\end{itemize}

Se rappresenta lo spazio-tempo (eliminando una delle coordinate spaziali) attorno ad un evento preso come origine, si può dividerlo in tre zone definite dal cono in \cref{fig:cono-luce}:

\begin{center}
\begin{minipage}{0.6\textwidth}
\begin{itemize}
    \item \textbf{Futuro}: tutti i punti con cui l'evento in O può interagire tramite corpi che seguono una traiettoria fisica, cioè la regione che può essere raggiunta da segnali emessi dall'origine.
    \vspace{0.8em}
    \item \textbf{Passato}: tutti i punti da cui l'evento in O può essere influenzato, cioè la regione da cui possono provenire segnali che raggiungono l'origine.
    \vspace{0.8em}
    \item \textbf{Altrove}: tutti i punti che non possono né influenzare né essere influenzati dall'evento in O, poiché richiederebbero segnali con velocità maggiore di c. Un osservatore in moto relativo può vedere O e un punto in questa regione come eventi simultanei.
\end{itemize}
\end{minipage}
\hspace{1em}
\begin{minipage}{0.3\textwidth}
\vspace{-1.8em}
\begin{figure}[H]
    \centering
    \includegraphics[width=\textwidth]{assets/cono-luce.png}

    \vspace{-0.3em}
    \caption{Cono di luce}
    \label{fig:cono-luce}
\end{figure}
\end{minipage}
\end{center}

\vspace{0.5em}

Consideriamo ora un osservatore con un regolo (per misurare le distanze $\dd l$) e un orologio (pendolo, per misurare i tempi $\dd \tau$), il tempo scandito da questo orologio è detto \bfit{tempo proprio}. Consideriamo due eventi $A$ e $B$ che si verificano in tempi diversi ma nello stesso luogo ($dx = dy = dz = 0$). Si ha che tra $\dd s$ e $\dd t$ vale la relazione: $\dd s^2 = c^2 \dd t^2$.

La distanza $\dd s$ tra due medesimi eventi, per un osservatore per cui avvengono nello stesso punto, e per un altro osservatore per cui avvengono a distanza $\dd l$, sarà la stessa:
\vspace{0.4em}
$$
\dd s^2 = c^2 \dd \tau^2 = c^2 \dd t^2 - |\dd \bar l^2|
\qquad \to \qquad
\dd \tau^2 = \dd t^2 \left(
    1 - \dfrac 1{c^2} \dfrac{|\dd \bar l|}{\dd t} \cdot \dfrac{|\dd \bar l|}{\dd t}
\right) = \dd t^2 \left(
    1 - \dfrac {v^2}{c^2}
\right)
$$

dove $v$ è la velocità della particella per l'osservatore che la vede in moto, ed anche la velocità relativa dei due osservatori. Detto $\beta \equiv v/c$ e $\gamma \equiv 1/\sqrt{1 - \beta^2}$ si ha che $\dd t = \gamma \dd \tau$. Poichè $\gamma \geq 1$ allora $\dd t \geq \dd \tau$.

\begin{itemize}

\item Definiamo la \bfit{quadrivelocità} come $u^\alpha \equiv \frac{\dd x^\alpha}{\dd s}$; è un vettore poichè $\dd x^\alpha$ è un vettore e $\dd s$ è uno scalare.

In un sistema generico, non in quiete con la particella la quale ha una velocità $\bar v \equiv \dfrac{\dd \bar x}{\dd t}$, sarà:

$$
\begin{array}{l}
    u^0 = \dfrac{\dd x^0}{\dd s} = \dfrac{\dd(ct)}{c \dd \tau} = \gamma \\[0.6em]
    u^i = \dfrac{\dd x^i}{\dd s} = \dfrac{\dd x^i}{\dd t} \cdot \dfrac{\dd t}{\dd s} = \dfrac{v^i}{\gamma}
\end{array}
$$

per cui si può scrivere che $u^\alpha = \gamma (1, \bar \beta)$. Se la particella è in quiete, allora $u^\alpha = (1, 0, 0, 0)$.

La quantità $u^\alpha u_\alpha$ è invariante: $u^\alpha u_\alpha = \eta_{\alpha \beta} u^\alpha u^\beta = u^0u^0 - (u^1 u^1 + u^2 u^2 + u^3 u^3) = \gamma^2 - (\gamma^2 v^2/c^2) = 1$. $u^\alpha$ rappresenta il vettore (versore) tangente alla traiettoria della particella (nello spazio-tempo 4-D).
\vspace{0.5em}
\item Definiamo ora il \bfit{quadrimomento} $P^\alpha \equiv m u^\alpha$, dove $m_0$ è la massa a riposo della particella. Se si ricorda che $\bar P = m \bar v = \gamma m_0 \bar v$ e che $E = mc^2 = m_0c^2\gamma$, si ha:

$$
P^0 = \gamma m_0 = \dfrac{E}{c^2} \qquad \qquad P^i = \gamma m_0 \dfrac{v^i}{c} = m \dfrac{v^i}{c}
$$

$$
p^\alpha P_\alpha = \gamma^2 m_0^2 - \gamma^2 m_0^2 \dfrac{v^2}{c^2} = \gamma^2 m_0^2 \left(1 - \dfrac{v^2}{c^2}\right) = m_0^2
$$

$$
P^\alpha P_\alpha = m_0^2 = \dfrac{E^2}{c^4} - \dfrac{1}{c^2} \bar P \cdot \bar P
\qquad \to \qquad
m_0^2 c^2 = \dfrac{E^2}{c^2} - |\bar P|^2
$$

Possiamo quindi scrivere che $P^\alpha = cost$ e dunque $E = cost, \bar P = cost$: è la conservazione dell'energia e del momento (quantità di moto).
\vspace{0.5em}

\item Definiamo la \bfit{quadriaccelerazione} come $\dfrac{\dd^2 x^\alpha}{\dd s^2} = \dfrac{\dd u^\alpha}{\dd s}$. Nella relatività ristretta, mentre posizione, velocità e tempi sono relativi, l'accelerazione è assoluta: se è zero in un sistema di riferimento, lo è in qualsiasi altro sistema di riferimento inerziale (legato al primo da una trasformazione di Lorentz); questo perchè la quadriaccelerazione è un tensore, come $u^\alpha$.

\end{itemize}

\vspace{0.5em}

L'equazione per le \bfit{geodetiche} è sempre:
\vspace{0.4em}
$$
\dfrac{\dd^2 x^\alpha}{\dd s^2} + \Gamma^\alpha_{\beta \gamma} \dfrac{\dd x^\beta}{\dd s} \dfrac{\dd x^\gamma}{\dd s} = 0
$$

se la metrica è data da $\eta_{\alpha \beta}$, allora i $\Gamma^\alpha_{\beta \gamma}$ osno nulli e resta $d^2x^\alpha/ds^2 = 0$, cioè $x^\alpha = a^\alpha \cdot s + b^\alpha$, ovvero:

$$
\begin{cases}
    ct = a^0 \cdot s + b^0\\
    \bar x = \bar a \cdot s + \bar b
\end{cases}
$$

e la traiettoria è una retta percorsa di moto rettilineo uniforme.

Se si utilizza un altro tensore metrico, espresso ad esempio in coordinate polari sul piano (per semplicità): $ds^2 = c^2 dt^2 - (dr^2 + r^2 d\theta^2)$, i $\Gamma^\alpha_{\beta \gamma}$ non sono tutti nulli, ma si ottiene sempre come traiettoria una retta (ma ora in coordinate polari).

\begin{tipsblock}[Geodetiche nella Relatività]
    Mentre nello spazio 3-D Euclideo la geodetica tra due punti è un segmento di retta, quindi è la distanza minima tra due punti, nella Relatività la quantità $\int_B^A ds$ è massimo rispetto a variazioni di percorso con gli estremi fissi. È $\Delta \tau = \Delta s/c$, e si pensi al paradosso dei gemelli, nel quale il tempo è massimo per il gemello che è rimasto fermo.
\end{tipsblock}

\subsection{Tensore energia-impulso}

Per affrontare la Relatività Generale e la Cosmologia, abbiamo bisogno di un oggetto matematico che descriva le proprietà di un mezzo continuo, come densità e velocità, e che le colleghi alla conservazione dell'energia e della quantità di moto. Consideriamo inizialmente un fluido in cui l'unica forza presente è quella gravitazionale, chiamato "polvere" (dust), le cui particelle non interagiscono tra loro.

Il campo di materia sarà descritto in ogni punto dalla quadrivelocità $u^\alpha = \gamma(1, \bar v/c)$ e dalla densità propria $\rho_0(x)$, cioè quella misurata da un osservatore che segue il fluido. Con queste grandezze possiamo formare il \bfit{tensore energia-impulso}, che è il tensore più semplice che possiamo costruire per descrivere tali quantità:
\vspace{0.4em}
$$
T^{\alpha \beta} = \rho_0 c^2 u^\alpha u^\beta
$$
Nel dettaglio abbiamo:
\vspace{0.4em}
$$
T^{00} = \rho_0 c^2 \gamma^2 = \gamma^2 \rho_0 c^2 = \rho c^2 \qquad \text{posto } \rho = \gamma^2 \rho_0
$$
Per interpretare questo risultato si ricorda che la massa è $m = \gamma m_0$ (con $m_0$ massa a riposo) e che un elemento di volume in moto appare contratto di un fattore $1/\gamma$, per cui la densità cresce di un ulteriore fattore $\gamma$. Perciò, se la densità propria è $\rho_0$, un osservatore rispetto al quale il fluido ha velocità $\bar v$ misurerà una densità $\gamma^2 \rho_0$. $T^{00}$ misura quindi la \textbf{densità di massa-energia} (qui l'unico contributo all'energia viene dal moto della materia).

Le componenti del tensore energia-impulso $T^{\alpha \beta}$ possono essere espresse in forma matriciale come:

$$
T^{\alpha \beta} =
\rho c^2 \cdot
\begin{pmatrix}
    1 & v_x/c & v_y/c & v_z/c\\
    v_x/c & v_x^2/c^2 & v_x v_y/c^2 & v_x v_z/c^2\\
    v_y/c & v_x v_y/c^2 & v_y^2/c^2 & v_y v_z/c^2\\
    v_z/c & v_x v_z/c^2 & v_y v_z/c^2 & v_z^2/c^2
\end{pmatrix}
\qquad \text{(**)}
$$

Dalla conservazione del tensore energia-impulso, $\partial_\beta T^{\alpha \beta} = 0$ (o più in generale $T^{\alpha \beta}_{;\beta} = 0$ in presenza di gravità), possiamo ricavare le equazioni del moto del fluido. Analizziamo separatamente i casi $\alpha = 0$ e $\alpha = 1,2,3$:

\begin{itemize}
    \item Per $\bm{\alpha}\mathbf{= 0}$ si ha $\ \ \partial_\beta T^{0 \beta} = 0 \ \  \Leftrightarrow \ \ \frac {\partial T^{0\beta}}{\partial x^\beta} = 0\ \ $, che scritta per esteso:
    \vspace{0.4em}
    $$
    \dfrac 1c \dfrac {\partial(\rho c^2)}{\partial x} + \dfrac{\partial(\rho c v_x)}{\partial x} + \dfrac{\partial(\rho c v_y)}{\partial y} + \dfrac{\partial(\rho c v_z)}{\partial z} = 0
    $$
    Semplificando e dividendo per $c$, si ottiene l'\bfit{equazione di continuità}:
    $$
    \dfrac {\partial \rho}{\partial t} + \nabla \cdot (\rho \bar v) = 0
    $$
    dove $\bar v$ è la velocità totale. Questa equazione esprime la conservazione della massa-energia nel fluido.

    \item Per $\bm{\alpha} \mathbf{= 1, 2, 3}$ otteniamo le tre equazioni della quantità di moto:

    $$
    \begin{cases}
        \dfrac 1c \dfrac{\partial(\rho c v_x)}{\partial t} + \dfrac{\partial(\rho c v_x v_x)}{\partial x} + \dfrac{\partial(\rho c v_x v_y)}{\partial y} + \dfrac{\partial(\rho c v_x v_z)}{\partial z} = 0 \quad (\alpha = 1)
        \\[0.5em]
        \dfrac 1c \dfrac{\partial(\rho c v_y)}{\partial t} + \dfrac{\partial(\rho c v_y v_x)}{\partial x} + \dfrac{\partial(\rho c v_y v_y)}{\partial y} + \dfrac{\partial(\rho c v_y v_z)}{\partial z} = 0 \quad (\alpha = 2)
        \\[0.5em]
        \dfrac 1c \dfrac{\partial(\rho c v_z)}{\partial t} + \dfrac{\partial(\rho c v_z v_x)}{\partial x} + \dfrac{\partial(\rho c v_z v_y)}{\partial y} + \dfrac{\partial(\rho c v_z v_z)}{\partial z} = 0 \quad (\alpha = 3)
    \end{cases}
    $$  
\end{itemize}

Moltiplicando queste tre equazioni rispettivamente per i versori $\hat i, \hat j, \hat k$ e sommando membro a membro, otteniamo:
$$
\dfrac \partial {\partial t}(\rho \bar v) + \dfrac \partial {\partial x}(\rho v_x \bar v) + \dfrac \partial {\partial y}(\rho v_y \bar v) + \dfrac \partial {\partial z}(\rho v_z \bar v) = 0
$$
Sviluppando questa espressione e utilizzando l'equazione di continuità, si arriva a:
\vspace{0.4em}
$$
\rho \dfrac {\partial \bar v}{\partial t} + \bar v \left[
    \dfrac {\partial \rho}{\partial t} + \nabla \cdot (\rho \bar v)
\right] + 
\rho v_x \dfrac {\partial \bar v}{\partial x} +
\rho v_y \dfrac {\partial \bar v}{\partial y} +
\rho v_z \dfrac {\partial \bar v}{\partial z} = 0
$$
che può essere riscritta in due forme equivalenti:
\vspace{0.4em}
$$
\underbrace{\rho \left[
    \dfrac {\partial \bar v}{\partial t} + (\bar v \cdot \nabla) \bar v
\right] = 0}_{(I)}
\qquad \Leftrightarrow \qquad 
\underbrace{\rho \dfrac {d \bar v}{d t} = 0}_{(II)}
$$

\vspace{-0.5em}
Queste equazioni rappresentano il moto di un fluido ideale (senza pressione, viscosità e forze esterne) e esprimono la conservazione della quantità di moto.

\begin{tipsblock}[Equazione di continuità]
    La forma (I) corrisponde al punto di vista \bfit{Euleriano}, in cui si osserva il fluido da un punto fisso nello spazio, mentre la forma (II) corrisponde al punto di vista \bfit{Lagrangiano}, in cui si segue il moto delle singole particelle di fluido.
\end{tipsblock}

Vediamo quindi che il tensore $T^{\alpha \beta}$ esprime tutte le proprietà energetiche e dinamiche del fluido (polvere, non collisionale) in questione. $T^{\alpha \beta}$ è il \bfit{tensore energia-impulso}.

In un \textit{sistema inerziale localmente in quiete} (SILQ) rispetto al fluido, nel quale quindi $u^\alpha = (1, 0, 0, 0)$, $T^{\alpha \beta}$ ha la forma particolarmente semplice:
$$
T^{\alpha \beta}_{SILQ} =
\begin{pmatrix}
    \rho_0 c^2 & 0 & 0 & 0 \\
    0 & 0 & 0 & 0 \\
    0 & 0 & 0 & 0 \\
    0 & 0 & 0 & 0
\end{pmatrix}
$$
Veniamo adesso a considerare il caso in cui le particelle interagiscono nel modo più semplice, cioè attraverso urti dovuti all'agitazione termica: è presente una \textit{pressione} del fluido. Assumiamo che non vi sia trasporto di energia per conduzione o irraggiamento e non vi sia viscosità. Il \textbf{fluido} così definito è detto \textbf{perfetto}.

Se si è nel SILQ, ora $T^{\alpha \beta}$ non sarà più come quello scritto appena sopra, con solo $T^{00} \neq 0$. Le particelle avranno dei moti casuali attorno allo zero delle posizioni e delle velocità. Si deve rifare quindi alla forma precedente (**) di $T^{\alpha \beta}$, in cui però i termini che compaiono andranno mediati sul tempo e sulla distribuzione delle velocità delle particelle.

Questo però mi dà subito un'informazione importante: tutti i termini al di fuori della diagonale contengono elementi come $v_x, v_y$ o $v_z$ o loro prodotti; quando medio $\langle v_x \rangle = 0$ ed anche $\langle v_x v_y \rangle = \langle v_x \rangle \langle v_y \rangle = 0$ (assumendo $v_x$ e $v_y$ scorrelate). Quindi $T^{\alpha \beta}$ è diagonale in SILQ.

$T^{00}$ è il solo termine non nullo, e sarà quindi: SILQ (che esprime la densità di massa-energia) non sarà più semplicemente $\rho_0 c^2$, ma piuttosto un $\rho c^2$ con $\rho > \rho_0$ per tener conto del fatto che le particelle hanno velocità diverse da zero anche in SILQ e la loro densità di massa-energia è maggiore che nel caso della pura polvere. Per gli altri termini diagonali avremo $\langle \rho v_x^2 \rangle, \langle \rho v_y^2 \rangle, \langle \rho v_z^2 \rangle$.

Per interpretare questi termini facciamo una piccola digressione sulla teoria cinetica dei gas. 

\vspace{-1em}

\begin{figure}[H]
    \centering
    \includegraphics[width=0.39\textwidth]{assets/gas_perfetto.png}
    \caption{Forza esercitata da una particella su una superficie}
    \label{fig:gas-perfetto}
\end{figure}

\vspace{-1em}

Siano $v$ e $P$ la velocità e la quantità di moto di una particella, ed $f_z$ la forza media esercitata da questa particella perpendicolarmente alla superficie $A$ (vedi \cref{fig:gas-perfetto}):
\vspace{0.4em}
$$
\bar v = (v_x, v_y, v_z) 
\qquad
\bar P = (P_x, P_y, P_z)
\qquad
f_z = \dfrac{\Delta P_z}{\Delta t} = \dfrac{2P_z}{2L / v_z} = \dfrac{P_z v_z}{L}
$$
per una particella.

Per N particelle la forza sarà (indichiamo con p la pressione):
\vspace{0.4em}
$$
f_z = \dfrac N L P_z v_z = \dfrac N {L^3} P_z v_z L^2= p \cdot A
$$

per cui, facendo in realtà la media sulla distribuzione delle velocità, sarà (A = L$^2$):
\vspace{0.4em}
$$
\begin{array}{rcl}
 & p = \dfrac N{L^3} \langle P_z v_z\rangle = n\langle P_z v_z \rangle &
\\
& P \cdot v = P_x v_x + P_y v_y + P_z v_z = 3P_z v_z & \text{(per simmetria)}
\\
\text{e quindi} & p = \dfrac n3 \langle \bar P \cdot \bar v \rangle &
\end{array}
$$

che vale anche per un gas degenere e relativistico. 

Possiamo riscrivere questa relazione come $p = \dfrac n3 \langle \bar P \cdot \bar v \rangle = \langle n P_x v_x \rangle = \langle n m v_x^2 \rangle = \langle \rho v_x^2 \rangle$. Quindi, per un \bfit{fluido perfetto} ($\langle v_x \rangle = 0 \ \ \to \ \ \langle v_x v_y \rangle = \langle v_x \rangle \langle v_y \rangle = 0$), sarà:
\vspace{0.4em}
$$
T_{SILQ}^{\alpha\beta} = 
\begin{pmatrix}
    \rho c^2 & 0 & 0 & 0 \\
    0 & p & 0 & 0 \\
    0 & 0 & p & 0 \\
    0 & 0 & 0 & p
\end{pmatrix}
$$
dove $\rho$ tiene conto anche della massa-energia dovuta all'agitazione termica.

È facile verificare che in SILQ, si può sintetizzare il tutto con:
\vspace{0.5em}
$$
\phantom{\text{equazione tensoriale \quad}}
T_{SILQ}^{\alpha\beta} = (\rho + \rho c^2)u^\alpha u^\beta - p \eta^{\alpha\beta}
\quad
\text{(equazione tensoriale)}
$$

Infatti, ad esempio, per $T^{00}$, essendo $u^0=1$ e $\eta^{00}=1$, si ha $T^{00} = p + \rho c^2 - p = \rho c^2$.

Essendo quest'espressione tensoriale, essa \textbf{è valida in qualsiasi sistema di riferimento}, con $u^\alpha \not = (1, 0, 0, 0)$ ed eventualmente l'opportuno tensore metrico al posto di $\eta^{\alpha\beta}$. Scritta con gli indici covarianti sarà quindi:
\vspace{0.5em}
$$
\boxed{T_{\alpha\beta} = (\rho + \rho c^2)u_\alpha u_\beta - p g_{\alpha\beta}}
$$

\begin{exampleblock}[Equazioni dell'idrodinamica relativistica]
    Per soddisfare le equazioni dell'idrodinamica, deve valere:
    \vspace{0.4em}
    $$
    \partial_\beta T^{\alpha\beta}  = T^{\alpha\beta}_{;\beta} = 0
    $$
    Sostituendo l'espressione tensoriale di $T^{\alpha\beta}$ e sfruttando la conservazione del tensore energia-impulso, si ottiene:
    $$
    \partial_\beta T^{\alpha\beta} = \partial_\beta \left[ (\rho + \rho c^2)u^\alpha u^\beta - p \eta^{\alpha\beta} \right] = 0
    $$
    $$
    \dfrac{\partial_\beta}{\partial x^\beta} \left[ (\rho + \rho c^2)u^\alpha u^\beta \right] - \dfrac{\partial_\beta}{\partial x^\beta} (p \eta^{\alpha\beta}) = 0
    $$
    
    Questa equazione esprime la conservazione dell'energia e dell'impulso in un fluido perfetto relativistico. Nel limite non relativistico, si recupera la nota equazione di continuità $\partial_t \rho + \nabla \cdot (\rho \vec{v}) = 0$.
\end{exampleblock}

\begin{exampleblock}[Conservazione dell'entropia per particella]

    Vediamo ora di ricavare una relazione scalare dalla $\partial _\beta  T^{\alpha \beta }$ (rango 1, controvariante); per fare questo la moltiplicheremo per $u_\alpha $ (rango 1, covariante): Abbiamo che $u_\alpha T^{\alpha\beta}_{;\beta}$ è uno scalare!
    
    Partiamo dal fatto che, come abbiamo già visto, $u^\alpha u_\alpha  = 1$. Sarà quindi:
    \vspace{0.5em}
    $$
    \begin{array}{rcl}
    \dfrac{\partial}{\partial x^\beta} (u^\alpha u_\alpha) 
    & = & 
    u^\alpha \dfrac{\partial u_\alpha}{\partial x^\beta} + u_\alpha \dfrac{\partial u^\alpha}{\partial x^\beta}\\
    & = &
    \eta^{\alpha\gamma} u_\gamma \dfrac{\partial u_\alpha}{\partial x^\beta} + u_\alpha \dfrac {\partial u^\alpha}{\partial x^\beta}\\
    & = &
    u_\gamma \dfrac{\partial u^\gamma}{\partial x^\beta} + u_\alpha \dfrac{\partial u^\alpha}{\partial x^\beta}\\
    & = &
    2 u_\alpha \dfrac{\partial u^\alpha}{\partial x^\beta} = 0
    \end{array}
    $$

    \vspace{-0.5em}

    da cui $\frac {\partial u^\alpha}{\partial x^\beta} = 0$ (abbiamo sfruttato il fatto che $\alpha$ e $\gamma$ sono indici muti).
    
    Se riprendiamo l'equazione che esprime la divergenza di $T^{\alpha \beta}$ e la moltiplichiamo per $u_\alpha$ otteniamo:
    \vspace{0.5em}
    $$
    u_\alpha \dfrac{\partial}{\partial x^\beta} \big[(p + \rho c^2)u\alpha  u^\beta \big] - \dfrac {\partial p} {\partial x^\beta}  \eta^{\alpha \beta} u_\alpha   = 0
    $$

    e, sviluppando la derivata del primo termine, abbiamo:
    \vspace{0.4em}
    $$
    u_\alpha
    \left\{
        u^\alpha \dfrac {\partial} {\partial x^\beta} 
        [(p + \rho c^2)u^\beta  ] + (p + \rho c^2)u^\beta \cancelto{0}{\dfrac {\partial u^\alpha} {\partial x^\beta} }\ \ \ 
    \right\}
    - \dfrac{\partial p} {\partial x^\beta}  u^\beta  = 0
    $$

    \vspace{-0.7em}

    Se ricordiamo che $u^\alpha u_\alpha  = 1$ e che $\frac {\partial u^\alpha}{\partial x^\beta}  = 0$ possiamo scrivere:
    \vspace{0.4em}
    $$
    \begin{array}{l}
    \dfrac \partial {\partial x^\beta} 
    [(p + \rho c^2)u^\beta  ] - u^\beta \dfrac {\partial p}
    {\partial x^\beta}  = 0
    \\[1em]
    (p + \rho c^2) \underbrace{\dfrac {\partial u^\beta}{\partial x^\beta}}  + u^\beta  \dfrac \partial {\partial x^\beta}  (p + \rho c^2) - u^\beta \dfrac {\partial p} {\partial x^\beta}  = 0
    \end{array}
    $$
    Dalla \textit{conservazione del numero di particelle} abbiamo:
    \vspace{0.4em}
    $$
    \dfrac {\partial  (n u^\beta)}{\partial x^\beta}  = 0 
    \quad \Rightarrow \quad
    n \dfrac {\partial u^\beta}{\partial x^\beta}  + u^\beta \dfrac {\partial n} {\partial x^\beta}  = 0
    \qquad \Rightarrow \qquad
    \dfrac {\partial u^\beta }{\partial x^\beta}  = - \dfrac {u^\beta}n
    \dfrac {\partial n}{\partial x^\beta}
    $$
    Sostituendo quest'ultimo risultato nella precedente relazione e raccogliendo $u^\beta$  abbiamo:
    \vspace{0.4em}
    $$
    u^\beta \left\{ \underbrace{\dfrac {\partial (p + \rho c^2)}
    {\partial x^\beta} - \dfrac {p + \rho c^2}n
    \dfrac {\partial n}{\partial x^\beta}} - \dfrac {\partial p}{\partial x^\beta} 
    \right\}
    = 0
    $$

    \vspace{-0.5em}

    Osserviamo ora che:
    \vspace{0.4em}
    $$
    \dfrac{\partial}{\partial x^\beta} \left( \dfrac{p + \rho c^2}n \right) = \dfrac 1{n^2} \left[
        \dfrac{\partial(p + \rho c^2)}{\partial x^\beta}n - (p + \rho c^2) \dfrac{\partial n}{\partial x^\beta}
    \right] = \dfrac 1n \left[
        \dfrac{\partial (p + \rho c^2)}{\partial x^\beta} - \dfrac{p + \rho c^2}n \dfrac{\partial n}{\partial x^\beta}
    \right]
    $$
    Sostituendo questo risultato nella precedente relazione, otteniamo:
    \vspace{0.4em}
    $$
    u^\beta \left\{
        n\dfrac{\partial}{\partial x^\beta} \left(\dfrac pn \right) + \dfrac{\partial}{\partial x^\beta}\left(\dfrac{\rho c^2}n\right)
    \right\} = 0
    $$
    $$
    \Rightarrow \quad
    u^\beta \left\{
        np\dfrac{\partial}{\partial x^\beta} \left(\dfrac 1n \right) + \cancel{\dfrac nn \dfrac{\partial p}{\partial x^\beta}} + \dfrac{\partial}{\partial x^\beta}\left(\dfrac{\rho c^2}n\right)
        - \cancel{\dfrac{\partial p}{\partial x^\beta}}
    \right\} = 0
    $$
    $$
    \Rightarrow \quad
    n u^\beta \left\{
        p\dfrac{\partial}{\partial x^\beta} \left(\dfrac 1n \right) + \dfrac{\partial}{\partial x^\beta}\left(\dfrac{\rho c^2}n\right)
    \right\} = 0
    $$
    Ricordiamo ora il primo principio della termodinamica: $dU = dQ - dL$; se introduciamo l'entropia $S$ (con $dQ = TdS$ e $dL = pdV$) possiamo scrivere: $TdS = dU + pdV$, dove l'energia interna è $U = \rho c^2$. Se lo riscrivo riferendomi ad una particella, avrò $n = \frac{\#part}{V} = \frac{1}{V}$ e $Td\sigma = d \left(\frac{\rho c^2}n \right) + pd\left(\frac 1n \right)$, con $\sigma$ \textit{entropia per particella}. Sviluppando i differenziali si ha:
    \vspace{0.4em}
    $$
    Td\sigma = \dfrac {\partial \sigma}{\partial x^\beta}dx^\beta =
    \dfrac {\partial}{\partial x^\beta} \left(\dfrac{\rho c^2}n \right) dx^\beta + p \dfrac {\partial}{\partial x^\beta} \left( \dfrac 1n \right) dx^\beta \qquad / \cdot \dfrac 1{ds}
    $$
    Se ricordiamo che $\frac {d x^\beta}{ds} \equiv u^\beta$ e confrontiamo questa relazione con la precedente, otteniamo:
    \vspace{0.4em}
    $$
    u^\beta \dfrac {\partial \sigma}{\partial x^\beta} = 0
    $$
    che, sviluppando, diventa:
    $$
    \gamma \dfrac 1c \dfrac {\partial \sigma}{\partial t} + \gamma \dfrac {v_x}c \dfrac {\partial \sigma}{\partial x} + \gamma \dfrac {v_y}c \dfrac {\partial \sigma}{\partial y} + \gamma \dfrac {v_z}c \dfrac {\partial \sigma}{\partial z} = 0
    $$
    $$
    \dfrac{\partial\sigma}{\partial t} + (\bar v \cdot \nabla) \sigma = 0
    \qquad \Leftrightarrow \qquad
    \underbrace{\dfrac{d\sigma}{dt} = 0}_{\text{conservazione dell'entropia per particella}}
    $$

    dunque l'entropia si conserva se il numero di particelle non varia.
    
    \end{exampleblock}
    
    Si ha come risultato che, nel sistema in cui il fluido è in quiete, l'entropia per particella (o, se preferiamo, l'entropia per un certo numero $N$ di particelle contenute in un volume cubico $V$ di spigolo $L$, che può variare mantenendo però sempre al suo interno lo stesso numero di particelle) è costante. Questo è legato al fatto che, nel fluido ideale, non c'è scambio di energia per conduzione (o irraggiamento), né vi è dissipazione. Dal $1^{\circ}$ principio della termodinamica, nel sistema che segue il fluido, $dQ = dU + pdV$ e $U = \rho c^2 \cdot V$, per cui:
    \vspace{0.4em}
    $$
    dQ = \rho c^2 dV + Vd(\rho c^2) + pdV = (p + \rho c^2)dV + Vd(\rho c^2) = TdS
    $$
    Poichè $dQ = 0 \rightarrow dS = 0$, cioè l'entropia totale (o per particella) si conserva durante l'espansione adiabatica del fluido ideale.
    
    Se si scrive $p = w \rho c^2$ (con $w$ costante, anche se, più in generale, potrà dipendere dalla temperatura $w = w(T)$), si ottiene:
    $$
    (1 + w) \rho c^2 dV = - V d(\rho c^2)
    $$
    Dividendo entrambi i membri per $\rho c^2 V$ e integrando, si trova:
    $$
    \frac{d\rho}{\rho} = - (1 + w) \frac{dV}{V} \qquad \Rightarrow \qquad \rho V^{1 + w} = \text{costante}
    $$
    Questa relazione esprime come varia la densità di energia del fluido al variare del volume, a seconda del valore di $w$.
    
    Incontreremo tre casi interessanti in cosmologia:
    
    \begin{enumerate}
        \item Per un \textbf{gas non-relativistico} ($p \ll \rho_0 c^2$, quindi $\rho \approx \rho_0$), si ha $w \approx 0$ e quindi $\rho_0 V \approx \text{costante}$. Detto $L$ lo spigolo di un volume cubico $V = L^3$, si ottiene:
        $$
        \rho \propto \frac{1}{L^3}
        $$
        In questo caso, la densità di energia (che coincide praticamente con la densità di massa) diminuisce semplicemente perché il numero di particelle rimane costante mentre il volume aumenta. Questo è il comportamento tipico della materia ordinaria (polvere cosmica).
    
        \item Per un \textbf{gas di fotoni} (radiazione), la densità di energia è legata alla temperatura dalla legge di Stefan-Boltzmann: $\rho_{\text{rad}} \propto aT^4$, e la pressione è $p = \frac{1}{3} \rho c^2$, quindi $w = \frac{1}{3}$:
        $$
        T^4 V^{4/3} = \text{costante}
        \quad\Rightarrow\quad
        T \cdot V^{1/3} = \text{costante}
        \quad\Rightarrow\quad
        V \propto L^3
        \quad\Rightarrow\quad
        T \propto \frac{1}{L}
        $$
        Quindi, se il volume aumenta, la temperatura diminuisce come l'inverso della lunghezza caratteristica del sistema.  
        Inoltre:
        $$
        \rho_{\text{rad}} V^{4/3} = \text{costante}
        \quad\Rightarrow\quad
        V^{4/3} \propto L^4
        \quad\Rightarrow\quad
        \rho_{\text{rad}} \propto \frac{1}{L^4}
        $$
        Questo riflette il fatto che, oltre alla diluizione dovuta all'espansione del volume, l'energia dei fotoni diminuisce anche a causa del redshift cosmologico (allungamento della lunghezza d'onda).
    
        \item Se $p = - \rho c^2 \quad (w = -1)$, si ha:
        \vspace{0.4em}
        $$
        \rho V^0 = \text{costante} \qquad \Rightarrow \qquad \rho = \text{costante}
        $$
        In questo caso, la densità di energia non dipende dal volume: anche se l'universo si espande, la densità di energia rimane costante. Questo è il comportamento tipico dell'energia oscura (costante cosmologica).
    \end{enumerate}
    
    Possiamo riscrivere il primo principio in un'altra forma utile, ponendo $V \propto L^3$:
    \vspace{0.4em}
    $$
    \left( \rho + \frac{p}{c^2} \right) dV + V d\rho = 0 \qquad \rightarrow \qquad \left(\rho + \frac{p}{c^2}\right) \cdot 3L^2 dL + L^3 d\rho = 0
    $$
    da cui:
    $$
    3 \left(
        \rho + \frac{p}{c^2}
    \right) \frac{dL}{L} + d\rho = 0
    $$
    e, tenendo conto di una dipendenza di $L$ dal tempo:
    \vspace{0.4em}
    $$
    3 \left(
        \rho + \frac{p}{c^2}
    \right) \frac{dL}{dt} + \frac{d\rho}{dt} = 0
    $$
    Queste relazioni sono fondamentali in cosmologia per descrivere l'evoluzione della densità di energia dei diversi componenti dell'universo durante l'espansione.
    
    Abbiamo scritto $\partial_\alpha T^{\beta \alpha} = 0$ nello spazio di Minkowski; se però $\Gamma^\alpha_{\beta\gamma}$ non sono tutti nulli (cioè in uno spazio-tempo curvo), al posto della derivata parziale semplice si deve utilizzare la derivata covariante:
    \vspace{0.4em}
    $$
    T^{\alpha\beta}_{;\beta} = 0
    $$
    che esprime le leggi di conservazione in un sistema di riferimento generico.




