\chapter{Relatività Generale}

La relatività generale è una teoria fisica che descrive la gravità come una manifestazione della curvatura dello spazio-tempo causata dalla presenza di massa ed energia. Sviluppata da Albert Einstein tra il 1907 e il 1915, questa teoria rappresenta una generalizzazione della relatività speciale e della legge di gravitazione universale di Newton.

A differenza della relatività speciale, che si occupa di sistemi di riferimento inerziali in assenza di gravità, la relatività generale estende questi concetti a sistemi di riferimento accelerati e include gli effetti gravitazionali. La teoria si basa sul principio di equivalenza, secondo il quale gli effetti della gravità sono localmente indistinguibili da quelli di un'accelerazione.

La relatività generale ha avuto un impatto profondo sulla nostra comprensione dell'universo, permettendo di spiegare fenomeni come la precessione dell'orbita di Mercurio, la deflessione della luce da parte di corpi massivi e il redshift gravitazionale. La teoria ha anche portato alla predizione di oggetti esotici come i buchi neri e le onde gravitazionali, quest'ultime osservate per la prima volta nel 2015.

\begin{tipsblock}[Convenzioni sugli indici]
    Nel seguito useremo, per convenzione, gli indici greci $\alpha, \beta, \gamma, \dots$
    se questi variano da 0 a 3, mentre useremo gli indici italici $i, j, k, \dots$ se questi variano da 1 a 3.
    \end{tipsblock}

\section{Lo spazio di Minkovski}

Nella relatività speciale, la distanza infinitesima tra due eventi è data da:
\vspace{0.4em}
$$
ds^2 = -c^2 dt^2 + (dx^2 + dy^2 + dz^2)
$$

Se definiamo $x^0 = ct; x^1 = x; x^2 = y; x^3 = z$, possiamo riscrivere l'espressione come:
\vspace{0.4em}
$$
ds^2 = \eta_{\alpha \beta} dx^\alpha dx^\beta
\qquad \text{con} \qquad
\eta_{\alpha \beta} = \begin{pmatrix}
    1 & 0 & 0 & 0 \\
    0 & -1 & 0 & 0 \\
    0 & 0 & -1 & 0 \\
    0 & 0 & 0 & -1
\end{pmatrix}
$$

Abbiamo la metrica dello spazio di Minkoswki, che è "pseudo-euclideo", ma è comunque piatto: infatti gli $\eta_{\alpha \beta}$
sono costanti e quindi i $\Gamma^i_{jk}$ e $R^i_{jkl}$ sono nulli.

\begin{warningblock}[Attenzione ai segni]
    In letteratura si trova anche la convenzione opposta. Anche $\eta_{\alpha \beta}$ è definito spesso con i segni opposti, cioè con
    la segnatura $(-1, 1, 1, 1)$ invece di $(1, -1, -1, -1)$.
\end{warningblock}

Diciamo anche che l'intervallo $ds^2$ è:
\begin{itemize}
    \item di \textbf{tipo tempo} se $ds^2 > 0$ (corrispondente ad una traiettoria fisica con $v < c$)
    \item di \textbf{tipo spazio} se $ds^2 < 0$
    \item di \textbf{tipo luce} o \textbf{nulla} se $ds^2 = 0$ (corrispondente al moto di particelle, come i fotoni, con velocità $c$)
\end{itemize}

Se rappresento lo spazio-tempo (eliminando una delle coordinate spaziali) attorno ad un evento preso come origine, posso dividerlo in tre zone definite dal cono in \cref{fig:cono-luce}:

\begin{center}
\begin{minipage}{0.6\textwidth}
\begin{itemize}
    \item \textbf{Futuro}: tutti i punti con cui l'evento in O può interagire tramite corpi che seguono una traiettoria fisica, cioè la regione che può essere raggiunta da segnali emessi dall'origine.
    \vspace{1em}
    \item \textbf{Passato}: tutti i punti da cui l'evento in O può essere influenzato, cioè la regione da cui possono provenire segnali che raggiungono l'origine.
    \vspace{1em}
    \item \textbf{Altrove}: tutti i punti che non possono né influenzare né essere influenzati dall'evento in O, poiché richiederebbero segnali con velocità maggiore di c. Un osservatore in moto relativo può vedere O e un punto in questa regione come eventi simultanei.
\end{itemize}
\end{minipage}
\hspace{1em}
\begin{minipage}{0.3\textwidth}
\vspace{-1.8em}
\begin{figure}[H]
    \centering
    \includegraphics[width=\textwidth]{assets/cono-luce.png}

    \vspace{-1em}
    \caption{Cono di luce}
    \label{fig:cono-luce}
\end{figure}
\end{minipage}
\end{center}

Consideriamo ora un osservatore con un regolo (per misurare le distanze) e un orologio (per misurare i tempi). Consideriamo due eventi $A$ e $B$ che si verificano in tempi diversi ma nello stesso luogo ($dx = dy = dz = 0$). Si ha che tra $\dd s$ e $\dd t$ vale la relazione: $\dd s = c \dd t$.

La distanza ds tra due medesimi eventi, per un osservatore per cui avvengono nello stesso punto, e per un altro osservatore per cui avvengono a distanza $\dd l$, sarà la stessa:

$$
\dd s = c \dd \tau^2 = c^2 \dd t^2 - |\dd \bar l^2|
\qquad \to \qquad
\dd \tau^2 = \dd t^2 \left(
    1 - \dfrac 1{c^2} \dfrac{|\dd \bar l|}{\dd t} \cdot \dfrac{|\dd \bar l|}{\dd t}
\right) = \dd t^2 \left(
    1 - \dfrac {v^2}{c^2}
\right)
$$






