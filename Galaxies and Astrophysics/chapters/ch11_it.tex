\chapter{Cosmologia}

\section{Introduzione}

La cosmologia (\textit{Cosmos}: Universo, bellezza; \textit{Logos}: studio) è la scienza che studia l'origine, l'evoluzione e la struttura dell'universo. 

Le prime teorie cosmologiche risalgono a circa 2500 anni fa, quando i filosofi greci iniziarono a riflettere sulla natura dell'universo. 

Nel Rinascimento, Copernico rivoluzionò il pensiero cosmologico proponendo un modello eliocentrico, in cui il Sole occupa la posizione centrale invece della Terra. Keplero, analizzando con precisione le orbite planetarie, introdusse le leggi ellittiche del moto celeste, contribuendo a dare forma al nuovo paradigma. Galileo Galilei, con le sue osservazioni dettagliate tramite il telescopio, fornì evidenze empiriche a supporto di queste teorie, ribaltando definitivamente il vecchio modello geocentrico.


\begin{observationblock}[Paradosso di Olber]
\textbf{Paradosso di Olber}: Se l'universo fosse infinito, statico e omogeneo, ogni linea di vista dovrebbe colpire una stella, rendendo il cielo completamente luminoso. 
$$
f = \frac{L}{4 \pi d^2} \qquad dJ = \frac{L}{4 \pi d^2} nr^2 dr
$$
Universo infinito:
$$
J = \int_{r = 0}^{\infty} \frac{L}{4 \pi d^2} nr^2 dr = \infty
$$

Tuttavia, il cielo notturno è scuro. Questo paradosso suggerisce due possibili conclusioni:
\begin{enumerate}
    \item l'universo non è infinito
    \item l'universo non è statico
\end{enumerate}

\end{observationblock}

I primi studi sulle distanze nell'universo hanno segnato una svolta nell'astronomia. Una delle tecniche adottate è la \bfit{parallasse}, ovvero il cambiamento apparente della posizione di una stella rispetto allo sfondo, osservato quando la Terra si muove lungo la sua orbita. Questa tecnica ha permesso di determinare le distanze assolute delle stelle, aprendo la strada a una comprensione più profonda della struttura galattica.

William Herschel, attraverso osservazioni sistematiche, ha contribuito a delineare un quadro tridimensionale della Via Lattea. Utilizzando sia la distribuzione che l'intensità luminosa delle stelle, Herschel ha avanzato ipotesi sulla forma ed estensione della nostra galassia, gettando le basi per le future ricerche in cosmologia.

All'inizio del '900 ci stavano due teorie cosmologiche in competizione: Curtis sosteneva che la Via Lattea fosse una galassia isolata (e dunque coincidente con l'universo), mentre Shapley credeva che fosse solo una delle tante galassie nell'universo. 

\todo{collegamento tra i due paragrafi}

Il redshift indica lo spostamento verso il rosso delle linee spettrali, simile all'effetto Doppler: quando una sorgente luminosa si allontana dall'osservatore, la lunghezza d'onda aumenta, spostando lo spettro verso il rosso. In cosmologia, questo fenomeno è interpretato come l'effetto dell'espansione dell'universo.

Nel 1929, Edwin Hubble scoprì che la velocità di allontanamento delle galassie è proporzionale alla loro distanza dalla Terra, formulando la legge di Hubble:

$$
v = H_0 d
$$

dove $v$ è la velocità di recessione, $H_0$ è la costante di Hubble e $d$ è la distanza. Questa scoperta ha fornito una prova fondamentale dell'espansione dell'universo e ha portato alla formulazione del modello cosmologico attuale.

Le moderne teorie cosmologiche si basano su due assunzioni fondamentali: l'\textbf{omogeneità} e l'\textbf{isotropia} dell'universo su larga scala. L'omogeneità implica che la distribuzione della materia e dell'energia sia la stessa in ogni regione sufficientemente ampia dell'universo, mentre l'isotropia significa che l'universo appare uguale in tutte le direzioni. Queste ipotesi, note come \textit{Principio Cosmologico}, sono supportate dalle osservazioni e semplificano notevolmente la descrizione matematica dell'universo.

Si assume che l'universo sia spazialmente omogeneo e isotropo (su larga scala), il che implica che le leggi fisiche siano le stesse ovunque e in tutte le direzioni. 

\begin{observationblock}[Legge di Hubble e Big Bang]
    La legge di Hubble afferma che la velocità di recessione delle galassie è proporzionale alla loro distanza dalla Terra. Questa scoperta ha portato alla formulazione del modello del Big Bang, che descrive l'universo come in espansione a partire da uno stato iniziale estremamente denso e caldo.
\end{observationblock}

Prove dell'isotropia dell'universo:

- simmetria a 100 mpc della struttura dell'universo
- Il fondo cosmico a microonde, che segue uno spettro di corpo nero con una temperatura di circa 2.725 K, conferma l'omogeneità e l'isotropia dell'universo su larga scala.

---

Abbiamo prove evidenti (dalle osservazioni) che l'universo non è sempre stato come lo vediamo oggi. Inizialmente questo era infatti \bfit{opaco}, cioè non era possibile vedere oltre i 3000 anni luce. Questo perché la radiazione elettromagnetica non poteva viaggiare liberamente, essendo assorbita dalle particelle di materia. Col tempo l'universo si è espanso e raffreddato, permettendo alla radiazione di viaggiare liberamente. Questo fenomeno è noto come \bfit{decoupling} e ha portato alla formazione del fondo cosmico a microonde (CMB), una radiazione residua che permea l'universo e fornisce informazioni preziose sulla sua evoluzione.

---

\section{Principio Cosmologico}

"\textit{Ad ogni epoca fissata l'universo appare lo stesso in ogni punto, a parte le irregolarità locali}"

Immaginiamo di riempire fittamente questo spazio "smussato" e omogeneo di osservatori, ognuno con orologio e regoli, ognuno in quiete rispetto al moto medio della materia circostante. Le linee di universo (cioè le geodetiche) di questi osservatori non si intersecano, eccetto possibilmente in un punto singolare nel passato e, forse, nel futuro. C'è una sola geodetica che passa per un punto dello spazio-tempo, e quindi la materia possiede, in ogni punto, una ben definita velocità. Questo substrato "smussato" si comporta come un fluido perfetto. La regolarità del moto degli osservatori (postulato di Weyl) permette di definire, per ogni valore del tempo cosmico, una sezione spaziale $t = cost$ dello spazio-tempo. Queste sezioni spaziali sono perpendicolari alle geodetiche descritte dagli osservatori (vedi più avanti).

\begin{figure}[H]
    \centering
    \includegraphics[width=0.3\textwidth]{assets/principio_cosmologico.png}
\end{figure}

\dots

$$
\Delta \bar x \cdot \Delta \bar t = 0 = g_{0i} \Delta t^0 \Delta x^i \qquad \forall \Delta t^0, \ \forall \Delta x^i \qquad \Rightarrow \qquad g_{0i} = 0
$$

La metrica sarà del tipo:

$$
ds^2 = g_{00} (dx^0)^2 + g_{ij} dx^i dx^j \qquad (i,j = 1,2,3)
$$

Se noi prendiamo uno di questi osservatori comoventi, avremo che le sue coordinate spaziali sono costanti

Ciò non vuol dire c he la distanza relativa di questi osservatori non cambia, ma che la loro distanza relativa cambia in modo isotropo con l'espansione dell'universo.

\begin{figure}[H]
    \centering
    \includegraphics[width=0.3\textwidth]{assets/principio_cosmologico_2.png}
\end{figure}

--- [formalmente]

Se consideriamo infatti uno di questi osservatori O in quiete rispetto al moto medio locale della materia, la sua geodetica sarà per lui definita dalle condizioni xi = cost (i = 1, 2, 3) ; se consideriamo un osservatore vicino, che si trovi sulla stessa superficie t = t0 = cost, cioè x0 = cost, di O, il vettore ∆x che unisce l'evento O all'evento 0' sarà perpendicolare al vettore ∆t parallelo alla geodetica per O ed alla quadrivelocità di componenti (1, 0, 0, 0). Se fosse ∆t · ∆x 6 = 0 gli eventi O ed O' non sarebbero più contemporanei, perchè ∆x avrebbe una componente non nulla lungo l'asse dei tempi di O.

Questo ci permette di semplificare la scelta della metrica per l'osservatore O, che sarà in generale:
$$
ds^2 = g_{\alpha \beta} dx^\alpha dx^\beta
$$

\dots

Consideriamo un osservatore co-movente O. Le sue coordinate spaziali saranno xi = cost, per cui dxi = 0; l'intervallo ds2 tra due eventi successivi lungo la linea d'universo di O sarà quindi ds2 = g00(dx0)2, ma questo è anche uguale, per definizione, a c2dτ 2 con τ tempo proprio associato ad O:
$$
c^2 d\tau^2 = g_{00} (dx^0)^2
$$

Lo spazio è omogeneo, e questa relazione deve valere per qualunque osservatore, quali che siano le sue coordinate $x^i$, per cui $g_{00}$ deve dipendere solo da $x^0$. Possiamo quindi definire una nuova scala di tempo cosmico tale che:
$$
c \dd t = \sqrt{g_{00}} dx^0
$$
che coinciderà con il tempo proprio degli osservatori co-moventi e scriverò, usando t per indicare il tempo proprio:
$$
\dd s^2 = c^2 \dd t^2 + g_{ij} dx^i dx^j
$$

Un sistema di riferimento in cui sia $g_{00} \equiv 1$ e $g_{0i} \equiv 0$ è detto \textit{sincrono}. In questo caso le linee d'universo $x^i = cost$ sono linee geodetiche. Infatti il quadrivettore tangente alla linea d'universo $u^\alpha \equiv \dd x^\alpha / \dd s$ ha le componenti uguali a $(1, 0, 0, 0)$ e soddisfa automaticamente l'equazione delle geodetiche, perchè:

$$
\dfrac{\dd u^\alpha}{\dd s} + \Gamma^\alpha_{\beta \gamma} u^\beta u^\gamma \approxeq \Gamma^\alpha_{00}
$$

ma essendo $g_{00} = 1$, e $g_{0i} = 0$, abbiamo:

$$
\Gamma^\alpha_{00} = \dfrac 12 g^{\alpha \sigma} \left( \dfrac{\partial g_{\sigma 0}}{\partial x^0} + \dfrac{\partial g_{\sigma 0}}{\partial x^0} - \dfrac{\partial g_{00}}{\partial x^\sigma} \right) = 0
$$

\todo{missimg something}

