\chapter{Cosmologia}

\section{Introduzione}

La cosmologia (\textit{Cosmos}: Universo, bellezza; \textit{Logos}: studio) è la scienza che studia l'origine, l'evoluzione e la struttura dell'universo. 

Le prime teorie cosmologiche risalgono a circa 2500 anni fa, quando i filosofi greci iniziarono a riflettere sulla natura dell'universo. 

Nel Rinascimento, Copernico rivoluzionò il pensiero cosmologico proponendo un modello eliocentrico, in cui il Sole occupa la posizione centrale invece della Terra. Keplero, analizzando con precisione le orbite planetarie, introdusse le leggi ellittiche del moto celeste, contribuendo a dare forma al nuovo paradigma. Galileo Galilei, con le sue osservazioni dettagliate tramite il telescopio, fornì evidenze empiriche a supporto di queste teorie, ribaltando definitivamente il vecchio modello geocentrico.


\begin{observationblock}[Paradosso di Olber]
\textbf{Paradosso di Olber}: Se l'universo fosse infinito, statico e omogeneo, ogni linea di vista dovrebbe colpire una stella, rendendo il cielo completamente luminoso. 
$$
f = \frac{L}{4 \pi d^2} \qquad dJ = \frac{L}{4 \pi d^2} nr^2 dr
$$
Universo infinito:
$$
J = \int_{r = 0}^{\infty} \frac{L}{4 \pi d^2} nr^2 dr = \infty
$$

Tuttavia, il cielo notturno è scuro. Questo paradosso suggerisce due possibili conclusioni:
\begin{enumerate}
    \item l'universo non è infinito
    \item l'universo non è statico
\end{enumerate}

\end{observationblock}

I primi studi sulle distanze nell'universo hanno segnato una svolta nell'astronomia. Una delle tecniche adottate è la \bfit{parallasse}, ovvero il cambiamento apparente della posizione di una stella rispetto allo sfondo, osservato quando la Terra si muove lungo la sua orbita. Questa tecnica ha permesso di determinare le distanze assolute delle stelle, aprendo la strada a una comprensione più profonda della struttura galattica.

William Herschel, attraverso osservazioni sistematiche, ha contribuito a delineare un quadro tridimensionale della Via Lattea. Utilizzando sia la distribuzione che l'intensità luminosa delle stelle, Herschel ha avanzato ipotesi sulla forma ed estensione della nostra galassia, gettando le basi per le future ricerche in cosmologia.

All'inizio del '900 ci stavano due teorie cosmologiche in competizione: Curtis sosteneva che la Via Lattea fosse una galassia isolata (e dunque coincidente con l'universo), mentre Shapley credeva che fosse solo una delle tante galassie nell'universo. 

\todo{collegamento tra i due paragrafi}

Il redshift indica lo spostamento verso il rosso delle linee spettrali, simile all'effetto Doppler: quando una sorgente luminosa si allontana dall'osservatore, la lunghezza d'onda aumenta, spostando lo spettro verso il rosso. In cosmologia, questo fenomeno è interpretato come l'effetto dell'espansione dell'universo.

Nel 1929, Edwin Hubble scoprì che la velocità di allontanamento delle galassie è proporzionale alla loro distanza dalla Terra, formulando la legge di Hubble:

$$
v = H_0 d
$$

dove $v$ è la velocità di recessione, $H_0$ è la costante di Hubble e $d$ è la distanza. Questa scoperta ha fornito una prova fondamentale dell'espansione dell'universo e ha portato alla formulazione del modello cosmologico attuale.

Le moderne teorie cosmologiche si basano su due assunzioni fondamentali: l'\textbf{omogeneità} e l'\textbf{isotropia} dell'universo su larga scala. L'omogeneità implica che la distribuzione della materia e dell'energia sia la stessa in ogni regione sufficientemente ampia dell'universo, mentre l'isotropia significa che l'universo appare uguale in tutte le direzioni. Queste ipotesi, note come \textit{Principio Cosmologico}, sono supportate dalle osservazioni e semplificano notevolmente la descrizione matematica dell'universo.

Si assume che l'universo sia spazialmente omogeneo e isotropo (su larga scala), il che implica che le leggi fisiche siano le stesse ovunque e in tutte le direzioni. 

\begin{observationblock}[Legge di Hubble e Big Bang]
    La legge di Hubble afferma che la velocità di recessione delle galassie è proporzionale alla loro distanza dalla Terra. Questa scoperta ha portato alla formulazione del modello del Big Bang, che descrive l'universo come in espansione a partire da uno stato iniziale estremamente denso e caldo.
\end{observationblock}

Prove dell'isotropia dell'universo:

- simmetria a 100 mpc della struttura dell'universo
- Il fondo cosmico a microonde, che segue uno spettro di corpo nero con una temperatura di circa 2.725 K, conferma l'omogeneità e l'isotropia dell'universo su larga scala.

---

Abbiamo prove evidenti (dalle osservazioni) che l'universo non è sempre stato come lo vediamo oggi. Inizialmente questo era infatti \bfit{opaco}, meentre ora è \bfit{trasparente}. Questo significa che l'universo ha subito un'evoluzione nel tempo, passando da uno stato iniziale in cui la materia era concentrata e calda a uno stato attuale in cui è più dispersa e fredda.

---

\section{Principio Cosmologico}

Se vogliamo applicare la Relatività Generale (intesa come la miglior teoria disponibile per descrivere il moto dei corpi per effetto della distribuzione di materia) allo studio del cosmo, dovremo aspettarci che, in generale, la geometria dello spazio tempo non sia statica, ma dipenda dal tempo. Questo è anche suggerito dall'evidenza osservativa di un moto generale di allontanamento delle galassie da noi (legge di Hubble).

L'espansione dell'universo appare abbastanza regolare. Vi sono, a causa della presenza di disomogeneità (come gruppi, ammassi di galassie), delle perturbazioni nei moti delle galassie indotti dall'azione gravitazionale di queste disomogeneità. Ma questi moti sono relativamente "piccoli", con velocità dell'ordine di $100 \div 1000 km/s$, rispetto alle velocità di allontanamento (\textit{recessione}) da noi delle galassie che, nelle survey ottiche, arrivano anche a frazioni significative della velocità della luce. Inoltre questi moti sono generalmente non sistmatici.

\curlyquotes{\textit{Ad ogni epoca fissata l'universo appare lo stesso in ogni punto, a parte le irregolarità locali}}

Immaginiamo di riempire fittamente questo spazio "smussato" e omogeneo di osservatori, ognuno con orologio e regoli, ognuno in quiete rispetto al moto medio della materia circostante. Le linee di universo (cioè le geodetiche) di questi osservatori non si intersecano, eccetto possibilmente in un punto singolare nel passato e, forse, nel futuro. C'è una sola geodetica che passa per un punto dello spazio-tempo, e quindi la materia possiede, in ogni punto, una ben definita velocità. Questo substrato "smussato" si comporta come un fluido perfetto. La regolarità del moto degli osservatori (postulato di Weyl) permette di definire, per ogni valore del tempo cosmico, una sezione spaziale $t = cost$ dello spazio-tempo. Queste sezioni spaziali sono perpendicolari alle geodetiche descritte dagli osservatori (vedi più avanti).

\begin{figure}[H]
    \centering
    \includegraphics[width=0.3\textwidth]{assets/principio_cosmologico.png}
\end{figure}

\dots

$$
\Delta \bar x \cdot \Delta \bar t = 0 = g_{0i} \Delta t^0 \Delta x^i \qquad \forall \Delta t^0, \ \forall \Delta x^i \qquad \Rightarrow \qquad g_{0i} = 0
$$

La metrica sarà del tipo:

$$
ds^2 = g_{00} (dx^0)^2 + g_{ij} dx^i dx^j \qquad (i,j = 1,2,3)
$$

Se noi prendiamo uno di questi osservatori comoventi, avremo che le sue coordinate spaziali sono costanti

Ciò non vuol dire c he la distanza relativa di questi osservatori non cambia, ma che la loro distanza relativa cambia in modo isotropo con l'espansione dell'universo.

\begin{figure}[H]
    \centering
    \includegraphics[width=0.3\textwidth]{assets/principio_cosmologico_2.png}
\end{figure}

--- [formalmente]

Se consideriamo infatti uno di questi osservatori O in quiete rispetto al moto medio locale della materia, la sua geodetica sarà per lui definita dalle condizioni xi = cost (i = 1, 2, 3) ; se consideriamo un osservatore vicino, che si trovi sulla stessa superficie t = t0 = cost, cioè x0 = cost, di O, il vettore ∆x che unisce l'evento O all'evento 0' sarà perpendicolare al vettore ∆t parallelo alla geodetica per O ed alla quadrivelocità di componenti (1, 0, 0, 0). Se fosse ∆t · ∆x 6 = 0 gli eventi O ed O' non sarebbero più contemporanei, perchè ∆x avrebbe una componente non nulla lungo l'asse dei tempi di O.

Questo ci permette di semplificare la scelta della metrica per l'osservatore O, che sarà in generale:
$$
ds^2 = g_{\alpha \beta} dx^\alpha dx^\beta
$$

\dots

Consideriamo un osservatore co-movente O. Le sue coordinate spaziali saranno xi = cost, per cui dxi = 0; l'intervallo ds2 tra due eventi successivi lungo la linea d'universo di O sarà quindi ds2 = g00(dx0)2, ma questo è anche uguale, per definizione, a c2dτ 2 con τ tempo proprio associato ad O:
$$
c^2 d\tau^2 = g_{00} (dx^0)^2
$$

Lo spazio è omogeneo, e questa relazione deve valere per qualunque osservatore, quali che siano le sue coordinate $x^i$, per cui $g_{00}$ deve dipendere solo da $x^0$. Possiamo quindi definire una nuova scala di tempo cosmico tale che:
$$
c \dd t = \sqrt{g_{00}} dx^0
$$
che coinciderà con il tempo proprio degli osservatori co-moventi e scriverò, usando t per indicare il tempo proprio:
$$
\dd s^2 = c^2 \dd t^2 + g_{ij} dx^i dx^j
$$

Un sistema di riferimento in cui sia $g_{00} \equiv 1$ e $g_{0i} \equiv 0$ è detto \textit{sincrono}. In questo caso le linee d'universo $x^i = cost$ sono linee geodetiche. Infatti il quadrivettore tangente alla linea d'universo $u^\alpha \equiv \dd x^\alpha / \dd s$ ha le componenti uguali a $(1, 0, 0, 0)$ e soddisfa automaticamente l'equazione delle geodetiche, perchè:

$$
\dfrac{\dd u^\alpha}{\dd s} + \Gamma^\alpha_{\beta \gamma} u^\beta u^\gamma \approxeq \Gamma^\alpha_{00}
$$

ma essendo $g_{00} = 1$, e $g_{0i} = 0$, abbiamo:

$$
\Gamma^\alpha_{00} = \dfrac 12 g^{\alpha \sigma} \left( \dfrac{\partial g_{\sigma 0}}{\partial x^0} + \dfrac{\partial g_{\sigma 0}}{\partial x^0} - \dfrac{\partial g_{00}}{\partial x^\sigma} \right) = 0
$$

\todo{missimg something}

\subsection{Metrica di Robertson-Walker}

Usiamo anzitutto il fatto di avere una simmetria sferica dovuta all'isotropia scegliendo un sistema di coordinate sferico, che riflette questa simmetria. Restiamo, per ora, nello spazio euclideo e definiamo:

$$
\begin{cases}
    x = R \sin \theta \cos \varphi \\
    y = R \sin \theta \sin \varphi \\
    z = R \cos \theta
\end{cases}
$$

Definita in questo modo la superficie sferica è immediato ricavare il tensore metrico:

$$
\begin{array}{l}
    \bar x_R = (\sin \theta \cos \varphi, \sin \theta \sin \varphi, \cos \theta) \\
    \bar x_\theta = (R \cos \theta \cos \varphi, R \cos \theta \sin \varphi, -R \sin \theta) \\
    \bar x_\varphi = (-R \sin \theta \sin \varphi, R \sin \theta \cos \varphi, 0)
\end{array}
\qquad \Rightarrow \qquad
\ ^3g_{ij} =
\begin{pmatrix}
    1 & 0 & 0 \\
    0 & R^2 & 0 \\
    0 & 0 & R^2 \sin^2 \theta
\end{pmatrix}
$$

da cui:

$$
\dd l^2 = \dd R^2 + R^2 (\dd \theta^2 + \sin^2 \theta \dd \varphi^2) = \dd R^2 + R^2 \dd \Omega^2
$$

Per $R = cost$ sarà $dl^2 = R^2 \dd \Omega^2$, con $\ ^2g_{ij} = \begin{pmatrix} R^2 & 0 \\ 0 & R^2 \sin^2 \theta \end{pmatrix}$. Ricordiamo che l'area della superficie si può ottenere dalla relazione:

$$
dA = \sqrt{\,^2g}\ \dd \theta \dd \varphi = R^2 \sin \theta \dd \theta \dd \varphi
$$

da cui l'area della sfera è:

$$
A = \int_{0}^{2\pi} \int_{0}^{\pi} R^2 \sin \theta \dd \theta \dd \varphi = 4 \pi R^2
$$

Questa relazione si può generalizzare alle tre dimensioni ottenendo in questo caso un volume:

$$
dV = \sqrt{\,^3g}\ \dd R \dd \theta \dd \varphi = R^2 \sin \theta \dd R \dd \theta \dd \varphi
$$

da cui otteniamo:

$$
V = \dfrac 43 \pi R^3
$$

\todo{missing something}





\newpage

\chapter{Cosmologia}

La cosmologia (\textit{Cosmos}: Universo, bellezza; \textit{Logos}: studio) è la branca della fisica e dell'astronomia che indaga l'origine, l'evoluzione, la struttura su larga scala e il destino ultimo dell'Universo. Sebbene le prime riflessioni sulla natura del cosmo risalgano ai filosofi greci, la cosmologia moderna affonda le sue radici nella rivoluzione scientifica del Rinascimento. Figure come Copernico, con il suo modello eliocentrico, Keplero, che descrisse le orbite ellittiche dei pianeti, e Galileo Galilei, le cui osservazioni telescopiche fornirono prove cruciali, scardinarono la visione geocentrica millenaria, aprendo la via a una nuova comprensione del nostro posto nell'Universo.

\section{Introduzione Storica e Concettuale}

Comprendere la vastità dell'Universo e la nostra posizione al suo interno divenne la sfida successiva. Tecniche come la \textbf{parallasse} stellare – il cambiamento apparente della posizione di una stella vicina rispetto allo sfondo lontano dovuto al moto orbitale della Terra – permisero le prime misure dirette di distanza, rivelando le immense scale del cosmo. Astronomi come William Herschel tentarono di mappare la struttura tridimensionale della Via Lattea basandosi sulla distribuzione e luminosità delle stelle.

Tuttavia, all'inizio del XX secolo, la natura stessa delle "nebulose" spiraliformi osservate era oggetto di un acceso dibattito (il "Great Debate"): erano parte della nostra Galassia, rendendola di fatto l'intero Universo conosciuto (posizione sostenuta da Heber Curtis), o erano esse stesse "universi-isola", galassie distinte e lontane simili alla nostra (come proposto da Harlow Shapley)?

Questa domanda fondamentale si intrecciava con un antico enigma noto come \textbf{Paradosso di Olbers}:
\begin{observationblock}[Paradosso di Olber]
    Perché il cielo notturno è buio, se l'Universo fosse infinito, eterno e uniformemente pieno di stelle? In tale scenario, ogni linea di vista dovrebbe intercettare la superficie di una stella, rendendo la volta celeste abbagliante.
    $$
    f = \frac{L}{4 \pi d^2} \qquad \text{Flusso da una stella a distanza } d
    $$
    $$
    dJ = \frac{L}{4 \pi d^2} n (4\pi d^2 dr) = L n dr \qquad \text{Contributo da un guscio sferico}
    $$
    Integrando su un universo infinito:
    $$
    J = \int_{r = 0}^{\infty} L n dr = \infty
    $$
    L'oscurità del cielo suggerisce che almeno una delle ipotesi (infinito, eterno, statico, omogeneo) debba essere errata. Le possibili soluzioni includono:
    \begin{enumerate}
        \item L'universo ha un'età finita (la luce dalle stelle più lontane non ci ha ancora raggiunto).
        \item L'universo è in espansione (la luce delle galassie lontane perde energia).
    \end{enumerate}
\end{observationblock}

La chiave per risolvere il dibattito sulla natura delle nebulose e per gettare nuova luce sul paradosso di Olbers giunse dalle misure di velocità e distanza delle galassie. Già si osservava che molte nebulose mostravano uno spostamento sistematico delle loro linee spettrali verso il rosso (\textbf{redshift}), interpretabile come un allontanamento dovuto all'effetto Doppler. Fu Edwin Hubble, nel 1929, a stabilire la relazione cruciale: misurando le distanze di varie galassie (usando stelle Cefeidi come riferimento) e correlandole con i loro redshift, scoprì la \textbf{Legge di Hubble}:
$$
v = H_0 d
$$
dove $v$ è la velocità di recessione, $d$ la distanza e $H_0$ è la \textbf{costante di Hubble}. Questa legge fornì la prova definitiva che le nebulose erano altre galassie esterne alla nostra e, soprattutto, che l'Universo è in \textbf{espansione}.

L'espansione cosmica è il pilastro del modello cosmologico standard attuale, il \textbf{Big Bang}. Estrpolando l'espansione indietro nel tempo, si deduce che l'universo dovesse trovarsi in uno stato iniziale estremamente denso e caldo, dal quale si è espanso e raffreddato fino allo stato attuale. Questo scenario fornisce un quadro coerente per spiegare sia la legge di Hubble sia il paradosso di Olbers.

Per descrivere matematicamente un universo in espansione, la cosmologia moderna adotta il \textbf{Principio Cosmologico}: su scale sufficientemente grandi (tipicamente > 100 Mpc), l'Universo è \textbf{omogeneo} (uguale in ogni punto) e \textbf{isotropo} (uguale in ogni direzione). Sebbene l'universo appaia chiaramente disomogeneo su piccola scala (pianeti, stelle, galassie), queste assunzioni sono ben supportate dalle osservazioni su larga scala, come la distribuzione degli ammassi di galassie e, in modo spettacolare, dalla \textbf{Radiazione Cosmica di Fondo} (CMB). Quest'ultima, un residuo dell'universo primordiale caldo, permea tutto lo spazio con uno spettro di corpo nero quasi perfetto (T ≈ 2.725 K) e mostra un'isotropia eccezionale, confermando le ipotesi del Principio Cosmologico.

Infine, numerose prove osservative indicano che l'universo \textit{evolve} nel tempo. Ad esempio, l'universo primordiale era \textbf{opaco} alla radiazione, mentre oggi è largamente \textbf{trasparente}. Questo cambiamento di stato, insieme all'espansione e al raffreddamento testimoniati dalla CMB e dalla legge di Hubble, sottolinea la natura dinamica del cosmo che andremo a studiare nei prossimi capitoli.

\dots

\section{La metrica di Robertson e Walker}

Usiamo anzitutto il fatto di avere una simmetria sferica dovuta all'isotropia scegliendo un sistema di coordinate sferico, che riflette questa simmetria. Restiamo, per ora, nello spazio euclideo e definiamo:

$$
\begin{cases}
    x = R\sin\theta\cos\varphi\\
    y = R\sin\theta\sin\varphi\\
    z = R\cos\theta
\end{cases}
$$

\dots

$$
ds^2 = c^2 dt^2 + a(t)^2 g_{ij} sx^isx^j
$$

dove la dipendenza dal tempo è tutta nella funzione a(t) detta fattore di scala, ed i gij non dipendono dal tempo. 

Il rapporto $a(t_1)/a(t_0)$ rappresenta l'ingrandimento al tempo $t_1$, rispetto al ltempo $t_0$, di una lunghezza misurata lungo le due superfici $t = t_1$ e $t = t_0$.

\dots

$$
d\ell^2 = R^2d\Omega^2 \quad \Rightarrow \quad d\ell^2 = \underbrace{g(r')}_{r^2} d\Omega^2
$$



\begin{observationblock}
    $g_{r\theta}$ e $g_{r\phi}$ sono nulle 
\end{observationblock}

\dots
$$
t = t_0 \quad \Rightarrow a(t_0) = cost
$$


$$
g_{ij} = 
\begin{pmatrix}
a^2f(r)\\
& a^2r^2\\
&& a^2r^2\sin^2\theta
\end{pmatrix}
\qquad
g^{ij} =
\begin{pmatrix}
    \frac1{a^2f(r)}\\
    & \frac 1{a^2r^2}\\
    && \frac 1{a^2r^2\sin^2\theta}
\end{pmatrix}
\qquad
g = a^6f(r)r^4\sin^2\theta
$$

Poichè siamo in uno spazio omogeneo, abbiamo che in ogni punto in cui $t = t_0$ la curvatura è costante.

Vogliamo imporre la condizione che lo spazio sia omogeneo; questo significa che anche la curvatura dello spazio sarà costante ovunque. Ho una sola funzione da definire, $f (r)$, per cui mi basta una sola condizione, cioè che lo scalare di Ricci della sezione spaziale a tempo cosmico costante, 3R, sia costante.

Ricordiamo che:

$$
\ ^3R=g^{\alpha\beta}\ ^3R_{\alpha\beta}, \quad e \quad
$$

abbiamo messo l'apice $\ ^3$ davanti a $R$ per indicare che ci riferiamo alla parte spaziale, non allo spazio-tempo completo.

Si inizia come sempre dalle connessioni affini (simboli di Kristoffel); ce ne sono 18 indipendenti; di queste solo 7 sono diverse da zero:

$$
\Gamma^1_{11} = \dfrac 12 \dfrac {\dd f}{\dd r}
\quad
\Gamma^1_{22} = - \dfrac rf
\quad
\Gamma^1_{33} = \dfrac {r \sin^2 \theta}{f}
\quad
\Gamma^2_{12} = \Gamma^2_{21} = \dfrac 1r
\quad
\Gamma^3_{13} = \Gamma^3_{31} = \dfrac 1r
\quad
\Gamma^2_{33} = - \sin \theta \cos \theta
\quad
\Gamma^3_{23} = \Gamma^3_{32} = \dfrac{\cos \theta}{\sin \theta}
$$

Poichè  $\ ^3R = g^{\alpha\beta} R_{\alpha\beta} = g^{11} R_{11} + g^{22} R_{22} + g^{33} R_{33}$, abbiamo:

$$
\ ^3R_{11} = \dfrac 1r \cdot \dfrac 1f \dfrac {\dd f}{\dd r}
\qquad
\ ^3R_{22} = 1 - \dfrac 1f + \dfrac 12 \dfrac r{f^2} \dfrac {\dd f}{\dd r}
\qquad
\ ^3R_{33} = \sin^2 \theta \cdot \ ^3R_{22}
$$

da cui, imponendo $\ ^3R = costante = K$:

$$
\ ^3R = g^{11}R_{11} + g^{22} R_{22} + g^{33} R_{33}
$$

$$
\ ^3R = K = \dfrac 2{a^2r^2} \left[
    1 - \dfrac 1f + \dfrac r{f^2} \dfrac{\dd f}{\dd r}
\right]
= \dfrac 2{a^2r^2}
\left[
    1 - \dfrac {\dd}{\dd r}\left(
        \dfrac rf
    \right)
\right]
=
\dfrac 2{a^2r^2} \dfrac \dd {\dd r}
\left[
    r\left(
        1 - \dfrac 1f
    \right)
\right]
$$

da cui:

$$
\dd \left[
    r \left(
        1 - \dfrac 1f
    \right)
\right]
=
\dfrac{Ka^2r^2}2 \dd r
$$

Integrando otteniamo:

$$
r \left(
1 - \dfrac 1f
\right)
=
\dfrac{Ka^2r^3}6 + A
\qquad \Rightarrow \qquad
f(r) = \dfrac1{1+\frac{Ka^2r^2}6 - \frac Ar}
$$

Ma, se $r \to 0$, la metrica sarà quella euclidea, per cui $f(r) \equiv 1$; ne segue che $A = 0$ eccetto

$$
\dd s^2 = c^2 \dd t^2 - a^2(t) \left[ \dfrac{\dd r^2}{1-\frac{Ka^2(t)r^2}6} + r^2 \dd \Omega \right]
$$

Osserviamo però che, come avevamo detto, la dipendenza dal tempo della parte spaziale si esplica attraverso la $a^2(t)$ davanti alla parentesi quadra, ed il termine all'interno di questa non dipende dal tempo. Questo significa che $Ka^2$ è funzione di $r$. Sarà allora $K = K(t)$.

Definiamo un cambiamento di scala in $r$ tale che $\dfrac{Ka2r2}6 ≡ k\tilde r^2$, dove $k = 0$ se $K = 0$, altrmenti $k$ ha lo stesso segno di $K$, ma modulo 1. 

$$
k = \begin{cases}
    0 & if\ K = 0\\
    sign(K) & if\ K \not = 0
\end{cases}
$$

Abbiamo quindi:

$$
r^2 = \dfrac {6k}{Ka^2} \tilde r^2
\qquad \Rightarrow \qquad
r = \tilde r \sqrt{\dfrac {6k}{Ka^2}} \quad \dd r = \sqrt{\dfrac {6k}{Ka^2}} \dd\tilde r
$$

e quindi:

$$
\dd t^2 = \tilde a^2(t) \left[
    \dfrac{6k}{K(t)a^2(t)} \cdot \dfrac{\dd \tilde r^2}{1 - k \tilde r^2} + \dfrac{6k}{K(t)a^2(t)} \tilde r^2 \dd \Omega^2
    \right]
    =
    \dfrac{6k}{K(t)} \left[
    \dfrac{\dd \tilde r^2}{1 - k \tilde r^2} + \tilde r^2 \dd \Omega^2
    \right]
$$

Se vogliamo che $dt^2 = \tilde a^2(t)[...]$, definiamo $\dfrac{6k}{K(t)} = \tilde a^2(t)$, da cui:

$$
dt^2 = \tilde a^2(t) \left[
    \dfrac{\dd \tilde r^2}{1 - k \tilde r^2} + \tilde r^2 \dd \Omega^2
    \right]
$$

in cui $K(t) = \ ^3R(t) \equiv \dfrac{6k}{\tilde a^2(t)}$.

\section{Topologia dell'Universo}

Vediamo ora in dettaglio le proprietà topologiche dei modelli cosmologici corrispondenti ai tre casi k = 0, +1, -1.

\begin{enumerate}
    \item \textbf{Caso $k = 0$} (universo piatto): 
    
    In questo caso l'universo è omogeneo e isotropo, ma non ha curvatura. La geometria è euclidea e le linee di universo sono rette. La topologia è quella di uno spazio euclideo tridimensionale, quindi l'universo è infinito e non ha bordi.
    $$\mathbb E^3 \to "spazio\ piatto" \quad 0 < r < \infty \quad \to \quad "spazio\ infinito"$$
    \item \textbf{Caso $k = +1$} (universo chiuso): 

    Per $\dd \theta = \dd \varphi = 0$ abbiamo $\dd \ell = a(t) \dfrac{\dd r}{\sqrt{1-r^2}}$ per cui $|r|<1$ e la metrica diverge se $r \to 1$.
    Possiamo eliminare tale divergenza scegliendo una nuova coordinata $\chi$ al posto di $r$, tale che $r = \sin \chi$ così che:

    $$
    \dd r = \cos \chi \dd \chi = \sqrt{1-\sin^2\chi}\dd\chi = \sqrt{1-r^2}\dd\chi
    $$
    $$
    \dd \ell^2 = a^2(t) \left[
        \dfrac{\cancel{(1 - r^2)}\dd \chi^2}{\cancel{1 - r^2}} + \sin^2 \chi \dd \Omega^2
    \right] =
    a^2(t) \left[
        \dd \chi^2 + \sin^2 \chi \dd \Omega^2
        \right]
    $$

    Con $0\le\chi\le\pi$, $0\le\theta\le\pi$, $0\le\varphi\le2\pi$.
    
    Confonrtiamo tale topologia con la sfera in 2-D in $\mathbb E^3$. Sarà $\chi = s/a$; $s = a\chi$, $a\sin\chi = x$:
    $$
    \dd t^2 = a^2 \dd \chi^2 + x^2 \dd \varphi^2 = a(\dd\chi^2 + \sin^2\chi \dd \varphi^2)
    $$

    Inoltre, detto $u = x/a, \ u = \sin \chi,\ \dd u ? \cos \chi \dd \chi = \sqrt{1-\sin^2\chi}\dd\chi = \sqrt{1-u^2}\dd \chi$ e la metrica diventa:

    $$
    \dd t^2 = a^2 \left[
        \dfrac{\dd u^2}{1-u^2} + u^2 \dd \varphi^2
    \right]
    $$

    Vedo che la coordinata r in Robertson-Walker con $k = +1$ corrisponde a $x/a$ nel caso 2-D della sfera; $\chi$ varia tra $0$ e $\pi$.

    \begin{figure}[H]
        \centering
        \includegraphics[width=0.3\textwidth]{assets/caso_k_1.png}
        \caption{Universo chiuso}
        \label{fig:universo_chiuso}
    \end{figure}

    Consideriamo ora una ipersfera: una sfera 3-D in $\mathbb E^4$; L'equazione della sfera, estendendo il teorema di pitagora e ricordando che la quarta dimensione sarà in qualche modo perpendicolare alle altre tre, sarà:

    $$
    x^2 + y^2 + z^2 + u^2 = a^2 \qquad \quad a = cost = \text{raggio della sfera}
    $$

    Sia $r^2 = x^2 + y^2 + z^2$ (sfera 2-D per cui $u = cost$) così che $a^2 = r^2 + u^2 = cost$. Differenziando otteniamo:

    $$
    2r\dd r + 2 u \dd u = 0 \qquad \Rightarrow \qquad r\dd r = -u \dd u
    $$
    $$
    \dd u^2 = \dfrac{r2\dd r^2}{u^2} = \dfrac{r^2 \dd r^2}{a^2 - r^2} 
    $$
 
    Questo per garantire che lo spostamento $\dd u$ sia sulla ipersfera in $\mathbb E^4$; Il $\dd\ell^2$ in $\mathbb E^4$ sarà:

    $$
    \dd\ell^2 = \dd x^2 + \dd y^2 + \dd z^2 + \dd u^2
    $$

    In coordinate polari abbiamo:

    $$
    \dd\ell^2 = \dd r^2 + r^2 \dd\Omega^2 + \dd u^2 = \dd r^2 + r^2 \dd\Omega^2 + \dfrac{r^2 \dd r^2}{a^2 - r^2} = \dd r^2 \left[ 1 + \dfrac{r^2}{a^2 - r^2} \right] + r^2 \dd\Omega^2 = \dfrac{\dd r^2}{1-r^2/a^2} + r^2 \dd\Omega^2
    $$

    Posto $\tilde r = r/a$ abbiamo che ($\dd r = a \dd \tilde r$):

\end{enumerate}