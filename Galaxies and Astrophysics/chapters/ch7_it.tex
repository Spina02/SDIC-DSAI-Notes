\chapter{Lecture 01/04/2025}

\dots

In un \bfit{sistema inerziale localmente in quiete} (\textbf{SILQ}) rispetto al fluido, nel quale quindi $u^a = (1, 0, 0, 0)$, $T^{\alpha\beta}$ ha la forma particolarmente semplice:

$$
T^{\alpha\beta}_{SILQ} = \rho_0 c^2 u^\alpha u^\beta 
=
\begin{pmatrix}
    \rho_0 c^2 & 0 & 0 & 0 \\
    0 & 0 & 0 & 0 \\
    0 & 0 & 0 & 0 \\
    0 & 0 & 0 & 0
\end{pmatrix}
$$

Veniamo adesso a considerare il caso in cui le particelle interagiscono nel modo più semplice, cioè attraverso urti dovuti all'agitazione termica: è presente una pressione del fluido. Assumiamo che non vi sia trasporto di energia per conduzione o irraggiamento e non vi sia viscosità. Il fluido così definito è detto perfetto

\dots

$$
T^{\alpha\beta}_{SILQ} = \rho_0 c^2 u^\alpha u^\beta 
=
\rho_0 c^2 \begin{pmatrix}
    1 & \langle v_x \rangle / c & \langle v_y \rangle / c & \langle v_z \rangle / c \\
    & \langle v_x^2 \rangle / c^2 & \langle v_x v_y \rangle / c^2 & \langle v_x v_z \rangle / c^2 \\ 
    & & \langle v_y^2 \rangle / c^2 & \langle v_y v_z \rangle / c^2 \\
    & & & \langle v_z^2 \rangle / c^2
\end{pmatrix}
= 
\rho_0 c^2
\begin{pmatrix}
    1 & 0 & 0 & 0 \\
    0 & p & 0 & 0 \\
    0 & 0 & p & 0 \\
    0 & 0 & 0 & p
\end{pmatrix}
$$

\dots

$$
T^{\alpha\beta}_{SILQ} = (p + \rho c^2) u^\alpha u^\beta - p \eta^{\alpha\beta}
$$

\dots

\begin{exampleblock}[Conservazione dell'entropia per particella]

Vediamo ora di ricavare, come detto poco sopra, una relazione scalare dalla $\partial _\beta  T^{\alpha \beta }$; per fare questo la moltiplicheremo per $u_\alpha $.

Partiamo dal fatto che, come abbiamo già visto, $u^\alpha u_\alpha  = 1$. Sarà quindi:

$$
\begin{array}{rcl}
\dfrac{\partial}{\partial x^\beta} (u^\alpha u_\alpha) 
& = & 
u^\alpha \dfrac{\partial u_\alpha}{\partial x^\beta} + u_\alpha \dfrac{\partial u^\alpha}{\partial x^\beta}\\
& = &
\eta^{\alpha\gamma} u_\gamma \dfrac{\partial u_\alpha}{\partial x^\beta} + u_\alpha \dfrac {\partial u^\alpha}{\partial x^\beta}\\
& = &
u_\gamma \dfrac{\partial u^\gamma}{\partial x^\beta} + u_\alpha \dfrac{\partial u^\alpha}{\partial x^\beta}\\
& = &
2 u_\alpha \dfrac{\partial u^\alpha}{\partial x^\beta} = 0
\end{array}
$$

da cui $\dfrac {\partial u^\alpha}{\partial x^\beta} = 0$ (abbiamo sfruttato il fatto che $\alpha$ e $\gamma$ sono indici muti). Se riprendiamo l'equazione che esprime la divergenza di $T^{\alpha \beta}$ e la moltiplichiamo per $u_\alpha$ otteniamo:

$$
u_\alpha \dfrac{\partial}{\partial x^\beta} \big[(p + \rho c^2)u\alpha  u^\beta \big] - \dfrac {\partial p} {\partial x^\beta}  \eta^{\alpha \beta} u_\alpha   = 0
$$

e, sviluppando la derivata del primo termine, abbiamo:

$$
u_\alpha
\left\{
    u^\alpha \dfrac {\partial} {\partial x^\beta} 
    [(p + \rho c^2)u^\beta  ] + (p + \rho c^2)u^\beta \dfrac {\partial u^\alpha} {\partial x^\beta} 
\right\}
- {\partial p} {\partial x^\beta}  u^\beta  = 0
$$

Se ricordiamo che $u^\alpha u_\alpha  = 1$ e che $\dfrac {\partial u^\alpha}{\partial x^\beta}  = 0$ possiamo scrivere:

$$
\dfrac \partial {\partial x^\beta} 
[(p + \rho c^2)u^\beta  ] - u^\beta \dfrac {\partial p}
{\partial x^\beta}  = 0
$$
$$
(p + \rho c^2) \dfrac {\partial u^\beta}{\partial x^\beta}  + u^\beta  \dfrac \partial {\partial x^\beta}  (p + \rho c^2) - u^\beta \dfrac {\partial p} {\partial x^\beta}  = 0
$$

Dalla conservazione del numero di particelle abbiamo
$$
\dfrac {\partial  (n u^\beta)}{\partial x^\beta}  = 0 
\quad \Rightarrow \quad
n \dfrac {\partial u^\beta}{\partial x^\beta}  + u^\beta \dfrac {\partial n} {\partial x^\beta}  = 0
\qquad \Rightarrow \qquad
\dfrac {\partial u^\beta }{\partial x^\beta}  = - \dfrac {u^\beta}n
\dfrac {\partial n}{\partial x^\beta}
$$

Sostituendo quest'ultimo risultato nella precedente relazione e raccogliendo $u^\beta$  abbiamo:
$$
u^\beta \left\{ \dfrac {\partial (p + \rho c^2)}
{\partial x^\beta} - \dfrac {p + \rho c^2}n
\dfrac {\partial n}{\partial x^\beta} - \dfrac {\partial p}{\partial x^\beta} 
\right\}
= 0
$$

\dots

Ricordiamo ora il primo principio della termodinamica: $dU = dQ + dL$; se introduciamo l'entropia $S$ possiamo scrivere: $TdS = dU + pdV$, dove l'energia interna è $U = \rho c^2$. Se lo riscrivo riferendomi ad una particella, avrò $Td\sigma = d \left(\dfrac{\rho c^2}n \right) + pd\left(
dfrac 1n \right)$, con $\sigma$ entropia per particella. Sviluppando i differenziali si ha:

$$
Td\sigma = \dfrac {\partial \sigma}{\partial x^\beta}dx^\beta =
\dfrac {\partial}{\partial x^\beta} \left(\dfrac{\rho c^2}n \right) dx^\beta + p \dfrac {\partial}{\partial x^\beta} \left( \dfrac 1n \right) dx^\beta \qquad / \cdot \dfrac 1{ds}
$$

Se ricordiamo che $\dfrac {d x^\beta}{ds} \equiv u^\beta$ e confrontiamo questa relazione con la precedente, otteniamo:

$$
u^\beta \dfrac {\partial \sigma}{\partial x^\beta} = 0
$$

che, sviluppando, diventa:

$$
\gamma \dfrac 1c \dfrac {\partial \sigma}{\partial t} + \gamma \dfrac {v_x}c \dfrac {\partial \sigma}{\partial x} + \gamma \dfrac {v_y}c \dfrac {\partial \sigma}{\partial y} + \gamma \dfrac {v_z}c \dfrac {\partial \sigma}{\partial z} = 0
$$
$$
\dfrac{\partial\sigma}{\partial t} + (\bar v \cdot \nabla) \sigma = 0
\qquad \Leftrightarrow \qquad
\underbrace{\dfrac{d\sigma}{dt} = 0}_{\text{conservazione dell'entropia per particella}}
$$

\end{exampleblock}

Si ha come risultato che, nel sistema in cui il fluido è in quiete, l'entropia per particella (o, se preferiamo,
l'entropia per un certo numero $N$ di particelle contenute in un volume cubico $V$ di spigolo $L$, che può variare
mantenendo però sempre al suo interno lo stesso numero di particelle) è costante. Questo è legato al fatto che,
nel fluido ideale, non c'è scambio di energia per conduzione (o irraggiamento), né vi è dissipazione. Dal $1^{\circ}$
principio della termodinamica, nel sistema che segue il fluido, $dQ = dU + pdV$ e $U = \rho c^2 \cdot V$, per cui:
$$
dQ = \rho c^2 dV + Vd(\rho c^2) + pdV = (p + \rho c^2)dV + Vd(\rho c^2) = TdS
$$

Poichè $dQ = 0 \rightarrow dS = 0$

Se scriviamo $p = w \rho c^2$ (con $w$ costante, anche se, più in generale, potrà dipendere dalla temperatura $w = w(T)$),

$$
(1 + w) \rho c^2 dV = - V d(\rho c^2)
$$

e se $w = cost$, ho $d \rho / \rho = - (1 + w) dV / V$, cioè $\rho V^{1 + w} = cost$.

Incontreremo tre casi interessanti in cosmologia:

\begin{enumerate}
    \item Per un gas non-relativistico $p \ll \rho_0 c^2$ ($\rho \approx \rho_0$) per cui $w \approx 0$ e quindi $\rho_0 V \approx \text{cost}$. Detto $L$ lo spigolo di un volume cubico $V = L^3$, abbiamo $\rho \propto 1/L^3$

    \item Per un gas di fotoni $\rho_{\text{rad}} \propto aT^4$ e $p = \frac{1}{3} \rho c^2$; $w = \frac{1}{3}$:
    
    $$T^4 V^{4/3} = \text{cost}
    \quad\Rightarrow\quad
    T \cdot V^{1/3} = \text{cost}
    \quad\Rightarrow\quad
    V \propto L^3
    \quad\Rightarrow\quad
    T \propto \frac{1}{L}$$

    $$\rho_{\text{rad}}V^{4/3} = \text{cost}
    \quad\Rightarrow\quad
    V \propto L^3
    \quad\Rightarrow\quad
    V^{4/3} \approx L^4
    \quad\Rightarrow\quad
    \rho_{\text{rad}} \approx \frac{1}{L^4}$$

    \item Se $p = - \rho c^2 \quad (w = -1) \quad \leftarrow \quad \rho V^0 = cost$, cioè $\rho$ non dipende da $V$ e da $L$ e rimane costante se $V$ varia.
\end{enumerate}

Possiamo riscrivere il primo principio in un'altra forma utile, ponendo $V \propto L^3$:

$$
\left( \rho + \dfrac {p}{c^2} \right) dC + V d\rho = 0 \qquad \rightarrow \qquad \left(\rho + \dfrac {p}{c^2}\right) \cdot 3L^2 dL + L^3 d\rho = 0
$$

da cui:

$$
3 \left(
    \rho + \dfrac {p}{c^2}
\right) \dfrac {dL}{L} + d\rho = 0
$$

e, tenendo conto di una dipendenza di $L$ dal tempo:

$$
3 \left(
    \rho + \dfrac {p}{c^2}
\right) \dfrac {dL}{dt} + \dfrac {d\rho}{dt} = 0
$$

Abbiamo scritto $\partial_\alpha T^{\beta \alpha} = 0$ nello spazio di Minkowski; se però $\Gamma^\alpha_{\beta\gamma}$ non sono tutti nulli, ed in generale sarà così, al posto della derivata parziale semplice si deve utilizzare la derivata covariante:
$$
T^{\alpha\beta}_{;\beta} = 0
$$
che esprime le leggi di conservazione in un sistema di riferimento generico.