\chapter{Distanza Comovente Radiale}

Una grandezza fondamentale, come vedremo, per il confronto tra modelli cosmologici ed osservazioni è la coordinata co-movente radiale $r$. In particolare, è essenziale la sua dipendenza dal redshift $z$:
$$
\dfrac{a_0 \dd r}{\sqrt{1 - k r^2}} = -c \dd t = - c \dfrac{\dd a}{\dot a} = -c{\dd a}{aH}
$$

$$
\begin{cases}
a = \dfrac{a_0}{1+z}
\dd a = - \dfrac{a_0}{(1+z)^2}\dd z
\end{cases}

$$

dunque:

$$
\dfrac{a_0 \dd r}{\sqrt{1 - k r^2}} =  \dfrac{c}{H(z)} \dd z
$$
con $H(z) = H_0 E(z)$.

$$
\int_0^t \dfrac{c \dd t'}{a(t')} = \int_0^r \dfrac{\dd r'}{\sqrt{1 - k r'^2}}
= 
f_k(r) = \text{Distanza comovente}
$$

$$
f_k(r) = \int_0^r \dfrac{\dd r'}{\sqrt{1 - k r'^2}}
=
\dfrac c{a_oH_0} \int_0^z \dfrac{\dd z'}{E(z')}
= 
\begin{cases}
    \arcsin(r) & \text{se } k = 1 \\
    r & \text{se } k = 0 \\
    \text{arcsinh}(r) & \text{se } k = -1
\end{cases}
$$

Se ricordiamo che:

$$
a_0 = \dfrac{c}{H_0 \sqrt{|\Omega_0 - 1|}}
$$

e, definendo:
$$
sinn(x) = \begin{cases}
    \sin(x) & \text{se } k = 1 \ \ (\Omega_0 > 1)\\
    \sinh(x) & \text{se } k = -1 \ \ (\Omega_0 < 1)\\
\end{cases}
$$

Otteniamo, Per $k = \pm 1$:

$$
a_0 r(z) = \dfrac c{H_0 \sqrt{|\Omega_0 - 1|}} sinn\left[
    \sqrt{|\Omega_0 - 1|} \int_0^z \dfrac{\dd z'}{E(z')}
\right]
$$

mentre, per $k = 0$ $(\Omega_0 = 1)$:
$$
a_0 r(z) = \dfrac c{H_0} \int_0^z \dfrac{\dd z'}{E(z')}
$$

\dots

\section{Test osservativi dei modelli cosmologici}

Abbiamo definito $q_0 = - \dfrac{\ddot a(t_0) a_0}{\dot a(t_0)^2}$ e, usando $H_0 = \dfrac{\dot a(t_0)}{a(t_0)}$, $a(t)$ si può riscrivere pertanto come:

$$
a(t) = a_0 \left[ 1 + H_0 (t - t_0) - \dfrac 12 q_0 H_0^2 (t - t_0)^2 + \dots \right]
$$

Usando il redshift $z$ tale che $1+z = \dfrac{a_0}{a(t)}$, otteniamo:

$$
\begin{array}{rcl}
1+z & = & \left[ 1 + H_0 (t - t_0) - \dfrac 12 q_0 H_0^2 (t - t_0)^2 + \dots \right]^{-1}\\
& \simeq & 1 - H_0 (t - t_0) + \dfrac 12 q_0 H_0^2 (t - t_0)^2 + \dots + \dfrac{-1 (-1 -1)}2 \left[H_0^2(t - t_0)^2 + \dots\right]
\end{array}
$$

da cui:

$$
z \simeq H_0(t_0 - t) + (1 + \dfrac {q_0}2) H_0^2 (t_0 - t)^2 + \dots
$$

\missing{calcoli}

\dots

$$
r \simeq \dfrac c{a_0} \left[ (t_0 - t) + \dfrac 12 H_0(t_0 - t)^2 + \dots \right]
$$

\subsubsection{Distanza di luminosità}

La distanza propria non è calcolabile, ma si può definire la \bfit{distanza di luminosità}.

Possiamo definire l'energia irradiata dalla una sorgente di Luminosità $L$, per unità di tempo $\dd \lambda$, in funzione della lunghezza d'onda $\lambda$ come:

$$
\dd L = L \cdot \rho(\lambda) \dd \lambda
$$

\dots

$$
\rho(\nu)  = \lambda^2 \rho(\lambda)
$$

\dots

possiamo scrivere che:

$$
\dd s^2 = - r^2 a_0^2 \underbrace{\dd \Omega^2}_{= \dd \theta^2 + \sin^2 \theta \dd \phi^2}
$$

$$
\dd A = r^2a_0^2 \sin \theta \dd \theta \dd \phi
$$

Vogliamo osservare quanta energia arruva nell'intervallo di tempo, perpendicolarmente alla linea di vista, sull'unità di superficie e nella banda $a_0, a_0 + \dd a_0$. Questo è il flusso $F(\lambda_0)$ per l'ampiezza $\dd \lambda_0$ della banda:

\dots

$$
E_\lambda^{em} = \dfrac{hc}{\lambda_{em}} = \dfrac{hc}{\lambda_0} (1 + z)
$$

$$
\delta N = \dfrac{\dd E^{em}_{\lambda_{em}}}{\dd \lambda_{em}} \dd \lambda_{em}
=
L q\left(\dfrac{\lambda_{em}}{1 + z}\right) \dd t_1 \dfrac{\dd \lambda_{0}}{hc(1+z)}
$$


Calcoliamo il numero di fotoni $\gamma$ ricevuti con $\lambda_0$ per unità di tempo $\dd t_0$ e area $4 \pi r_1^2a_0^2$:

$$
\dfrac{L \lmabda_0}{ch} \dfrac 1{(1+z)^2} \phi \left(\dfrac{\lambda_0}{1+z}\right) \dd \lambda_0 \dfrac{\dd t_1}{\dd t_0} \dfrac 1{4 \pi r_1^2 a_0^2}
$$

Il flusso è dato dal numero di fotoni per l'energia di ogni fotone:

$$
F(\lambda_0) = \delta N \cdot E_\gamma = \dfrac{L}{1+z} \phi \left( \dfrac{\lambda_0}{1+z}\right) \dfrac 1{4 \pi r_1^2 a_0^2}
$$

possiamo riscriverlo in funzione della frequenza

$$
\tilde F(\nu_0) = \dfrac{\lambda_0^2}{c} F(\lambda_0) = (1+z^2) \dfrac{\lambda_{em}^2}{c} F(\lambda_0) = \dfrac{L}{1+z} \dfrac{\dfrac{\lambda_{em}^2}{c} \phi (\lambda_{em})}{4 \pi r_1^2 a_0^2}
$$
$$
= \dfrac{L}{1+z} \dfrac{\tilde \phi(\nu_0(1+z))}{4 \pi r_1^2 a_0^2}
$$

dunque, il \bfit{flusso bolometrico} è:

$$
\tilde F_{bol} = \dfrac{L}{4 \pi r_1^2 a_0^2}
$$

\dots

$$
F_{bol} = \int_0^\infty F(\lambda_0) \dd \lambda_0 = \dfrac{L}{(1+z)^2 4 \pi r^2a^2} \int_0^\infty \phi\left(\dfrac{\lambda_0}{1+z}\right) \dfrac{\dd \lambda_0}{1 + z} (1+z) = \dfrac{L}{(1+z)^2 4 \pi r^2a^2}
$$

Nello spazio euclideo:

$$
F = \dfrac{L}{4 \pi d^2} \quad \Rightarrow \quad d = \left(\dfrac{L}{4 \pi F}\right)^{1/2}
$$

Distanza di luminosità:

$$
d_L = \left(\dfrac{L}{4 \pi F}\right)^{1/2} = r a_0 (1+z) = \dfrac {r a_0}{a}
$$

