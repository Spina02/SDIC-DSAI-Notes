
\chapter{PyTorch}

PyTorch (PT) is a Python (and C++) library for Machine Learning (ML) particularly suited for Neural Networks and their applications.

Its great selection of built-in modules, models, functions, CUDA capability, tensor arithmetic support and automatic differentiation functionality make it one of the most used scientific libraries for Deep Learning.

Note: for this series of labs, we advise to install Python >= 3.7



We advise to install PyTorch following the directions given in its \href{https://pytorch.org/get-started/locally/}{home page}. Just typing \texttt{pip install torch} mat not be the correct action as you have to take into account the compatibility with \texttt{cuda}. If you have \texttt{cuda} installed, you can find your version by typing \texttt{nvcc --version} in your terminal (Linux/iOS).

If you're using Windows, we first suggest to install Anaconda and then install PyTorch from the \texttt{anaconda prompt} software via \texttt{conda} (preferably) or \texttt{pip}.

If you're using Google Colab, all the libraries needed to follow this lecture should be pre-installed there.

\begin{observationblock}[For Colab users]
    Google Colab is a handy tool that we suggest you use for this course---especially if your laptop does not support CUDA or has limited hardware capabilities. Anyway, note that we'll try to avoid GPU code as much as possible.

Essentially, Colab renders available to you a virtual machine with a limited hardware capability and disk where you can execute your code inside a given time window. You can even ask for a GPU (if you use it too much you'll need to start waiting a lot before it's available though).