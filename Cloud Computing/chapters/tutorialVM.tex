
\chapter{Tutorial: Virtual Machines}

This tutorial is made with VirtualBox in mind, but the general concepts should be applicable to other virtualization software as well.

Important things to keep in mind:
\begin{itemize}
    \item This tutorial is made with MacOS 
    \item The VMs are running Ubuntu 20.04
    \item Only the Master node is connected to the internet (NAT)
    \item Working nodes are connected to the master node via an Internal Network
\end{itemize}

\section{Master Node}

\subsection{Master Node as a Gateway}

\begin{itemize}
    \item \textbf{Enable IP forwarding on the Master Node}
    \\Run the following command:
    \begin{codeblock}[language=bash]
echo 1 | sudo tee /proc/sys/net/ipv4/ip\_forward
    \end{codeblock}
    To make this change permanent, edit the file \texttt{/etc/sysctl.conf} and uncomment the line \texttt{net.ipv4.ip\_forward=1}.

    Apply the changes with the following command:
    \begin{codeblock}[language=bash]
sudo sysctl -p
    \end{codeblock}
    
    \item \textbf{Enable NAT on the Master Node}
    Assuming your master node has two interfaces:
    \begin{itemize}
        \item \texttt{enp0s8} connected to the internet (NAT)
        \item \texttt{enp0s9} connected to the internal network
    \end{itemize}
    

    Run the following command:
    \begin{codeblock}[language=bash]
sudo iptables -t nat -A POSTROUTING -o enp0s8 -j MASQUERADE
sudo iptables -A FORWARD -i eth1 -o eth0 -j ACCEPT
sudo iptables -A FORWARD -i eth0 -o eth1 -m state --state RELATED,ESTABLISHED -j ACCEPT
    \end{codeblock}
    Replace \texttt{enp0s8} with the name of the interface connected to the internet.
    
    To persist the rules:
    \begin{codeblock}[language=bash]
sudo apt-get install iptables-persistent # For Debian/Ubuntu 
sudo netfilter-persistent save
sudo netfilter-persistent reload
    \end{codeblock}
    \item \textbf{Configure the Woring Node to Use the Master Node as a Gateway}
    On the working node, set the master node as the default gateway:
    \begin{codeblock}[language=bash]
sudo ip route add default via <MASTER\_NODE\_IP>
    \end{codeblock}
    Replace \texttt{<MASTER\_NODE\_IP>} with the IP address of the master node.
    \item \textbf{Set up DNS on the Working Node}
    Edit the file \texttt{/etc/resolv.conf} and add the following line:
    \begin{codeblock}[language=bash]
sudo vim /etc/resolv.conf
echo "nameserver <DNS\_SERVER\_IP>" | sudo tee /etc/resolv.conf
    \end{codeblock}
    Replace \texttt{<DNS\_SERVER\_IP>} with the IP address of the DNS server (ex. 8.8.8.8 for Google's DNS).

    \item \textbf{Test the Connection}
    Run the following command on the working node:
    \begin{codeblock}[language=bash]
ping google.com
ping 8.8.8.8
    \end{codeblock}
    If the connection is successful, the working node is now connected to the internet via the master node.
\end{itemize}

\begin{warningblock}
    It could happen that after restarting the woring node cannot more connect to the internet. At this point, first check if you can ping the Master Node:
    \begin{codeblock}[language=bash]
ping <MASTER\_NODE\_IP>
    \end{codeblock}
    If the connection is successful, check the default route pointing to the Master Node:
    \begin{codeblock}[language=bash]
ip route show
    \end{codeblock}
    If the default route is not set, add it again:
    \begin{codeblock}[language=bash]
sudo ip route add default via <MASTER\_NODE\_IP>
    \end{codeblock}
    Now it should work.
\end{warningblock}

\subsection{DNS Configuration}

The master node may not have a working nameserver. Check your current DNS settings:
\begin{codeblock}[language=bash]
cat /etc/resolv.conf
\end{codeblock}

If it is empty or does not contain a valid nameserver, add one:
\begin{codeblock}[language=bash]
echo "nameserver 8.8.8.8" | sudo tee /etc/resolv.conf
\end{codeblock}

For a permanent solution, edit the file \texttt{/etc/netplan/01-netcfg.yaml} and add the following lines:
\begin{codeblock}[language=bash]
network:
    version: 2
    ethernets:
        enp0s3:
            dhcp4: true
            nameservers:
                addresses: [8.8.8.8, 8.8.4.4]
\end{codeblock}

\begin{warningblock}
There could appear a warning saying that the Netplan configuration file has permissions that are too open, making it accessible to unauthorized users. 

Run the following command to fix the permissions:
\begin{codeblock}[language=bash]
sudo chmod 600 /etc/netplan/01-netcfg.yaml
\end{codeblock}

Apply the changes with the following command:
\begin{codeblock}[language=bash]
sudo netplan apply
\end{codeblock}
\end{warningblock}

Apply the changes with the following command:
\begin{codeblock}[language=bash]
sudo netplan apply
\end{codeblock}

Restart the network service:
\begin{codeblock}[language=bash]
sudo systemctl restart systemd-networkd
\end{codeblock}

Now the master node should have a working nameserver, allowing it to resolve domain names. Try pinging a domain name to test the connection:
\begin{codeblock}[language=bash]
ping google.com
\end{codeblock}

\section{DHCP Server Configuration}

Here we will set up a DHCP server to assign IPs dynamically to the working nodes and configure a DNS server for name resolution.

\begin{enumerate}
    \item Install and Configure a DHCP Server on the Master Node 
    \begin{codeblock}[language=bash]
sudo apt-get install isc-dhcp-server
    \end{codeblock}
    Edit the file \texttt{/etc/dhcp/dhcpd.conf} and add the following lines:
    \begin{codeblock}[language=bash]
subnet 192.168.0.0 netmask 255.255.255.0 {
    range 192.168.0.100 192.168.0.200;
    option routers 192.168..1; # Master Node as the gateway 
    option domain-name-servers 192.168.0.1, 8.8.8.8;
    option domain-name "internal.local";
    default-lease-time 600;
    max-lease-time 7200;

}
    \end{codeblock}
    Save and exit.

    Set the internal network interface as the DHCP server interface by editing the file \texttt{/etc/default/isc-dhcp-server}:
    \begin{codeblock}[language=bash]
INTERFACESv4="enp0s9"
    \end{codeblock}
    Replace \texttt{enp0s9} with the name of the internal network interface.

    Restart the DHCP server:
    \begin{codeblock}[language=bash]
sudo systemctl restart isc-dhcp-server
sudo systemctl enable isc-dhcp-server
    \end{codeblock}

    \item Configure DNS on the Master Node 
    
    Install the DNS server:
    \begin{codeblock}[language=bash]
sudo apt install dnsmasq
    \end{codeblock}

    Edit the file \texttt{/etc/dnsmasq.conf} and add the following lines:
    \begin{codeblock}[language=bash]
interface=enp0s9
dhcp-range=192.168.0.100,192.168.0.200,12h
server=8.8.8.8
domain=internal.local
    \end{codeblock}
    Replace \texttt{enp0s9} with the name of the internal network interface.

    Restart the DNS server:
    \begin{codeblock}[language=bash]
sudo systemctl restart dnsmasq
sudo systemctl enable dnsmasq
    \end{codeblock}

    \item Configure the Working Nodes to Use the DHCP Server
    
    Edit the file \texttt{/etc/netplan/01-netcfg.yaml} and add the following lines:
    \begin{codeblock}[language=bash]
network:
    version: 2
    ethernets:
        enp0s8:
            dhcp4: true
    \end{codeblock}
    Apply the changes with the following command:
    \begin{codeblock}[language=bash]
\end{enumerate}