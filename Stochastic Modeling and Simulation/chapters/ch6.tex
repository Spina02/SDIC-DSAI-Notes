\newpage

\chapter{Lecture 21/03/2025}

\section{Evolution of the Probability Density Function: The Fokker-Planck Equation}

Last time we examined Malthusian processes, where the dynamics of the state variable $x(t)$ are governed by the stochastic differential equation (SDE)
$$
dx = f(x)\,dt + g(x)\,dW.
$$
Over an infinitesimal time increment $dt$, the update can be written as
$$
x(t+dt) = x(t) + \underbrace{f(x)\,dt}_{O(dt)} + \underbrace{g(x)\,dW}_{O(\sqrt{dt})}.
$$

Since the increment $dW$ is of order $\sqrt{dt}$ and the change in $x$ is infinitesimal, the evolution of the probability density function (PDF) $\rho(x,t)$ for $x(t)$,
$$
\Pr\Bigl[x(t) \in [\hat{x},\,\hat{x}+d\hat{x}]\Bigr] = \rho(\hat{x},t)d\hat{x},
$$
depends only on the local properties of $\rho(x,t)$ (specifically its first and second derivatives). In other words, the future evolution of $\rho(x,t)$ is determined by its current state and the local changes, which leads to a Partial Differential Equation (PDE) for $\rho(x,t)$.

For instance, when the process $x(t)$ is a pure diffusion process—as is the case for a Wiener process—the PDF is given by
$$
\rho(w,t) = A\,\exp\Bigl(-\frac{w^2}{2t}\Bigr),
$$
where the normalization constant $A$ ensures that the total probability is unity. In this case, one can derive that
$$
\frac{\partial \rho}{\partial t} = \frac{\partial^2 \rho}{\partial w^2}.
$$

This equation is a particular instance of the more general \textbf{\textit{Fokker-Planck equation}}, which describes how the probability density function of a stochastic process evolves over time. In the general case, the Fokker-Planck equation incorporates the contributions from both the drift term (related to $f(x)$) and the diffusion term (related to $g(x)$). It provides a powerful framework for understanding the dynamics of stochastic processes across various disciplines.

\newpage

$$
dx = a(x) dt + b(x) dW
$$

$$
y = \Psi(x) \quad \Rightarrow \quad d\Psi = [\Psi'(x) a(x) + Psi''(x) \dfrac{b^2(x)}{2}] dt + \Psi'(x) b(x) dW
$$

$$
\langle d\Psi \rangle = \langle \Psi'(x) a(x) + \Psi''(x) \dfrac{b^2(x)}{2} \rangle dt + \underbrace{\cancel{\langle \Psi'(x) b(x) \rangle dW}}_{\text{set to zero}}
$$

The last term is oscillatory and averages to zero so we can ignore it. The first term is the drift term of the process $y(t)$. We have:

$$
\dfrac{d}{dt}\langle\Psi\rangle = \langle \Psi'(x) a(x) + \Psi''(x) \dfrac{b^2(x)}{2} \rangle
$$

The average is given by $\int_{s} z(x) \rho(x,t) dx$, where $z(x)$ is the function of $x$ that we want to average. We have:

$$
\dfrac{d}{dt} \int_{\mathbb{R}} \Psi(x) \rho(x,t) dx = \int_{\mathbb{R}} \left[\Psi'(x)a(x)\rho(x,t) + \Psi''(x) \dfrac{b^2(x)}{2} \rho(x,t)\right] dx
$$

$$
\int_{\mathbb{R}} \Psi(x) \dfrac{\partial\rho}{\partial t}(x,t) dx 
=
\underbrace{\int_{\mathbb{R}} \Psi'(x) a(x) \rho(x,t) dx}_{I_1} + \underbrace{\int_{\mathbb{R}} \Psi''(x) \dfrac{b^2(x)}{2} \rho(x,t) dx}_{I_2}
$$

Let's integrate by parts the first term on the right-hand side:

$$
\begin{array}{rll}
I_1 = & \int_{-\infty}^{+\infty} \Psi'(x) R(x, t) dx = \big|\Psi(x) R(x, t)\big|_{-\infty}^{+\infty} - \int_{-\infty}^{+\infty} \Psi(x) \dfrac{\partial R}{\partial x} dx = 0 - \int_{-\infty}^{+\infty} \Psi(x) \dfrac{\partial R}{\partial x} dx
\\
I_2 = & \int_{-\infty}^{+\infty} \Psi''(x) z(x,t) dx
=
\big|\Psi'(x)z(x,t) \big| - \int_{-\infty}^{+\infty} \Psi'(x) \dfrac{\partial z}{\partial x} dx
=
- \int_{-\infty}^{+\infty} \Psi'(x) \dfrac{\partial z}{\partial x} dx
\\ = &
- \big|\Psi(x) \dfrac{\partial z}{\partial x} \big|_{-\infty}^{+\infty} + \int_{-\infty}^{+\infty} \Psi(x) \dfrac{\partial^2 z}{\partial x^2} dx
\end{array}
$$

Therefore, we have:

$$
\begin{array}{l}
    \displaystyle
    \int_{\mathbb{R}} \Psi(x) \dfrac{\partial\rho}{\partial t}(x,t) dx = - \int_{\mathbb{R}} \Psi(x) \left[ - \dfrac{\partial R}{\partial x} \right] dx + \int_{\mathbb{R}} \Psi(x) \left[ \dfrac{\partial^2 z}{\partial x^2} \right] dx
= 
\\
\displaystyle
=
\int_{\mathbb{R}} \Psi(x) \left\{ - \dfrac{\partial}{\partial x} \left[ a(x) \rho(x,t) \right] + \dfrac{\partial^2}{\partial x^2} \left[ \dfrac{b^2(x)}{2} \rho(x,t) \right] \right\} dx
\end{array}
$$

$$
\int_{- \infty}^{+ \infty} \Psi(x) \dfrac{\partial \rho}{\partial t} dx = \int_{- \infty}^{+ \infty} \left\{ - \dfrac{\partial}{\partial x} \left[ a(x) \rho(x,t) \right] + \dfrac{\partial^2}{\partial x^2} \left[ \dfrac{b^2(x)}{2} \rho(x,t) \right] \right\} \Psi(x) dx
$$

$$
\begin{cases}
\dfrac{\partial \rho}{\partial t} = - \dfrac{\partial}{\partial x} \left[ a(x) \rho(x,t) \right] + \dfrac{\partial^2}{\partial x^2} \left[ \dfrac{b^2(x)}{2} \rho(x,t) \right]
\\
\int_{- \infty}^{+ \infty} \rho(x,t) dx = 1
\end{cases}
$$

% \section{Derivation of the Fokker-Planck Equation via Ito's Formula}

% Consider the stochastic differential equation
% $$
% dx = a(x)\,dt + b(x)\,dW.
% $$
% Let $\Psi(x)$ be a twice differentiable test function. By applying Ito's formula to $\Psi(x)$, we have
% $$
% d\Psi = \Psi'(x)\,dx + \frac{1}{2}\Psi''(x)\,(dx)^2.
% $$
% Substituting
% $$
% dx = a(x)\,dt + b(x)\,dW,
% $$
% and recalling that $(dx)^2 = b^2(x)(dW)^2$, where the key property
% $$
% \boxed{\langle (dW)^2 \rangle = dt}
% $$
% holds, we obtain
% $$
% d\Psi = \left[\Psi'(x)a(x) + \frac{1}{2}\Psi''(x)b^2(x)\right]dt + \Psi'(x)b(x)\,dW.
% $$

% Taking expectations, the stochastic term (of order $O(\sqrt{dt})$) averages to zero:
% $$
% \langle d\Psi \rangle = \left\langle \Psi'(x)a(x) + \frac{1}{2}\Psi''(x)b^2(x) \right\rangle dt.
% $$
% Thus, the time derivative of the average of $\Psi(x)$ is
% $$
% \frac{d}{dt}\langle \Psi(x) \rangle = \left\langle \Psi'(x)a(x) + \frac{1}{2}\Psi''(x)b^2(x) \right\rangle.
% $$

% Now, the average of any function $z(x)$ with respect to the probability density $\rho(x,t)$ is given by
% $$
% \langle z(x) \rangle = \int_{\mathbb{R}} z(x)\,\rho(x,t)\,dx.
% $$
% In particular,
% $$
% \langle \Psi(x) \rangle = \int_{\mathbb{R}} \Psi(x)\,\rho(x,t)\,dx.
% $$
% Differentiating with respect to $t$, we have
% $$
% \frac{d}{dt}\int_{\mathbb{R}} \Psi(x)\,\rho(x,t)\,dx = \int_{\mathbb{R}} \Psi(x)\,\frac{\partial \rho}{\partial t}(x,t)\,dx.
% $$
% Equating the two expressions for $\frac{d}{dt}\langle \Psi(x) \rangle$, we obtain
% $$
% \int_{\mathbb{R}} \Psi(x)\,\frac{\partial \rho}{\partial t}(x,t)\,dx = \int_{\mathbb{R}} \left[\Psi'(x)a(x) + \frac{1}{2}\Psi''(x)b^2(x)\right]\rho(x,t)\,dx.
% $$

% Denote the two terms on the right-hand side by:
% \begin{align*}
% I_1 &= \int_{\mathbb{R}} \Psi'(x)a(x)\,\rho(x,t)\,dx, \\
% I_2 &= \int_{\mathbb{R}} \frac{1}{2}\Psi''(x)b^2(x)\,\rho(x,t)\,dx.
% \end{align*}

% Our goal is to shift the derivatives from $\Psi(x)$ onto $\rho(x,t)$ by integration by parts, so that the equality holds for any test function $\Psi(x)$.

% For the first term $I_1$, integrating by parts (and assuming that boundary terms vanish) yields
% $$
% I_1 = -\int_{\mathbb{R}} \Psi(x)\,\frac{\partial}{\partial x}\Bigl[a(x)\,\rho(x,t)\Bigr]\,dx.
% $$

% For the second term $I_2$, we perform integration by parts twice. First, we write
% $$
% I_2 = \left. \frac{1}{2}\Psi'(x)b^2(x)\,\rho(x,t) \right|_{-\infty}^{+\infty} - \int_{\mathbb{R}} \frac{1}{2}\Psi'(x) \frac{\partial}{\partial x}\Bigl[b^2(x)\,\rho(x,t)\Bigr]\,dx.
% $$
% Under the assumption that the boundary terms vanish, this simplifies to
% $$
% I_2 = -\int_{\mathbb{R}} \frac{1}{2}\Psi'(x) \frac{\partial}{\partial x}\Bigl[b^2(x)\,\rho(x,t)\Bigr]\,dx.
% $$
% Integrating by parts once more gives
% $$
% I_2 = \int_{\mathbb{R}} \Psi(x)\,\frac{\partial^2}{\partial x^2}\Bigl[\frac{b^2(x)}{2}\,\rho(x,t)\Bigr]\,dx.
% $$

% Combining $I_1$ and $I_2$, we have
% $$
% \int_{\mathbb{R}} \Psi(x)\,\frac{\partial \rho}{\partial t}(x,t)\,dx = \int_{\mathbb{R}} \Psi(x) \left\{ -\frac{\partial}{\partial x}\Bigl[a(x)\,\rho(x,t)\Bigr] + \frac{\partial^2}{\partial x^2}\Bigl[\frac{b^2(x)}{2}\,\rho(x,t)\Bigr] \right\} dx.
% $$

% Since this equality must hold for any smooth test function $\Psi(x)$, we conclude that the probability density $\rho(x,t)$ satisfies the Fokker-Planck equation:
% $$
% \boxed{
% \frac{\partial \rho}{\partial t}(x,t) = -\frac{\partial}{\partial x}\Bigl[a(x)\,\rho(x,t)\Bigr] + \frac{\partial^2}{\partial x^2}\Bigl[\frac{b^2(x)}{2}\,\rho(x,t)\Bigr]
% }
% $$
% with the normalization condition
% $$
% \int_{-\infty}^{+\infty} \rho(x,t)\,dx = 1.
% $$


\begin{exampleblock}[]

Let's consider the following SDE:

$$
\dot x = f(x) + \omega \xi(t)
\quad \rightarrow \quad
m\ddot x = -\dot x + f(x) + \omega \xi(t)
$$

if $m \ll 1$ we have $ \dot x = f(x) + \omega \xi(t) \quad f = - \dfrac{\partial U}{\partial x}$

$$
\begin{cases}
\dfrac{\partial \rho}{\partial t} = - \dfrac{\partial}{\partial x} \left[ f(x) \rho(x,t) \right] + \dfrac{\omega^2}{2} - \dfrac{\partial^2 \rho}{\partial x^2}
\\
\int_{- \infty}^{+ \infty} \rho(x,t) dx = 1
\end{cases}
$$

$\rho$ depends on $x$ and $t$, $P$ depends only on $x$:

$$
\begin{cases}
    \dfrac{\omega^2}{2} \dfrac{d^2 P}{dx^2} - \dfrac{d}{dx} \big[ f(x) P \big] = 0
    \\
    \int P(x) dx = 1
\end{cases}
$$
$$
\dfrac{\omega^2}{2} \dfrac{dP}{dx} = f(x) P
\quad \Rightarrow \quad
\dfrac{\omega^2}{2} \dfrac{dP}{x} = \dfrac{\partial U}{\partial x} P
\quad \Rightarrow \quad
\dfrac{dP}{P} = \dfrac{2}{\omega^2} \dfrac{\partial U}{\partial x} dx
$$

Now we can calculate C:

$$
P(x) = C e^{({2}/{\omega^2}) U(x)}
$$

\dots

$$
C = \dfrac 1{\displaystyle\int_{- \infty}^{+ \infty} e^{({2}/{\omega^2}) U(x)} dx}
$$

\dots

\begin{center}
    \begin{minipage}{0.4\textwidth}
    % tikz graphics of the potential U(x)
    \begin{tikzpicture}[scale=0.6]
        \begin{axis}[
            axis lines=left,
            xlabel=$x$,
            ylabel=$U(x)$,
            ymin=-1, ymax=105  % set y interval here for U(x)
        ]
        % Below the red parabola is defined
        \addplot [
            domain=-10:10,
            samples=100,
            color=red,
        ]
        {x^2};
        % \addlegendentry{$x^2$}
        \end{axis}
    \end{tikzpicture}
    \end{minipage}
    \begin{minipage}{0.4\textwidth}    
    % tikz graphics of P(x)
    \begin{tikzpicture}[scale=0.6]
        \begin{axis}[
            axis lines=left,
            xlabel=$x$,
            ylabel=$P(x)$,
            ymin=0, ymax=1.25  % set y interval here for P(x)
        ]
        % Below the red parabola is defined
        \addplot [
            domain=-10:10,
            samples=100,
            color=red,
        ]
        {exp(-x^2)};
        % \addlegendentry{$x^2$}
        \end{axis}
    \end{tikzpicture}
    \end{minipage}
\end{center}

\dots

\end{exampleblock}


\newpage

\section{Liouville Equation for Systems with Uncertain Initial Conditions}

Consider a physical system governed by the ordinary differential equation
$$
\begin{cases}
\displaystyle \frac{dx}{dt} = a(x),\\[1mm]
x(0)=x_0,
\end{cases}
$$
where the initial condition $x_0$ is not known exactly. Instead, we assume that $x_0$ is drawn from a probability distribution $\theta(x_0)$, so that the system is defined by
$$
\begin{cases}
\displaystyle \frac{dx}{dt} = a(x),\\[1mm]
x(0)=x_0 \sim \theta(x_0).
\end{cases}
$$

Suppose there exists an equilibrium point $x_e$ such that $a(x_e)=0$, and that $x_e$ is globally asymptotically stable (G.A.S.). This implies that, regardless of the uncertainty in the initial condition, the state $x(t)$ converges to $x_e$ as $t\to\infty$. Consequently, the probability density function (PDF) $\rho(x,t)$ of $x(t)$ evolves towards a Dirac delta distribution centered at $x_e$:
$$
\lim_{t\to\infty} \rho(x,t) = \delta(x-x_e).
$$

The time evolution of $\rho(x,t)$ is governed by the \textbf{Liouville equation}. For the deterministic dynamics
$$
dx = a(x)\,dt,
$$
the Liouville equation is given by
$$
\frac{\partial \rho}{\partial t} = - \frac{\partial}{\partial x}\Bigl[a(x)\,\rho(x,t)\Bigr].
$$

This partial differential equation expresses the conservation of probability along the flow of the system. The term $-\frac{\partial}{\partial x}\left[a(x)\,\rho(x,t)\right]$ represents the net flux of probability density in the state space due to the vector field $a(x)$. As time evolves and the system converges to the stable equilibrium $x_e$, the density $\rho(x,t)$ becomes increasingly concentrated around $x_e$, reflecting the loss of uncertainty in the long-term behavior of the system.

\newpage

\begin{align}
    m\dot v &= - \gamma v + F_s(t) \tag{I}\\
    Ri &= -L \dfrac{di}{dt} - K \dfrac{dB_{ext}}{dt} \tag{II}\\
    m \ddot x &= - \hat k x - \gamma \dot x + \hat F(t) \tag{III}
\end{align}

\dots

$$
\dot z = - \gamma z + \omega \xi(t) \leftrightarrow dz = . \gamma z dt + \omega \xi(t) dt \quad \Rightarrow \quad z = e^{- \gamma t} Q
$$

$$
- \gamma e^{\gamma t} Q dt + e^{-\gamma t} dQ = - \gamma e ^ {- \gamma t} Q + \omega dW
\quad \Rightarrow \quad 
e^{-\gamma t}dz = \omega dW
$$

So we have:

$$
dQ = e^{\gamma t} \omega dW \quad = \quad Q(t) = z(0) + \omega \int_{0}^{t} e^{\gamma s} dW(s) 
\quad \Rightarrow \quad
\boxed{
z(t) = z_0 e^{- \gamma t} + \omega \int_0^t e^{\gamma (s-t)} dW(s)
}
$$

$$
\langle z(t) \rangle = \langle z_0 \rangle e^{-\gamma t} + \Phi
$$

\dots

$$
z(t) = z(0)e^{- \gamma t} + \omega \int_0^t e^{\gamma (s-t)} dW(s)
$$

$$
\begin{array}{rl}
\langle z^2(t) \rangle =& 
\left\langle 
    \left(
        z_0 e^{-\gamma t} dW(s) + \int_0^t e^{\gamma(s - t)} \xi(s)ds
    \right)
    \left(
        z_0 e^{-\gamma t} + \int_0^t e^{\gamma(\theta - t) \xi(\theta)d\theta}
    \right)
\right\rangle
\\
=&
\left\langle
    \left(
        z_0 ^2 e^{-2\gamma t} + z_0 e^{-\gamma t} \int_0^t e^{\gamma(\theta - t)} \xi(\theta)d\theta + z_0 e^{-\gamma t} \int_0^t e^{\gamma(s - t)} \xi(s)ds + J(t)
    \right)
\right\rangle
\\
=&
\dots
\end{array}
$$

$$
\begin{array}{rl}
    \langle J(t) \rangle =& \omega^2 \int_0^t \int_0^t e^{\gamma(\theta + s - 2t)} \delta(\theta - s) d\theta ds\\
    =&
    \omega^2 e^{-2\gamma t} \int_0^t
     \left\{
        \int_0^t e^{\gamma(\theta + s)} \delta(\theta - s) d\theta 
    \right\}
    ds\\
    =&
    \omega^2 e^{-2\gamma t} \int_0^t e^{2 \gamma s} ds \\
    =&
    \omega^2 e^{-2\gamma t} \left[\dfrac{e^{2 \gamma t} - 1}{2 \gamma}\right] \\
    =&
    \dfrac{\omega^2}{2 \gamma} \big(1 - e^{-2 \gamma t}\big)
\end{array}
$$

$$
Var[z(t)] = \underbrace{\langle z_0^2 \rangle e^{-2 \gamma t} - (\langle z_0 \rangle)^2 e^{-2 \gamma t}}_{= \ Var(z_0)e^{-2 \gamma t}} + \dfrac{\omega^2}{2 \gamma} \big(1 - e^{-2 \gamma t}\big) = Var(z_0) e^{-2 \gamma t} + \dfrac{\omega^2}{2 \gamma} \big(1 - e^{-2 \gamma t}\big)
$$

$$
Var(z(t)) \to \dfrac {\omega}{2 \gamma} = \sigma^2
$$

$$
z(t) = e^{-\gamma t} + \int _0^t e^{\gamma(s-t)} \xi(s)dt
$$

\dots

$$z(0) = 0$$

$$
\langle z(t) z(q) \rangle = \omega^2 e^{-\gamma (t + q)} \int_0^t \int_0^q e^{\gamma(\theta + s)} \delta(\theta - s) d\theta ds
$$

We have 2 cases:

\begin{itemize}
    \item $\boldsymbol{t < q}$: 
    
        $$
        \langle z(t) z(q) \rangle = \omega^2 e^{-\gamma (t + q)} \int_0^t e^{2 \gamma s} ds 
        =
        \dfrac{\omega^2}{2 \gamma} e^{-\gamma (t - q)} \dots
        $$
        $$
        \langle z(t) z(q) \rangle = \dfrac{\omega^2}{2 \gamma} \left( e^{-\gamma |q - t|} - e^{-\gamma (q + t)} \right)
        $$

---

        $$
        z_0 = 0
        \quad
        \begin{cases}
        \langle z(t) \rangle = 0
        \\
        \langle z(q) \rangle = 0
        \end{cases}
        \quad
        C[\alpha, \beta] = \langle (\alpha - \hat \alpha)(\beta - \hat \beta) \rangle
        $$

        $$
        \rightarrow C[z(t), z(q)]; \quad q = t + h \quad X[z(t), z(t+h)] = \dfrac{\omega^2}{2 \gamma} \left( e^{-\gamma |h|} - e^{-\gamma h} e^{-2 \gamma t} \right)
        $$
        
        $$
        R_z(h) = \lim_{t \to \infty} C[z(t), z(t+h)] = \dfrac{\omega^2}{2 \gamma} e^{-\gamma |h|}
        $$
        
    \item $t > q$:
    

\end{itemize}