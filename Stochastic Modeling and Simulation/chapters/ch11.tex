\chapter{Lecture 11/04/2025}

---

$$
\dot x = a(x) + b(x) \eta_h (t)
$$

$$
\langle \eta_h(t)\rangle = 0
$$

$\mathbb R_h$ is a function with a peak at $|z| < h$ and $\mathbb R_h (z) = 0$ for $|z| > h$.

\vspace{0.5em}

If the limit $\lim_{h \to 0^+} \mathbb R_h (z) = \delta(\tau)$, then:

$$
\dot x_h = a(x_h) + b(x_h) \eta_h (t)
\quad \to \quad
\dot x = a(x) + b(x) \circ \xi(t)
$$

---

Let's consider a population and two opinions:

$$
x + y = 1
$$

People can change their opinion with a rate $R$ and the ratio of changing from $x$ to $y$ is $\theta$, and from $y$ to $x$ is $k$.

$$
\begin{cases}
    \dot x = + \theta xy - k yx - \varepsilon x + \varepsilon y \\
    \dot y = - \theta xy + k yx + \varepsilon x - \varepsilon y
\end{cases}
$$

where $\varepsilon$ is the rate of changing opinion. (?)

Substituting $y = 1 - x$ we can rewrite the first equation as:
$$
\dot x = x(1-x) (\theta - k) + \varepsilon (1-x) - \varepsilon x
$$

$$
\dfrac{dx}{dt} = \lambda x(1-x) + 1 - 2 x, \quad \lambda \to \lambda + \alpha \xi(t)
$$

$$
dx = \underbrace{\{\lambda x(1-x) + 1 - 2 x\}}_{a(x)} dt + \underbrace{\alpha x(1-x)}_{b(x)} \circ dW
$$

\dots

$$
dx = \left\{
    1 - 2x
\right\} dt + \alpha x(1-x) \circ dW
$$

$$
a(x) = \tfrac 12 b'(x) b(x),
\qquad
b(x) = \alpha (x - x^2),
\qquad
b'(x) = \alpha (1 - 2x)
$$

so

$$
1 - 2x = \dfrac {\alpha^2}2 x(1-x)(1-2x)
$$

which has two equilibrium points:

$$
x_1 = \dfrac 12, \qquad x_2: 1 = \dfrac {\alpha^2}2 x(1-x) \ \ (?)
$$

\dots

---

$$
m \ddot x = - \gamma_T \dot x + F_T(x) + \omega_T \xi(t)
$$

$$
\begin{cases}
    \dot x = v \\
    \dot v = - \dfrac{\gamma_T}m v + \dfrac{F_T(x)}m + \dfrac{\omega_T}m \xi(t)
\end{cases}
$$

to simplify the notation we can write:
$$
\gamma = \dfrac{\gamma_T}m, \qquad
F = \dfrac{F_T(x)}m, \qquad
\omega = \dfrac{\omega_T}m, \qquad
\underbrace{U = \int F_T(x) dx, \qquad
U' = \dfrac{dU}{dx} = F_T(x)}_{?}
$$

we get:

$$
\begin{cases}
    \dot x = v \\
    \dot v = - \gamma v + F(x) + \omega \xi(t)
\end{cases}
$$

$$
\dfrac{\partial p}{\partial t} = - v \dfrac{\partial p}{\partial x} - \dfrac{\partial}{\partial v} \left[
    (F(x) - \gamma v) p
\right] + \dfrac{\partial^2}{\partial v^2} \dfrac {\omega^2} 2
$$

$$
\dfrac{\partial p}{\partial t} = - v \dfrac{\partial p}{\partial x} - F(x) \dfrac{\partial p}{\partial v} + \gamma \dfrac{\partial}{\partial v}(vp) + \dfrac{\partial^2}{\partial v^2} \dfrac {\omega^2} 2
$$

$$
0 = - v \dfrac{\partial p}{\partial x} + U'(x) \dfrac{\partial p}{\partial v} + \gamma p + \gamma v \dfrac{\partial p}{\partial v} + \dfrac{\partial^2}{\partial v^2} \dfrac {\omega^2} 2
$$

$$
p(x,v) = A(x)B(v)
$$
$$
-v A'(x) B(v) + U'(x) A(x) B'(v) + \gamma A(x) B(v) + \gamma v A(x) B'(v) + \dfrac {\omega^2} 2 B''(v) = 0
$$

$$
\overbrace{\underbrace{- v \dfrac{A'(x)}{A(x)} + U'(x) \dfrac{B'(v)}{B(v)}}_{= \ 0}}^{1^{st} \text{ term}} + 1\overbrace{\gamma + \gamma v \dfrac{B'(v)}{B(v)} + \dfrac {\omega^2} 2 \dfrac{B''(v)}{B(v)}}^{2^{nd} \text{ term}} = 0
$$

We have to options:
 
\begin{enumerate}
    \item set the second term to zero
    \item boh
\end{enumerate}

Let's consider the first option and let's define some "test" variables $B_T$ and $B'_T$. We have:

$$
\dfrac{B'_T(v)}{B_T(v)} = - \eta v 
\quad \Rightarrow \quad
B'_T(v) = - \eta v B_T(v)
\quad \Rightarrow \quad
B(v) = C e^{-\eta v^2/2}
$$

we have

$$
B'(v) = - \eta v B(v), \qquad
B''(v) = - \eta v B'(v) = - \eta v \left( - \eta v B(v) \right)
$$

\missing{boh}

$$
P_s = \dfrac 1z e ^{- \tfrac \gamma{\omega^2} v^2 - \tfrac{2 \gamma}{\omega^2} U(x)}
=
\dfrac 1z e ^{- \tfrac {2\gamma}{\omega^2} \left[
    \tfrac{v^2}2 + U(x)
\right]}
$$

Applying back the transformation we have:

$$
p(x,v) = \dfrac 1z e ^{- \tfrac {2\gamma_T}{\omega_T} \left[
    \tfrac{mv^2}2 + U_T(x)
\right]}
$$
so:
$$
\iint p_s(x,v)dxdv = 1, \qquad
\dfrac 1z \iint e ^{- \tfrac {2\gamma_T}{\omega_T^2} E_T(x,v)} dxdv = 1
$$

\todo{check if this is correct}

\missing{fishes example ?}

$$
dx = f(x) dt - \underbrace{(cx dt + \omega x dW)}_{\# \text{fishes killed in } (t,\ t + dt)}
$$

We want the number of fishes to be positive.

\missing{end of the lecture}
