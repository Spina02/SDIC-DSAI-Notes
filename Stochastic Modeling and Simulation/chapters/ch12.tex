\chapter{Lecture: 05/05/2025}

\dots

... if there is no linearity, ...

$$
\dfrac{\partial P}{\partial t} = - \dfrac{\partial}{\partial x} \left\{
    (\theta \int z P(z,t) \dd z + (1-\theta)x - x^3)P
\right\} + \dfrac{\omega^2}2 \dfrac {\partial^2 P}{\partial x^2} \qquad N \gg 1
$$

$$
M(t) = \int_{\mathbb R} x P(x,t) dx
$$

$$
P_s(x,M_s) = C(M_s) \exp \left\{
    \theta M_s x + ...
\right\}
$$

\missing{end of the formula above}

This solution is not actually so "usable"

$$
M_s = \int_{\mathbb R} x P_s(x; M_s) dx \quad \Rightarrow \quad M_s = \Psi (M_s)
$$

$$
M_s = \Psi(M_s) \quad \to \quad \text{"unique solution"}
$$

There are more interesting cases, for instance when $\Psi(M_s)$ has more than one solution:

In this case, our system has more than one steady states. It means that we loose the unicity of the solution (so there is no more global actractiveness)

\vspace{0.5em}

E.g: 

$$\dot X_i = f(x_i, <x>) + g(x_i) \xi_i \quad N \gg 1$$

$$
\dot x = f(x,M(t)) + g(x) \xi(t)
$$

$$
M(t) = \int z P(z,t) dt
$$

The Fokker-Plank equation will be:

$$
\dfrac{\partial P}{\partial t} = - \dfrac \partial{\partial x} [f(x, M(t))] + \dfrac 12 \dfrac {\partial^2}{\partial x^2} [g^2(x)P]
$$

The steady state solutions will be the solutions of the following equation:

$$
\begin{cases}
0 = -\dfrac {\dd}{\dd x} [t(x,M_s)P] + \dfrac {\dd^2}{\dd x^2} \left[
    \dfrac{g^2(x)}2 P
\right]
\\[1em]
M_s = \int z P_s (z; M_s) dx
\end{cases}
$$

$$
\boxed{
    M_s = \Psi(M_s)
}
$$

\todo{add linking sentence}

$$
P(x, M_s, \theta) = C(M, \theta) \exp \left[
    \dfrac 2\omega \left(
        \theta M_s x + (1-\theta) \dfrac {x^2}2 - \dfrac {x^4}4
    \right)
\right]
$$

$$
\boxed{M_s = 0}
$$

\dots (?)

$0 < \theta < \theta_c$

\dots (?)

So we have two solutions:

$$
\begin{array}{l}
    M_s = a\\
    M_s = -a
\end{array}
$$

\begin{exampleblock}
    $$
    M_s = \Psi(M_s; \theta)
    $$

    $$
    \begin{cases}
        y = M_s\\
        y = \Psi(M_s; \theta)
    \end{cases}
    \quad \Rightarrow \quad
    \begin{array}{c}
        P_s(x;M_1)\\
        P_s(x;M_2)\\
        P_s(x;M_3)
    \end{array}
    $$

    so for $\theta = \theta_1$ we have \bfit{multistability}, while for $\theta = \theta_2$ we have \bfit{monostability}.

\end{exampleblock}


\newtheorem{theorem}{Theorem}

\begin{theorem}
    $$
    \left|
        \left.\dfrac {\dd \Psi}{\dd M_s}
    \right|_{M_s = M_c}\right| < 1
    \quad \Rightarrow \quad
    P_s(x;M_1, \theta^1) \text{ is locally stable}
    $$
    $$
    \left|
        \left.\dfrac {\dd \Psi}{\dd M_s}
    \right|_{M_s = M_c}\right| > 1
    \quad \Rightarrow \quad
    P_s(x;M_2, \theta^2) \text{ is locally unstable}
    $$
\end{theorem}

\begin{center}
    \begin{minipage}{0.4\textwidth}
        \includegraphics[width=0.8\textwidth]{assets/ex.png}
    \end{minipage}
    \begin{minipage}{0.4\textwidth}
        \includegraphics[width=0.8\textwidth]{assets/ex2.png}
    \end{minipage}%
\end{center}

\dots

\newpage

$$
\dot x = (ax + x^3 - x^5) - D(x - M(t)) + \alpha (1+x^2) \odot \xi(t)
$$

$$
M = \Psi(M;D,\alpha)
$$

we have that for small $D$ and $\alpha$ we have a unique solution, while for large $D$ and $\alpha$ we have 5 different solutions.

\begin{center}
    \begin{minipage}{0.4\textwidth}
        \begin{figure}[H]
            \centering
            \includegraphics[width=0.8\textwidth]{assets/ex_2.png}
            \caption{1 solution}
        \end{figure}
        \end{minipage}%
    \begin{minipage}{0.4\textwidth}
    \begin{figure}[H]
        \centering
        \includegraphics[width=0.8\textwidth]{assets/ex_1.png}
        \caption{5 solutions}
    \end{figure}
    \end{minipage}
\end{center}

So 0 is always a solution, and from a certain value of $D$ we have 5 solutions, 3 of which are stable and 2 are unstable.

\begin{figure}[H]
    \centering
    \includegraphics[width=0.3\textwidth]{assets/ex_3.png}
    \caption{Stable solutions of the system}
\end{figure}

\dots

$$
\dot x = F(x_i, <x>) + g(x_i, <x>) \xi_i(t)
$$

An example iis the movement of a guitar string that vibrates.

$$
m_i \ddot x_i = - \gamma \dot x_i
$$

If the deviation is big, we have a function in the copmplex space, but if the deviation is small, we can use the linear approximation.

$$
m_j \ddot z_j = - \gamma \dot z_j - k(z_j - z_{j-1}) - k(z_j - z_{j+1}) = - \gamma \dot z_j - k \left(z_{j-1} - 2z_j + z_{j+1}\right)
$$

we have now a discretization of the position of the string $z(t,x)$:

$$
m_j \dfrac {\dd^2 z}{\dd t^2} (t, x_j) = - \gamma \dfrac {\dd z}{\dd t} (t,x_j) + k [z(t, x_j + D) - 2z(t,x_j) + z(t, x_j - D)] 
$$

$$
\mu \dfrac{\partial^2 z}{\partial t^2} = - \gamma \dfrac{\partial z}{\partial t} + c \dfrac{\partial^2 z}{\partial x^2} + \hat \omega \xi(x,t)
$$

We can see the stochastic term as the wind that moves the string.

