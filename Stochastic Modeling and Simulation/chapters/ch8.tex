\chapter{Lecture 28/03/2025}

\subsubsection{multiplicative noise}

The general form of a SDE with multiplicative noise is given by:

$$
dx = fx dt + g(x) dW
$$

We already saw a case of multiplicative noise:

$$
\dot x = (r_1 + \omega \xi(t))x - r_2 x^2 \quad \rightarrow \quad dx = \underbrace{(r_1 x - r_2 x^2)}_{f(x)}dt + \underbrace{\omega x}_{g(x)} dW
$$

$$
\dfrac{\partial \rho}{\partial t} = - \dfrac{\partial}{\partial x} [f(x) \rho] + \dfrac{\partial ^2}{\partial x^2} \left[ \dfrac{g^2(x)}{2} \rho \right]
$$

$$
\dfrac{d^2}{dx^2} \left[ \dfrac{g^2(x)}{2} \rho \right] = - \dfrac d{dx} [f(x)P_s]
$$

$$
\begin{cases}
    \dfrac{d}{dx} \left[ \dfrac{g^2(x)}{2} \rho \right] = - f(x)P \\
    Q(x) = \dfrac{g^2(x)}{2} \rho \quad \Rightarrow \quad P = \dfrac 2 {g^2(x)} Q
\end{cases}
$$
$$
\dfrac{dQ}{dx} = - f(x) \dfrac 2 {g^2(x)} Q
$$
$$
\dfrac{dy}{dx} = a(x) y \quad \Rightarrow \quad
\begin{cases}
    y(x) = Ce^{A(x)}\\
    A(x) = \int_\alpha^x a(s) ds
\end{cases}
$$
$$
Q(x) = C \exp\left\{
    \int_\alpha^x - \dfrac{2 f(s)}{g^2(s)} ds
\right\}
\quad \Rightarrow \quad
P(x) = \dfrac{2}{g^2(x)} \exp \left\{
    \int_\alpha^x \dfrac{2 f(z)}{g^2(z)} dz
\right\}
$$

Now we have to derivate the probability density function, before doing this, let's rewrite it in a way that simplifies calculus.

Considering that:

$$
\dfrac 1{g^2(x)} = e^{-\ln g^2(x)} = e^{-2 \ln g(x)}
$$

We can rewrite the probability density function as:

$$
P(x) = 2 C \exp \left\{
    - 2 \log (g(x)) + \int_\alpha^x \dfrac{2 f(s)}{g^2(s)} ds
\right\}
$$

Now we can derivate:

$$
P'(x) = 2 C \exp \left\{
    - 2 \dfrac{g'(x)}{g(x)} + \dfrac{2 f(x)}{g^2(x)}
    \right\}
$$

\dots

$$
\boxed{f(x) \ge g'(x)g(x)}
$$

\dots

$$
g(x) = \omega \Rightarrow P'_s(x) \ge 0 \equiv f(x) \ge 0 \qquad - \dfrac {dU}{dx} \ge 0
$$

As results we have that:

\begin{enumerate}
    \item Extrema of $P_s$ are different from extrema of $U(x)$ (= equoilibrium point of the deterministic system).
    
    \item The number of extrema of $P_s$ is different from the numnber of equilibrium points.
\end{enumerate}

\subsubsection{Something biological idk why}

Proteins in our cells follows a self-assembly model. The model is given by the following SDE:

$$
\dfrac {dx}{dt} = \Pi_R(x) - \delta x
$$

where $\Pi_R(x)$ is the production rate of the protein.

The production rate is given by the following equation:

$$
\Pi_R(x) = R + \dfrac {k x^2}{k_0 + x^2}
$$

The equilibrium points are given by:

$$
\Pi_R(x) = \delta x
$$

Let's now consider $\delta = \delta + \alpha \xi(t)$, where $\alpha$ is a constant and $\xi(t)$ is a white noise.
We can rewrite the equation as:

$$
\dfrac{dx}{dt} = \left(
    R + k \dfrac{x^2}{k_0 + x^2} - \delta x
\right)
+ \alpha x \xi(t)
$$

In this case we have a multiplicative noise, so we can use the previous results, but now we do not have to intercept $\Pi_R$ with $\delta$, but we have to intercept it with $\delta + \alpha^2$.

$$
\begin{cases}
    R + k \dfrac{x^2}{k_0 + x^2} - \delta x \ge \alpha^2 x \\
    R + k \dfrac{x^2}{k_0 + x^2} \ge (\delta + \alpha^2) x
\end{cases}
$$

\newpage

$$\dot z = r_1 z - r_2 z, \qquad x = r_2 z \to z = \dfrac {x}{r_2}$$

$$
\begin{cases}
    \dot x = r_1 x - x^2 \\
    r_1 \to r_1 + \omega \xi(t)
\end{cases}
\quad \Rightarrow \quad
dx = (r_1 x - x^2) dt + \omega x dW
$$

$$
f(x) = r_1 x - x^2,
\qquad
g(x) = \omega x
$$

$$
r x - x^2 \ge \omega^2 x
\qquad \Rightarrow \qquad
r_1 - x \ge \omega^2
$$

in $\omega^2 = r_1$ we have a transition from an unimodal stationary distribution to a decreasing function.

$$
P_s = \dfrac {2C}{\omega^2} x^{2r_1/\omega^2 - 2}\ e^{-(2/\omega^2) x}
$$

which is not integrable

\subsubsection{voglio anna a casa}

$$
dx = f(x) dt + g(x) dW 
$$

$$
x(t + dt) = x(t) + f(x(t))dt + g(x(t)) G_t \sqrt{dt}
$$

We have that the probability of moving from $s$ to $a$ in a time $dt$ is given by:

$$
\Pr\left(x(t + dt) "=" a | x(t) = s\right) = \Omega(s, a) dt 
$$

While the probability of not moving is given by:

$$
\Pr\left(x(t + dt) "=" s | x(t) = s\right) = 1 - \int \Omega(s,a)dt 
$$


