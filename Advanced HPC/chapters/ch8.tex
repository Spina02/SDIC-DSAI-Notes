\chapter{Message Passing Interface - MPI}

Advanced Usage of some MPI features on Topology awareness, Shared memory and One-sided communications.

\begin{itemize}
    \item Basics of the One-sided communication
    \item Buildinga  hierarchy of Communicators that reflects the topology
    \item Exchanging data in shared memory windows among MPI processes
\end{itemize}

and we will see briefly other advanced features of MPI:
\begin{itemize}
    \item Derived Data-Types
    \item More details on Groups and Communicators
    \item Virtual topologies (cartesian and graph communicators) and neighborhood collectives
    \item Non-blocking collectives
\end{itemize}

\subsubsection{Two-Sided communications}

By its very nature, the message-passing paradigm is designed around the concept of cooperative exchange of informations among two or more processes whose address space are isolated and not directly inaccessible by other processes.


    
% \section{One-Sided Communication}

\section{Remote Memory Access - RMA}

\subsection{Memory Windows}

% TODO: talk about the window creation

\begin{warningblock}[\plaintt{disp\_unit}]
    The \plaintt{disp\_unit} parameter is often misunderstood. Setting it to anything other than 1 (byte) can cause subtle bugs when mixing datatypes. The displacement calculation becomes:
\begin{codeblock}[language=C]
target_addr = base + (target_disp * disp_unit)
\end{codeblock}
\end{warningblock}

% \subsection{Data Movement}

\begin{observationblock}[Flush vs Sync]

The fundamental difference between \plaintt{MPI\_Win\_flush} and \plaintt{MPI\_Win\_sync} is that:
\begin{itemize}
    \item \texttt{MPI\_Win\_flush}: Forces completion of RMA operations from the origin to the target.
    \item \texttt{MPI\_Win\_sync}: Synchronizes local memory with the window (target-side operation).
\end{itemize}

Or, in other words:

\begin{itemize}
    \item \texttt{MPI\_Win\_flush} is about \textit{Network Completion} (\emph{Did the data arrive?})
    \item \texttt{MPI\_Win\_sync} is about \textit{Memory Consistency} (\emph{Is the data visible to the CPU?})
\end{itemize}

This distinction is critical and often misunderstood: these calls operate on different sides of the communication and serve completely different purposes.
\end{observationblock}
    

% \subsection{Ordering and Synchronization}

% \subsubsection{Fences}

% \subsubsection{Post-Start-Complete-Wait - PSCW}

\subsubsection{Locks}

Locks are a way to synchronize the access to a shared resource. They are a way to ensure that only one process can access the shared resource at a time.

Passive mode: the target does not participate in opertions: one sided asynchronous communication.

\begin{codeblock}[language=C]
MPI_Win_lock(int lock_type, int rank, int assert, MPI_Win win);

MPI_Win_unlock(int rank, MPI_Win win);

MPI_Win_flush/flush_local(int rank, MPI_Win win);
\end{codeblock}

where \texttt{lock\_type} is one of the following:
\begin{itemize}
    \item \texttt{MPI\_LOCK\_EXCLUSIVE}: only one process can access the shared resource at a time.
    \item \texttt{MPI\_LOCK\_SHARED}: multiple processes can access the shared resource at a time.
\end{itemize}

\texttt{rank} is the rank of the process that will access the shared resource.

\texttt{assert} is the assertion that will be used to synchronize the access to the shared resource.

\begin{itemize}
    \item \texttt{MPI\_Win\_flush}: complete operations on remote target process; data will be then available to the target task, or to other tasks
    \item \texttt{MPI\_Win\_flush\_local}: locally complete operations to the target process
\end{itemize}


\begin{warningblock}[Misconception about locks]
    Despite the unfortunate naming, MPI locks are not mutex-like. They do not provide mutual exclusion for the target process's local access. The target can modify its window memory during a locked epoch, leading to race conditions.
\end{warningblock}

\subsubsection{Assertions}

% TODO: talk about synchronization asserts, cite the main ones

% \subsection{RMA Memory Model}

\section{Shared Memory}

An HPC machine is made of several nodes that are connected by a top-level network We call these nodes "hosts".

Currently every host contains a NUMA region which consists in a number of sockets. Each of which is physically connected to RAM banks.
In turn, the sockets are interconnected so that a process running on a given core can access a memory bank not physically attached to its socket.

\subsection{Getting Shared Memory Pools}

\subsubsection{General and specific routines}

MPI provides a method to know about that and to distinguish the tasks depending some of their charateristics:

\begin{itemize}

    \item a \plaintt{general routine} to split a communicator into sub-communicators:

\begin{codeblock}[language=C]
MPI_Comm_split ( 
    MPI_Comm comm,      // the communicator to be splitted
    int color,          // discriminate task
    int key,            // hint for new ranks
    MPI_Comm *newcomm   // the new communicator
);
\end{codeblock}

\item a \plaintt{specific routine} to split a communicator into sub-communicators following the hardware hierarchy (i.e. NUMA region, socket, core, etc.):
\begin{codeblock}[language=C]
MPI_Comm_split_type ( 
    MPI_Comm comm,      // the communicator to be splitted
    int split_type,     // the splitting property
    int key,            // hint for new ranks
    MPI_Info info,      // helper for the splitting property
    MPI_Comm *newcomm   // the new communicator
);
\end{codeblock}

The \plaintt{split\_type} parameter is used to specify the splitting property. It can be one of the following values:
\begin{itemize}
    \item \plaintt{MPI\_COMM\_TYPE\_SHARED}: split the communicator based on the shared memory pools.
    \item \plaintt{MPI\_COMM\_TYPE\_HW\_GUIDED}: split the communicator based on the hardware guided property.
    \item \plaintt{MPI\_COMM\_TYPE\_HW\_UNGUIDED}: split the communicator based on the hardware unguided property.
\end{itemize}

The \plaintt{key} parameter is used to hint for the new ranks. It is used to sort the tasks depending on their charateristics.

The \plaintt{info} parameter is used to provide helper for the splitting property. It is used to provide additional information about the splitting property.

\end{itemize}

\subsubsection{Placement}

\begin{codeblock}[language=C]
MPI_Win_allocate_shared ( 
    MPI_Aint size,        // the size of the window
    int disp_unit,        // the displacement unit
    MPI_Info info,        // the information to be used
    MPI_Comm comm,        // the communicator
    void *baseptr,        // the base pointer
    MPI_Win *win          // the window
);
\end{codeblock}

As with other window-creation routines, the memory allocation for \texttt{MPI\_Win\_allocate\_shared} does not require the allocated size to be uniform: each task may allocate a different amount. There is no strict specification as to where the memory should be physically placed: an MPI implementation will typically attempt to maximize data locality and affinity.

By default, the allocation is \textbf{contiguous}. However, this contiguity does not only refer to the virtual address space: ideally, it also implies allocation in the same physical memory bank. On many systems, though, it is not possible to remap memory at arbitrary sizes to guarantee this property. Specifying a non-contiguous allocation can allow the MPI implementation to more flexibly optimize the placement, possibly enhancing the affinity for each individual task.

% TODO: mentioning caveats


A useful routine related to window allocation and memory sharing is \plaintt{MPI\_Win\_shared\_query}, which allows processes to retrieve a pointer to shared memory in a window, for either direct access or load/store operations. Its prototype is:

\begin{codeblock}[language=C]
MPI_Win_shared_query(
    MPI_Win   win,         // the window object
    int       rank,        // rank whose segment to query, or MPI_PROC_NULL
    MPI_Aint *size,        // returns segment size
    int      *disp_unit,   // returns displacement unit
    void    **baseptr      // returns base pointer in shared memory
);
\end{codeblock}

This function enables a process to obtain the base address and size of the memory segment exposed by any rank in a shared-memory communicator window. Commonly, it is called with \plaintt{rank = MPI\_PROC\_NULL}, which gives a pointer to the start of the entire shared memory region (regardless of which process allocated it), and returns:

\begin{itemize}
    \item the address of the first byte in the shared memory window (for contiguous allocation), or
    \item the base address of the first process that specified a non-zero \plaintt{size} (for non-contiguous allocation).
\end{itemize}

% \begin{center}
% \includegraphics[width=0.95\textwidth]{contiguity_example.png}
% \end{center}

The figure above illustrates the difference between default contiguous allocation and non-contiguous allocation across multiple processes. With contiguous allocation, addresses increase sequentially across processes, so the pointer returned can be used by all processes to access the underlying shared array as a unified space. In non-contiguous mode, each process's memory may be mapped separately, and thus, only the base address of the first valid allocation is returned.

This routine is fundamental for writing portable and generic code utilizing shared memory windows, especially when contiguity cannot be assumed.

% \subsection{Shared Memory Access}

\section{Cartesian Communication}

Many scientific and engineering problems operate on regular, structured grids: finite difference methods for PDEs, stencil computations, image processing, and particle-mesh methods in astrophysics. When parallelizing these applications via domain decomposition, we partition the computational domain into sub-domains, each handled by a distinct MPI process.

The fundamental challenge is neighbor communication: processes must exchange boundary data (ghost zones, halo cells) with their neighbors. While we could manually compute neighbor ranks using arithmetic on process coordinates, this approach is error-prone, non-portable, and fails to leverage potential hardware optimizations.

\subsubsection{Why not Manual Rank Arithmetic}

Consder a 2D grid decomposition with dimensions ($P_x \times P_y$). Given a process with coordinates ($i, j$), its neighbors are:

\begin{codeblock}[language=C]
// Manual neighbor calculation (fragile approach)
int rank_left  = (i > 0)     ? i-1 + j*Px   : MPI_PROC_NULL;
int rank_right = (i < Px-1)  ? i+1 + j*Px   : MPI_PROC_NULL;
int rank_down  = (j > 0)     ? i + (j-1)*Px : MPI_PROC_NULL;
int rank_up    = (j < Py-1)  ? i + (j+1)*Px : MPI_PROC_NULL;
\end{codeblock}

The possible issues, or inconveniencies, with this approach are:

\begin{itemize}
    \item \textbf{Boundary handling}: Explicit conditionals for edges; periodic boundaries add complexity
    \item \textbf{Dimension ordering}: Row-major vs column-major rank assignment affects all calculations
    \item \textbf{Scalability}: 3D or higher dimensions require nested conditionals
    \item \textbf{Optimization}: MPI implementations cannot optimize message routing without knowing the topology
\end{itemize}

\subsubsection{Cartesian Communication Model}

MPI provides virtual topologies to express logical process arrangements. A \textbf{Cartesian topology} maps processes onto a regular $n$-dimensional grid, and MPI provides a suite of routines for creating and querying such topologies:

\begin{itemize}
    \item \textbf{Automatic coordinate $\leftrightarrow$ rank translation}: Convert between process ranks and their multidimensional coordinates in the grid.
    \item \textbf{Built-in periodic boundary support}: Specify whether each grid dimension is periodic (wrap-around), e.g., for toroidal grids.
    \item \textbf{Neighbor lookup via relative displacement}: Functions for obtaining neighbor ranks given directional offsets ("shifts")—no manual arithmetic required.
    \item \textbf{Hints for process placement optimization}: Inform the MPI implementation of communication patterns, potentially improving performance by optimizing process placement (implementation-dependent).
\end{itemize}

\begin{exampleblock}[Non-Periodic 3x2 Grid]

\begin{codeblock}[language=C, numbers=none]
        j=0    j=1
      +------+------+  
i=0   |  R0  |  R1  |   Rank = i + j*3
      +------+------+  
i=1   |  R2  |  R3  |   R0:(0,0) R1:(1,0)
      +------+------+   R2:(0,1) R3:(1,1)
i=2   |  R4  |  R5  |   R4:(0,2) R5:(1,2)
      +------+------+  
\end{codeblock}
\end{exampleblock}

\subsection{Core API Functions}

\subsubsection{MPI\_Dims\_create}

Computes a "balanced" distribution of processes across dimensions:

\begin{codeblock}[language=C]
int MPI_Dims_create (int nnodes,    // Total number of processes
                     int ndims,     // Number of dimensions
                     int dims[]);   // IN/OUT: dimension sizes
\end{codeblock}


\subsubsection{MPI\_Cart\_create}

Creates a new communicator with processes laid out on a Cartesian grid:

\begin{codeblock}[language=C]
int MPI_Cart_create(MPI_Comm comm_old, int ndims, const int dims[], const int periods[], int reorder, MPI_Comm *comm_cart);
\end{codeblock}

Key behavior: Pre-set dimensions (dims[i] > 0) are preserved; zero entries are computed.

The algorithm tries to minimize surface-to-volume ratio for communication efficiency.

\begin{exampleblock}[2D grid, 12 processes]
\begin{codeblock}[language=C]
// unconstrained: let MPI choose
int dims[2] = {0, 0};
MPI_Dims_create(12, 2, dims); // Result: dims = {4, 3} or {3, 4}

// constrained: force 4 columns
int dims2[2] = {0, 4};
MPI_Dims_create(12, 2, dims2); // Result: dims2 = {3, 4}
\end{codeblock}
\end{exampleblock}

\begin{warningblock}[MPI\_Dims\_create failure]
    MPI\_Dims\_create fails if \plaintt{nnodes} is not divisible by the product of pre-set dimensions. Always verify \plaintt{dims[i] > 0} for all \plaintt{i} after the call.
\end{warningblock}

\subsubsection{MPI\_Cart\_create}

Creates a new communicator with Cartesian topology:

\begin{codeblock}[language=C]
int MPI_Cart_create(
    MPI_Comm comm_old,      // Input communicator
    int ndims,              // Number of dimensions
    const int dims[],       // Size of each dimension
    const int periods[],    // Periodicity flags (0 or 1)
    int reorder,            // Allow rank reordering?
    MPI_Comm *comm_cart     // Output: new Cartesian comm
);
\end{codeblock}

where:
\begin{itemize}
    \item \texttt{periods[i]}: If non-zero, dimension \texttt{i} wraps around (periodic boundary)
    \item \texttt{reorder}: If non-zero, MPI may reassign ranks to optimize for hardware topology
    \item If size(\texttt{comm\_old}) > product(\texttt{dims}), excess processes receive \texttt{MPI\_COMM\_NULL}
\end{itemize}

\begin{tipsblock}[Best practices]
    Set reorder=1 in production code. This allows MPI implementations (especially on clusters with complex interconnects like fat-trees or torus networks) to map logically adjacent processes to physically adjacent nodes, reducing communication latency.
\end{tipsblock}