\chapter{Introduction}

Operating systems (OS) are the core software that manage a computer's hardware and software resources. Among the most widely used are Microsoft Windows, Apple's macOS, and the open-source Linux. While they differ in many aspects, both macOS and Linux are part of the \bfit{Unix-like} family of operating systems. These systems are renowned for their stability, security, and a powerful command-line interface known as the \bfit{shell}.

This chapter provides an introduction to the Unix shell, exploring its fundamental role in interacting with the system.

\subsubsection{File System}

A crucial component of the operating system is the file system, which organizes files and directories on a computer. It is a crucial component of the operating system that allows users to store, retrieve, and manage data efficiently. In unix-like systems, the file system is organized in a hierarchical structure.

\vspace{-0.5em}

\begin{figure}[H]
    \centering
    \includegraphics[width=0.8\textwidth]{assets/filesystem.png}
    \caption{The File System}
    \label{fig:filesystem}
\end{figure}

\subsubsection{Path}

A path specifies the unique location of a file or directory within the file system's hierarchy. Paths can be \bfit{absolute}, starting from the root directory (\texttt{/}), or \bfit{relative} to the current working directory.

\vspace{0.5em}

\begin{center}
    \begin{tabular}{clccl}
        \texttt{/} & Root Directory & & \texttt{$\sim$} & Home Directory \\
        \texttt{.} & Current Directory & & \texttt{..} & Parent Directory \\
        
    \end{tabular}
\end{center}

\vspace{1em}

\begin{exampleblock}[Example of a path]
    \begin{itemize}
        \item a file in the user's Desktop: \plaintt{/home/user/Desktop/file.txt} or \plaintt{\textasciitilde/Desktop/file.txt}
        \item a file in the current directory: \plaintt{./file.txt}
        \item a file in the parent directory: \plaintt{../file.txt}
        \item a file in the home directory: \plaintt{\textasciitilde/file.txt}
    \end{itemize}
\end{exampleblock}

\section{The Unix Shell}

\subsubsection{What is a shell?}

The shell is the primary interface between users and the computer's core system. When you type commands in a terminal, the shell interprets these instructions and communicates with the operating system to execute them, making complex system operations accessible through simple commands.

\vspace{-0.5em}

\begin{figure}[H]
    \centering
    \includegraphics[width=0.8\textwidth]{assets/shell.png}
    \caption{The Unix Shell}
    \label{fig:shell}
\end{figure}

\vspace{-1.5em}

\begin{definitionblock}[Shell]
    A shell is a program that provides the traditional, text-only user interface for Linux and other UNIX-like operating systems. Its primary function is to read commands that are typed into a console […] and then execute (i.e., run) them. The term shell derives its name from the fact that it is an outer layer of an operating system. A shell is an interface between the user and the internal parts of the OS (at the very core of which is the kernel). 
    \hfill \textit{\href{http://www.linfo.org/shell.html}{\textasciitilde Linfo}}
\end{definitionblock}

\subsection{Basic Commands}

A command is nothing more than a program that is executed by the shell. The usual syntax is:
\begin{codeblock}[language=bash, numbers=none]
command [options] [arguments]
\end{codeblock}

The options can be passed setting the corresponding flag e.g. \texttt{-h} for help.

\subsubsection{Navigation Commands}

It is possible to navigate and manage files or directories using the following commands:

\vspace{0.5em}

\begin{minipage}{0.6\textwidth}
    \begin{center}
        \begin{tabular}{c l l}
            \toprule
            \textbf{Command} & \textbf{Description} \\
            \midrule
        \texttt{pwd} & Print Working Directory \\
        \texttt{cd} & Change Directory \\
        \texttt{ls} & List Directory \\
        \texttt{mkdir} & Make Directory \\
        \texttt{rm} & Remove File or Directory \\
        \texttt{cp} & Copy File or Directory \\
        \texttt{mv} & Move File or Directory \\
        \texttt{touch} & Create File \\
        \bottomrule
    \end{tabular}
\end{center}
\end{minipage}%
\begin{minipage}{0.40\textwidth}
% \centering
\textbf{Recursive Commands}
\vspace{0.4em}

\texttt{rm}, \texttt{cp} and many other commands can be used recursively into a folder using the flag \texttt{-r} (recursive).

\vspace{0.8em}

\textbf{Help Flag} 

\vspace{0.4em}

Many commands support the \texttt{--help} flag to display information about the command and its usage.
\end{minipage}

\begin{tipsblock}[Hidden Files]
    Hidden files are files that are not shown by default when listing the contents of a directory. They are usually marked with a dot (.) at the beginning of their name. e.g. \plaintt{.config}
\end{tipsblock}

\subsubsection{Searching Tools}

Unix-like systems provide powerful tools for searching files and content. These tools often support regular expressions (regex), which are patterns used to match character combinations in text.

\subsubsection{File Search Commands}

\begin{center}
    \renewcommand{\arraystretch}{1.1}
    \begin{tabular}{c l l}
        \toprule
        \textbf{Command} & \textbf{Description} & \textbf{Example} \\
        \midrule
        \texttt{find} & Search for files and directories & \texttt{find /home -name "*.txt"} \\
        \texttt{locate} & Fast file search using database & \texttt{locate filename} \\
        \texttt{grep} & Search text within files & \texttt{grep "pattern" file.txt} \\
        \texttt{which} & Find location of executable & \texttt{which gcc} \\
        \texttt{whereis} & Find binary, source, and manual pages & \texttt{whereis python} \\
        \bottomrule
    \end{tabular}
\end{center}

\subsubsection{Wildcards and Pattern Matching}

Wildcards are special characters that represent one or more characters in a filename or text string:

\begin{center}
    \renewcommand{\arraystretch}{1.1}
    \begin{tabular}{c l l}
        \toprule
        \textbf{Character} & \textbf{Name} & \textbf{Description} \\
        \midrule
        \texttt{*} & asterisk & Matches zero or more characters \\
        \texttt{?} & question mark & Matches exactly one character \\
        \texttt{[]} & brackets & Matches any single character within brackets \\
        \texttt{[!]} & negated brackets & Matches any character not in brackets \\
        \bottomrule
    \end{tabular}
\end{center}

\begin{exampleblock}[Wildcard Examples]
    \begin{itemize}
        \item \texttt{*.txt}: matches all files ending with .txt
        \item \texttt{file?.c}: matches file1.c, file2.c, but not file10.c
        \item \texttt{[abc]*}: matches files starting with a, b, or c
        \item \texttt{[!0-9]*}: matches files not starting with digits
    \end{itemize}
\end{exampleblock}

\subsubsection{Operators}

Commands can be combined using operators to manipulate input and output streams:

\begin{center}
    \renewcommand{\arraystretch}{1.1}
    \begin{tabular}{c l l}
        \toprule
        \textbf{Operator} & \textbf{Name} & \textbf{Description} \\
        \midrule
        \texttt{>} & output redirection & Redirects standard output to a file, overwriting its contents\\
        \texttt{>>} & append redirection & Redirects standard output, appending it to the end of a file\\
        \texttt{|} & pipe & Passes the output of a command as input to another\\
        \texttt{\&\&} & logical AND & Executes the next command only if the previous one succeeds\\
        \texttt{||} & logical OR & Executes the next command only if the previous one fails\\
        \texttt{;} & separator & Executes commands sequentially, regardless of their outcome\\
        \bottomrule
    \end{tabular}
\end{center}

\subsection{Environment Variables}

Environment variables are dynamic values that affect the behavior of processes and programs running on the system. They provide a way to pass configuration information to applications without hardcoding values into the program. These variables are stored in the system's environment and can be accessed by any process.

\begin{center}
    \renewcommand{\arraystretch}{1.1}
    \begin{tabular}{c l l}
        \toprule
        \textbf{Variable} & \textbf{Description} & \textbf{Example} \\
        \midrule
        \texttt{PATH} & Colon-separated list of dirs to search for executables & \texttt{/usr/bin:/bin:/usr/sbin} \\
        \texttt{HOME} & Path to the user's home directory & \texttt{/home/username} \\
        \texttt{USER} & Current username & \texttt{username} \\
        \texttt{SHELL} & Path to the current shell & \texttt{/bin/bash} \\
        \texttt{PWD} & Current working directory & \texttt{/home/username/Desktop} \\
        \bottomrule
    \end{tabular}
\end{center}

To view environment variables, use the \texttt{env} command or \texttt{echo \$VARIABLE\_NAME}. To set a variable temporarily, use \texttt{export VARIABLE\_NAME=value}. For permanent changes, modify configuration files like \texttt{.bashrc} or \texttt{.bash\_profile}.

\subsection{File Permissions}

Unix-like systems use a permission system to control access to files and directories. Each file has three types of permissions: read (r), write (w), and execute (x). These permissions are assigned to three categories of users: owner (user), group, and others.

\subsubsection{Permission Structure}

File permissions are displayed using a 10-character string where the first character indicates the file type, and the remaining nine characters represent permissions for owner, group, and others:

\begin{center}
    \renewcommand{\arraystretch}{1.1}
    \begin{tabular}{c l l}
        \toprule
        \textbf{Character} & \textbf{Position} & \textbf{Meaning} \\
        \midrule
        \texttt{-} & 1st & Regular file \\
        \texttt{d} & 1st & Directory \\
        \texttt{r} & 2nd, 5th, 8th & Read permission \\
        \texttt{w} & 3rd, 6th, 9th & Write permission \\
        \texttt{x} & 4th, 7th, 10th & Execute permission \\
        \bottomrule
    \end{tabular}
\end{center}

\vspace{0.5em}

For instance, \texttt{-rwxr-xr--} means:

\begin{minipage}[t]{0.5\linewidth}
    \begin{itemize}
        \item Regular file (-)
        \item Owner: read, write, execute (rwx)
    \end{itemize}
\end{minipage}%
\begin{minipage}[t]{0.5\linewidth}
    \begin{itemize}
        \item Group: read, execute (r-x)
        \item Others: read only (r--)
    \end{itemize}
\end{minipage}

\subsubsection{Octal Notation}

Permissions can be represented numerically using octal notation: 

\begin{center}
    \renewcommand{\arraystretch}{1}
    \begin{tabular}{l c c}
        (\texttt{r}) Read & \textbf{4} & \texttt{100} \\
        (\texttt{w}) Write & \textbf{2} & \texttt{010} \\
        (\texttt{x}) Execute & \textbf{1} & \texttt{001} \\
    \end{tabular}
\end{center}

By summing these values for each user category (owner, group, and others), we get a three-digit code. For example, \texttt{rwx} permissions translate to $4+2+1=7$, and \texttt{r-x} to $4+0+1=5$. Thus, a full permission string like \texttt{rwxr-xr-x} can be concisely represented as 755.

Common combinations include: \textbf{755} (\texttt{rwxr-xr-x}), \textbf{644} (\texttt{rw-r--r--}), and \textbf{600} (\texttt{rw-------}).

\subsubsection{Permission Commands}

\begin{center}
    \renewcommand{\arraystretch}{1.1}
    \begin{tabular}{c l l}
        \toprule
        \textbf{Command} & \textbf{Description} & \textbf{Example} \\
        \midrule
        \texttt{chmod} & Change file permissions & \texttt{chmod 755 script.sh} \\
        \texttt{chown} & Change file ownership & \texttt{chown user:group file.txt} \\
        \texttt{chgrp} & Change group ownership & \texttt{chgrp staff file.txt} \\
        \texttt{umask} & Set default permissions & \texttt{umask 022} \\
        \bottomrule
    \end{tabular}
\end{center}

\todo{notes about file viewing and text processing, few words about multiprocessing and context switching}
