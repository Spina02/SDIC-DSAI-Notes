\chapter{Git, GitHub and SSH}

\section{Git and GitHub}

Git is a distributed version control system that allows you to track changes to your code over time. It is a tool that is used to manage your codebase and collaborate with others.

There are different services that provide Git repositories, one of the most popular is GitHub.

GitHub is a web-based platform that provides a Git repository hosting service. It allows you to create a repository and invite others to collaborate on it. It also provides a web interface for viewing and editing the code, as well as a command line interface for interacting with the repository.

% TODO: finish this section

\section{Internet and SSH}

In this section we briefly explain how the Internet works and how to use SSH securely.

\subsection{Internet}

Computers communicate on the Internet using the TCP/IP protocol suite. An IP address (e.g. \texttt{192.168.1.1} in IPv4) identifies a host on the network; services on that host are selected by port numbers (e.g. 22 for SSH, 80/443 for HTTP/HTTPS). DNS translates human-readable domain names (e.g. \texttt{www.example.com}) into IP addresses.

Data travels in \emph{packets}. Each packet has a header with routing metadata and a \emph{payload} with application data.

\begin{table}[h!]
    \centering
    \renewcommand{\arraystretch}{1.1}
    \begin{tabular}{|c|c|}
        \hline
        \multicolumn{2}{|c|}{\textbf{Header}} \\
        \hline
        Source & IP: \texttt{192.168.1.10}, Port: \texttt{54321} \\
        Destination & IP: \texttt{172.217.22.14}, Port: \texttt{443} \\
        Protocol & \texttt{TCP} \\
        \hline
        \multicolumn{2}{|c|}{\textbf{Payload}} \\
        \hline
        \textbf{Data} & \texttt{Hello, world!} \\
        \hline
    \end{tabular}
    \caption{Simplified structure of an Internet packet}
\end{table}

\vspace{-0.5em}

\begin{observationblock}[Encryption in transit]
    If traffic is not encrypted, the payload can be intercepted and read. Encryption in transit is typically provided by TLS (for the web, HTTPS) or by SSH (for remote access and related transfers).
\end{observationblock}

A \emph{LAN} (Local Area Network) covers a limited area (home, office, school). Many LANs use private address ranges (\texttt{10.0.0.0/8}, \texttt{192.168.0.0/16}, ...) and a router that performs \emph{NAT} (Network Address Translation): it maps private addresses/ports to a public address on the Internet (often using \emph{PAT}, Port Address Translation). This mapping is done by the router and lets many private hosts share one public IP.

\subsection{SSH}

SSH (Secure Shell) is a secure protocol for remote administration and tunneling. It uses port 22 by default and supports public-key authentication.

In the \texttt{\textasciitilde/.ssh/} directory you typically find:
\begin{itemize}
    \item \texttt{id\_ed25519} and \texttt{id\_ed25519.pub}: your private and public key (recommended). Historical variants: \texttt{id\_rsa}, \texttt{id\_rsa.pub}.
    \item \texttt{known\_hosts}: fingerprints of servers you have connected to (helps prevent MITM attacks).
    \item \texttt{config}: client options for specific hosts (alias, port, user, key).
    \item On a server: \texttt{~/.ssh/authorized\_keys} lists public keys allowed to connect.
\end{itemize}

\medskip

There are multiple tools that use SSH as transport:
\begin{itemize}
    \item \texttt{ssh}: opens a remote session to run commands/shell.
    \item \texttt{scp}: copies files between local/remote hosts over SSH.
    \item \texttt{rsync}: efficiently synchronizes directories; often uses SSH as its channel (\texttt{rsync -e ssh}).
    \item \texttt{git}: version control system; can use SSH as transport for clone/push.
\end{itemize}

\medskip

Practical examples:
\begin{codeblock}[language=bash, numbers=none]
# Generate a new key pair (Ed25519 recommended)
ssh-keygen -t ed25519 -C "name.surname@example.com"

# Install your public key on the server
ssh-copy-id user@server.example.com

# Test access
ssh user@server.example.com

# File transfers
scp file.txt user@server.example.com:/path/destination/
rsync -avz -e ssh folder/ user@server.example.com:/path/destination/

# Using Git over SSH
git clone git@github.com:organization/repository.git
\end{codeblock}

\begin{tipsblock}[Graphical clients and X forwarding]
Beyond the command line, graphical SSH clients exist. On Unix-like systems you can forward X11 applications with \plaintt{ssh -X} (if both server and client allow it); on Windows you can use clients like OpenSSH with an installed X server.
\end{tipsblock}