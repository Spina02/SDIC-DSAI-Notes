\chapter{Hydrodynamics}

In astrophysics, \textbf{numerical hydrodynamics} plays a crucial role in modeling the formation of baryonic structures within dark matter potential wells during non-linear evolutionary phases.The complexity of realistic astrophysical flows (involving turbulence, shocks, magnetic fields, and radiative processes) demands numerical approaches.

The field encompasses two primary computational paradigms. \textbf{Eulerian methods} track the flow of gas and energy through fixed spatial grids, with derivatives computed at stationary points in space; this approach excels at capturing shocks and complex geometries. \textbf{Lagrangian methods}, by contrast, follow individual fluid elements as they move through space, with derivatives calculated in a co-moving coordinate system; these methods provide excellent mass conservation and natural resolution adaptivity in regions of high density.

The \textbf{Euler equations} form the fundamental framework for describing inviscid fluid motion. This system of coupled PDEs governs the conservation of mass, momentum, and energy:
$$
\begin{array}{rlcl}
    \dfrac{d\vec{v}}{dt} &= - \dfrac{\nabla P}{\rho} && \text{(momentum conservation)} \\[0.5em]
    \dfrac{du}{dt} &= - \dfrac{P}{\rho} \nabla \cdot \vec{v} && \text{(energy conservation)} \\[0.5em]
    \dfrac{d\rho}{dt} + \rho \nabla \cdot \vec{v} &= 0 && \text{(mass conservation)} \\[0.5em]
    P &= (\gamma - 1) \rho u && \text{(equation of state)}
\end{array}
$$
where $\vec{v}$ is the fluid velocity field, $P$ represents pressure, $u$ denotes specific internal energy, $\rho$ is the density, and $\gamma$ is the adiabatic index (equal to $5/3$ for an ideal monoatomic gas).

The total (Lagrangian) derivative captures how quantities change following a fluid element:
$$
\dfrac{d}{dt} = \dfrac{dx^i}{dt} \dfrac{\partial}{\partial x^i} + \dfrac{\partial}{\partial t} = \vec v \cdot \nabla + \dfrac{\partial}{\partial t}
$$
This combines the local time rate of change with the advective transport due to fluid motion.

From the principle of mass conservation in Eulerian form, $\partial \rho / \partial t + \nabla \cdot (\rho \vec{v}) = 0$, we can derive the Lagrangian continuity equation. Applying the chain rule $d\rho/dt = \vec v \cdot \nabla \rho + \partial \rho / \partial t$ and expanding the divergence yields:
$$
\dfrac{d\rho}{dt} = - \rho \nabla \cdot \vec{v}
$$
This shows that density changes are directly related to the divergence of the velocity field: compression ($\nabla \cdot \vec{v} < 0$) increases density, while expansion ($\nabla \cdot \vec{v} > 0$) decreases it.

The energy equation derives from the first law of thermodynamics, $dU = dQ - P\,dV$. For adiabatic processes ($dQ = 0$), we substitute the specific volume relationship $dV \rightarrow d(1/\rho) = -d\rho/\rho^2$ and apply the continuity equation:
$$
du = \dfrac{P}{\rho^2} d\rho \quad\longrightarrow\quad \dfrac{du}{dt} = \dfrac{P}{\rho^2} \dfrac{d\rho}{dt} = - \dfrac{P}{\rho} \nabla \cdot \vec{v}
$$
This demonstrates how internal energy changes are coupled to the compression or expansion of the fluid through the pressure-volume work term.

\section{Smoothed Particle Hydrodynamics (SPH)}

\textbf{Smoothed Particle Hydrodynamics} (SPH) is a Lagrangian method that represents the fluid as a collection of discrete particles, each carrying hydrodynamic properties such as mass, velocity, and internal energy. The fundamental principle is that any quantity at a given point is not taken as a sharp value, but rather \emph{smoothed} over the contributions of neighboring particles within a characteristic length scale $h$, called the \textbf{smoothing length}.

Particles evolve under the Euler equations using these smoothed quantities. At each timestep, positions and velocities are updated, the smoothed fields are recomputed from the new configuration, and the cycle repeats. This approach offers several advantages: automatic adaptivity (particles naturally concentrate in high-density regions), exact conservation of mass and momentum, and straightforward handling of free boundaries.

\subsection{Smoothing and the kernel}

The smoothed estimate of a function $f(\vec{r})$ is defined through a convolution with a \textbf{kernel} (or window function) $W$:
$$
\tilde{f}_h(\vec{r}) = \int f(\vec{r}') \, W(\vec{r} - \vec{r}', h) \, d^3r'
$$
The kernel must satisfy several properties. First, \textbf{normalization}: $\int W(\vec{r}, h) \, d^3r = 1$, ensuring that in the limit $h \to 0$ the original function is recovered, i.e.\ $\lim_{h \to 0} \tilde{f}_h(\vec{r}) = f(\vec{r})$. Second, the kernel should be \textbf{radially symmetric} to conserve angular momentum. Third, it should have \textbf{compact support} (vanishing beyond some radius) to limit computational cost and avoid unphysical long-range interactions.

To convert the integral into a sum over particles, we multiply and divide by the density field:
$$
\tilde{f}_h(\vec{r}) = \int \frac{f(\vec{r}')}{\rho(\vec{r}')} \, W(\vec{r} - \vec{r}', h) \, \rho(\vec{r}') \, d^3r'
$$
Since each particle $b$ occupies a volume element $d^3r' \approx m_b / \rho_b$, we obtain:
$$
f(\vec{r}) \simeq \sum_b \frac{m_b}{\rho_b} \, f_b \, W(\vec{r} - \vec{r}_b, h)
$$
Setting $f = \rho$ yields the fundamental SPH density estimator:
$$
\boxed{\rho(\vec{r}) = \sum_b m_b \, W(\vec{r} - \vec{r}_b, h)}
$$
This formula, central to SPH, automatically satisfies the continuity equation when particle masses are conserved. Note that the particle itself contributes to its own density estimate (typically as the dominant term, since $W$ peaks at zero separation).

Since derivatives act on smooth functions, we can transfer the gradient operator onto the kernel:
$$
\nabla f(\vec{r}) = \sum_b \frac{m_b}{\rho_b} \, f_b \, \nabla W(\vec{r} - \vec{r}_b, h)
$$
This naive form, however, does not vanish for constant $f$. To enforce this property, we use a symmetrized formulation. Introducing an auxiliary field $\Phi$, the identity
$$
\frac{\partial A}{\partial x} = \frac{1}{\Phi} \left( \frac{\partial (\Phi A)}{\partial x} - A \frac{\partial \Phi}{\partial x} \right)
$$
leads to the SPH gradient:
$$
\left( \frac{\partial A}{\partial x} \right)_a = \frac{1}{\Phi_a} \sum_b m_b \frac{\Phi_b}{\rho_b} (A_b - A_a) \frac{\partial W_{ab}}{\partial x_a}
$$
where $W_{ab} \equiv W(|\vec{r}_a - \vec{r}_b|, h)$. This form vanishes identically when $A$ is constant. Common choices are $\Phi = 1$ and $\Phi = \rho$, giving respectively:
$$
\left( \frac{\partial A}{\partial x} \right)_a = \sum_b \frac{m_b}{\rho_b} (A_b - A_a) \frac{\partial W_{ab}}{\partial x_a}
\qquad \text{or} \qquad
\left( \frac{\partial A}{\partial x} \right)_a = \frac{1}{\rho_a} \sum_b m_b (A_b - A_a) \frac{\partial W_{ab}}{\partial x_a}
$$

For the divergence (needed in the continuity equation), analogous expressions hold:
$$
\frac{d\rho_a}{dt} = \sum_b m_b \, \vec{v}_{ab} \cdot \nabla_a W_{ab}
\qquad \text{or} \qquad
\frac{d\rho_a}{dt} = \rho_a \sum_b \frac{m_b}{\rho_b} \, \vec{v}_{ab} \cdot \nabla_a W_{ab}
$$

Differentiating the kernel twice amplifies numerical noise. A more stable approximation for the Laplacian is:
$$
(\nabla^2 f)_a = 2 \sum_b \frac{m_b}{\rho_b} (f_a - f_b) \frac{W_{ab}}{r_{ab}}
$$
Despite this, higher-order derivatives remain problematic in SPH, making diffusion-type equations (thermal conduction, physical viscosity) challenging to handle accurately.

To understand the accuracy of the smoothing procedure, we expand $f(\vec{r}')$ in a Taylor series about $\vec{r}$:
$$
f(\vec{r}') = \sum_{k=0}^\infty \dfrac{f^{(k)}(\vec{r})}{k!} (\vec{r}' - \vec{r})^k
$$
Substituting into the convolution integral and exploiting the symmetry properties of the kernel, odd-order terms vanish, yielding:
$$
\tilde{f}_h(\vec{r}) = f(\vec{r}) + C h^2 + O(h^4)
$$
where the coefficient $C$ contains second-order derivatives of the function. This means that constant and linear functions are reproduced exactly, while the leading error is second-order in the smoothing length.

In practice, the \textbf{cubic spline kernel} is the most widely used choice, offering a good balance between accuracy and computational efficiency:
$$
W(q) = \dfrac{\sigma}{h^d}
\begin{cases}
    1 - \dfrac{3}{2} q^2 + \dfrac{3}{4} q^3 & \text{if } 0 \leq q \leq 1 \\[0.3em]
    \dfrac{1}{4} (2 - q)^3 & \text{if } 1 < q \leq 2 \\[0.3em]
    0 & \text{if } q > 2
\end{cases}
$$
where $q = r_{ab}/h$ is the dimensionless separation, $d$ is the number of spatial dimensions, and $\sigma$ is a normalization constant: $\sigma = 2/3$ in 1D, $\sigma = 10/(7\pi)$ in 2D, and $\sigma = 1/\pi$ in 3D. The compact support (vanishing for $q > 2$) limits neighbor searches to within a radius of $2h$.

The \textbf{Gaussian kernel}, $W(r,h) \propto \exp(-r^2/h^2)$, is sometimes used for theoretical analysis due to its smoothness, but its non-compact support makes it computationally impractical. Higher-order kernels (quintic splines, Wendland functions) offer improved accuracy but at greater computational cost; they are becoming more common in modern implementations.

\begin{tipsblock}[Self-contribution]
    Each particle contributes to its own density estimate, typically as the dominant term since the kernel peaks at zero separation.
\end{tipsblock}

\subsection{SPH equations of motion}

With the discretization machinery in place, we can now write the SPH form of the Euler equations. The momentum equation becomes:
$$
\dfrac{d\vec{v}_a}{dt} = -\sum_b m_b \left( \dfrac{P_a}{\rho_a^2} + \dfrac{P_b}{\rho_b^2} \right) \nabla_a W_{ab}
$$
This symmetric form ensures exact conservation of linear and angular momentum. The energy equation takes the form:
$$
\dfrac{du_a}{dt} = \dfrac{1}{2} \sum_b m_b \left( \dfrac{P_a}{\rho_a^2} + \dfrac{P_b}{\rho_b^2} \right) \vec{v}_{ab} \cdot \nabla_a W_{ab}
$$
where the factor of $1/2$ arises from symmetrization. The system is closed by the equation of state $P = (\gamma - 1) \rho u$ and the density estimator derived earlier.

The standard SPH formulation does not explicitly conserve entropy. To address this, one can reformulate the equations using an \textbf{entropic function} $A(s)$ such that:
$$
P = A(s) \rho^\gamma
$$
where $s$ is the specific entropy. From the equation of state for an ideal gas, the specific internal energy becomes:
$$
u = \dfrac{A(s)}{\gamma - 1} \rho^{\gamma - 1}
$$
Integrating the evolution equation for $A(s)$ (which changes only due to irreversible processes like shocks) is then equivalent to integrating the energy equation while explicitly tracking entropy generation. This formulation is particularly useful when accurate entropy conservation is important, such as in cosmological simulations.

\subsection{Shock tube test}

The \textbf{shock tube} (or Sod problem) is the canonical test case for hydrodynamical codes. Consider a one-dimensional tube divided by a membrane at its center, with two fluid regions having different initial conditions. The left region typically has higher pressure and density, while the right region has lower values of both. Both regions are initially at rest. Typical initial conditions (in SI units) are:
\begin{center}
\begin{tabular}{lcc}
    \toprule
    \textbf{Quantity} & \textbf{Left zone} & \textbf{Right zone} \\
    \midrule
    Pressure $P$ & $10^5$ Pa & $10^4$ Pa \\
    Density $\rho$ & 1 kg/m$^3$ & 0.125 kg/m$^3$ \\
    Velocity $v$ & 0 m/s & 0 m/s \\
    \bottomrule
\end{tabular}
\end{center}

At time $t = 0$, the membrane is instantaneously removed, creating a classical \textbf{Riemann problem}. The sudden pressure imbalance drives a complex wave pattern that propagates through the tube. Three distinct phenomena emerge:

\textbf{Rarefaction fan.} On the high-pressure side, a rarefaction wave propagates backward into the undisturbed fluid, creating a smooth transition region where density, pressure, and velocity change continuously. The rarefaction fan spreads out over time, producing the characteristic curved profile in the density distribution.

\textbf{Contact discontinuity.} This interface separates the two original fluids. While pressure and velocity are continuous across this boundary, density exhibits a jump discontinuity. The contact discontinuity moves at the local fluid velocity.

\textbf{Shock wave.} On the low-pressure side, a sharp discontinuity propagates forward into the undisturbed fluid. Across the shock, all fluid properties change abruptly; the shock compresses and heats the fluid it encounters.

The resulting flow pattern consists of four distinct regions separated by these three wave types. This problem serves as an excellent test case because it contains both smooth (rarefaction) and discontinuous (shock, contact) solutions, challenging the numerical scheme's ability to handle different types of flow features accurately.

\begin{warningblock}[Euler equations and shocks]
    The Euler equations, being inviscid, do not contain the physical dissipation mechanisms that operate within real shock fronts. Without additional treatment, numerical solutions develop spurious oscillations near discontinuities.
\end{warningblock}

\subsection{Artificial viscosity}

To capture shocks correctly in SPH, we introduce an \textbf{artificial viscosity} term $\Pi_{ab}$ that adds dissipation only in regions of compression. The pressure terms in the momentum equation are modified as:
$$
\left(\dfrac{P_a}{\rho_a^2} + \dfrac{P_b}{\rho_b^2}\right) 
\quad \longrightarrow \quad
\left(\dfrac{P_a}{\rho_a^2} + \dfrac{P_b}{\rho_b^2} + \Pi_{ab}\right)
$$

The artificial viscosity is activated only when particles approach each other:
$$
\Pi_{ab} = \begin{cases}
\dfrac{-\alpha \bar{c}_{s,ab} \mu_{ab} + \beta \mu_{ab}^2}{\bar{\rho}_{ab}} & \text{if } \vec{x}_{ab} \cdot \vec{v}_{ab} < 0 \\[0.5em]
0 & \text{otherwise}
\end{cases}
$$
where the auxiliary quantity $\mu_{ab}$ is defined as:
$$
\mu_{ab} = \dfrac{\bar{h}_{ab} \, \vec{x}_{ab} \cdot \vec{v}_{ab}}{|\vec{x}_{ab}|^2 + \varepsilon \bar{h}_{ab}^2}
$$
Here, $\vec{x}_{ab} = \vec{r}_a - \vec{r}_b$ and $\vec{v}_{ab} = \vec{v}_a - \vec{v}_b$ are the relative position and velocity, $\bar{h}_{ab}$ and $\bar{\rho}_{ab}$ are arithmetic averages of the smoothing lengths and densities, $\bar{c}_{s,ab}$ is the average sound speed, and $\varepsilon \approx 0.01$ prevents singularities when particles are very close. Typical parameter values are $\alpha = 1$ and $\beta = 2$.

The linear term (proportional to $\alpha$) provides bulk viscosity that smears shocks over a few smoothing lengths, while the quadratic term (proportional to $\beta$) prevents particle interpenetration in high Mach number shocks.

A drawback of artificial viscosity is that it also damps legitimate shear flows. The \textbf{Balsara switch} reduces this spurious dissipation by weighting the viscosity with a factor that distinguishes between compression and shear:
$$
f_a = \dfrac{|\langle \nabla \cdot \vec{v} \rangle_a|}{|\langle \nabla \cdot \vec{v} \rangle_a| + |\langle \nabla \times \vec{v} \rangle_a| + 10^{-4} c_{s,a}/h_a}
$$
This factor approaches unity in pure compression (where $\nabla \times \vec{v} = 0$) and vanishes in pure shear (where $\nabla \cdot \vec{v} = 0$). The modified viscosity becomes:
$$
\Pi_{ab}' = \Pi_{ab} \cdot \frac{f_a + f_b}{2}
$$
The small term in the denominator prevents division by zero in quiescent regions.

\subsection{Time integration}

The SPH equations of motion must be integrated in time using a stable and accurate scheme. The \textbf{leapfrog} (or kick-drift-kick) integrator is the standard choice, offering second-order accuracy, time-reversibility, and excellent energy conservation for Hamiltonian systems.

The timestep must satisfy the \textbf{Courant-Friedrichs-Lewy (CFL) condition}, which ensures that information does not propagate more than one resolution element per timestep. For SPH, this becomes:
$$
\Delta t < \dfrac{h}{c_s}
$$
where $c_s = \sqrt{\gamma P / \rho}$ is the local sound speed. In practice, a safety factor $K \approx 0.1$--$0.15$ is applied, and the global timestep is the minimum over all particles:
$$
\Delta t = K \cdot \min_a \left( \dfrac{h_a}{c_{s,a} + 0.6(\alpha c_{s,a} + 2 \max_b |\mu_{ab}|)} \right)
$$
The additional terms in the denominator account for the artificial viscosity contribution to signal propagation.

The leapfrog integrator proceeds in three phases per timestep:

\textbf{Kick} (half-step): Update velocities and internal energies by half a timestep using the current accelerations and heating rates:
\begin{align*}
    \vec{v}^{n+1/2} &= \vec{v}^n + \frac{\Delta t}{2} \vec{a}^n \\
    u^{n+1/2} &= u^n + \frac{\Delta t}{2} \dot{u}^n
\end{align*}

\textbf{Drift} (full step): Update positions by a full timestep using the half-step velocities:
$$
\vec{r}^{n+1} = \vec{r}^n + \Delta t \, \vec{v}^{n+1/2}
$$

After the drift, recompute densities, pressures, and accelerations at the new positions.

\textbf{Kick} (half-step): Complete the velocity and energy update using the new accelerations:
\begin{align*}
    \vec{v}^{n+1} &= \vec{v}^{n+1/2} + \frac{\Delta t}{2} \vec{a}^{n+1} \\
    u^{n+1} &= u^{n+1/2} + \frac{\Delta t}{2} \dot{u}^{n+1}
\end{align*}

The complete SPH simulation loop can be summarized as follows:

\begin{enumerate}
    \item \textbf{Initialization}: Set up initial conditions (positions, velocities, masses, internal energies); compute initial densities and pressures.
    \item \textbf{Force computation}: Find neighbors within $2h$; compute accelerations and energy rates.
    \item \textbf{Timestep}: Determine $\Delta t$ from the Courant condition.
    \item \textbf{First kick}: Advance velocities and energies by $\Delta t/2$.
    \item \textbf{Drift}: Advance positions by $\Delta t$.
    \item \textbf{Density update}: Recompute densities and pressures from new positions.
    \item \textbf{Force update}: Recompute accelerations and energy rates.
    \item \textbf{Second kick}: Complete velocity and energy update.
    \item \textbf{Output}: Write snapshot if needed.
    \item \textbf{Loop}: Return to step 3 until final time is reached.
\end{enumerate}

\begin{figure}[H]
    \centering
    \includegraphics[width=0.7\textwidth]{assets/sph-scheme.png}
    \caption{Schematic representation of the SPH simulation cycle.}
    \label{fig:sph-scheme}
\end{figure}

\begin{observationblock}[Neighbor search]
    Steps 2, 6, and 7 require finding all particles within distance $2h$ of each particle. For large $N$, a naive $O(N^2)$ search becomes prohibitive. Tree-based algorithms (such as octrees or kd-trees) reduce this to $O(N \log N)$, making SPH practical for millions of particles.
\end{observationblock}

\subsection{Variable smoothing length}

In many astrophysical applications, the density contrast can span many orders of magnitude. Using a fixed smoothing length $h$ would either under-resolve dense regions or waste computational resources in voids. The solution is to let $h$ vary with the local density, typically maintaining a roughly constant number of neighbors $N_{\text{ngb}}$:
$$
h_a \propto \left( \frac{m_a}{\rho_a} \right)^{1/d}
$$
where $d$ is the number of spatial dimensions. A common choice is $N_{\text{ngb}} \approx 32$--$64$ in 3D.

When $h$ varies spatially, the SPH equations require correction terms (often called ``grad-h'' terms) to maintain energy conservation. These arise because the kernel normalization now depends on position. Modern SPH implementations include these corrections, though for pedagogical purposes a constant $h$ simplifies the formulation considerably.

\begin{tipsblock}[Choosing $h$]
    For a simple 1D simulation with $N$ particles over a domain of length $L$, setting $h = 32 L / N$ ensures each particle has approximately 64 neighbors within the kernel support radius $2h$.
\end{tipsblock}